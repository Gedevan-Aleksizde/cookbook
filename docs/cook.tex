% Options for packages loaded elsewhere
\PassOptionsToPackage{unicode}{hyperref}
\PassOptionsToPackage{hyphens}{url}
\PassOptionsToPackage{dvipsnames,svgnames*,x11names*}{xcolor}
%
\documentclass[
  lualatex,ja=standard,a4paper,jafont=noto-otf]{bxjsreport}
\usepackage{lmodern}
\usepackage{amssymb,amsmath}
\usepackage{ifxetex,ifluatex,ifpdf,ifthen}
\ifnum 0\ifxetex 1\fi\ifluatex 1\fi=0 % if pdftex
  \usepackage[T1]{fontenc}
  \usepackage[utf8]{inputenc}
  \usepackage{textcomp} % provide euro and other symbols
\else % if luatex or xetex
  \usepackage{unicode-math}
  \defaultfontfeatures{Scale=MatchLowercase}
  \defaultfontfeatures[\rmfamily]{Ligatures=TeX,Scale=1}
  \setmainfont[]{DejaVu Serif}
  \setsansfont[]{DejaVu Sans}
\fi

% japanese font setting
% if the preset is specified
\ifxetex
  \usepackage{xeCJKfntef}
\fi
\ifluatex
  \usepackage{luatexja}
  \usepackage[,noto-otf]{luatexja-preset}
\fi

  \IfFileExists{pxrubrica.sty}{\usepackage{pxrubrica}}{}
\ifluatex
  \ltjsetparameter{%
    jacharrange={-2,-3},
    alxspmode={`/,allow},
    alxspmode={`#,allow},
    alxspmode={92,allow}
  }
\fi


% Use upquote if available, for straight quotes in verbatim environments
\IfFileExists{upquote.sty}{\usepackage{upquote}}{}
\ifpdf
  \IfFileExists{microtype.sty}{% use microtype if available
    \usepackage[]{microtype}
    \UseMicrotypeSet[protrusion]{basicmath} % disable protrusion for tt fonts
  }{}
\fi
\makeatletter
\@ifundefined{KOMAClassName}{% if non-KOMA class
  \IfFileExists{parskip.sty}{%
    \usepackage{parskip}
  }{% else
    \setlength{\parindent}{0pt}
    \setlength{\parskip}{6pt plus 2pt minus 1pt}}
}{% if KOMA class
  \KOMAoptions{parskip=half}}
\makeatother
\usepackage{xcolor}
\IfFileExists{xurl.sty}{\usepackage{xurl}}{} % add URL line breaks if available
\IfFileExists{bookmark.sty}{\usepackage{bookmark}}{\usepackage{hyperref}}
\hypersetup{
  pdftitle={ザ・クックブック},
  pdfauthor={ill-identified},
  colorlinks=true,
  linkcolor=blue,
  filecolor=Maroon,
  citecolor=blue,
  urlcolor=magenta,
  pdfcreator={LaTeX via pandoc}}
\urlstyle{same} % disable monospaced font for URLs
% for compatible with kableExtra package functions.
\usepackage{longtable,booktabs,dcolumn}
%\usepackage{longtable,booktabs,dcolumn,array,multirow,wrapfig,float,colortbl,pdflscape,tabu,threeparttable,threeparttablex,makecell}
% Correct order of tables after \paragraph or \subparagraph
\usepackage{etoolbox}
\makeatletter
\patchcmd\longtable{\par}{\if@noskipsec\mbox{}\fi\par}{}{}
\makeatother
% Allow footnotes in longtable head/foot
\IfFileExists{footnotehyper.sty}{\usepackage{footnotehyper}}{\usepackage{footnote}}
\makesavenoteenv{longtable}
\usepackage{graphicx,grffile}
\makeatletter
\def\maxwidth{\ifdim\Gin@nat@width>\linewidth\linewidth\else\Gin@nat@width\fi}
\def\maxheight{\ifdim\Gin@nat@height>\textheight\textheight\else\Gin@nat@height\fi}
\makeatother
% Scale images if necessary, so that they will not overflow the page
% margins by default, and it is still possible to overwrite the defaults
% using explicit options in \includegraphics[width, height, ...]{}
\setkeys{Gin}{width=\maxwidth,height=\maxheight,keepaspectratio}
% Set default figure placement to htbp
\makeatletter
\def\fps@figure{htbp}
\makeatother
\setlength{\emergencystretch}{3em} % prevent overfull lines
\providecommand{\tightlist}{%
  \setlength{\itemsep}{0pt}\setlength{\parskip}{0pt}}
\setcounter{secnumdepth}{5}

\ifthenelse{\equal{fancy}{fancy}}{
  \usepackage{fancyhdr}
}{}
\pagestyle{fancy}

% compatible mukti-columns macro
% by "R Markdown Cookbook" Sec. 5.8
\newenvironment{cols}[1][]{}{}
\newenvironment{col}[1]{\begin{minipage}{#1}\ignorespaces}{%
\end{minipage}
\ifhmode\unskip\fi
\aftergroup\useignorespacesandallpars}
\def\useignorespacesandallpars#1\ignorespaces\fi{%
#1\fi\ignorespacesandallpars}
\makeatletter
\def\ignorespacesandallpars{%
\@ifnextchar\par
{\expandafter\ignorespacesandallpars\@gobble}%
{}%
}
\makeatother
%-------


% ---- XeLaTeX 専用のあれ ----
\ifxetex
  \usepackage{letltxmacro}
  \setlength{\XeTeXLinkMargin}{1pt}
  \LetLtxMacro\SavedIncludeGraphics\includegraphics
  \def\includegraphics#1#{% #1 catches optional stuff (star/opt. arg.)
    \IncludeGraphicsAux{#1}%
  }%
  \newcommand*{\IncludeGraphicsAux}[2]{%
    \XeTeXLinkBox{%
      \SavedIncludeGraphics#1{#2}%
    }%
  }%
\fi

% ---- custom blocks ----
\makeatletter
\newenvironment{kframe}{%
\medskip{}
\setlength{\fboxsep}{.8em}
 \def\at@end@of@kframe{}%
 \ifinner\ifhmode%
  \def\at@end@of@kframe{\end{minipage}}%
  \begin{minipage}{\columnwidth}%
 \fi\fi%
 \def\FrameCommand##1{\hskip\@totalleftmargin \hskip-\fboxsep
 \colorbox{shadecolor}{##1}\hskip-\fboxsep
     % There is no \\@totalrightmargin, so:
     \hskip-\linewidth \hskip-\@totalleftmargin \hskip\columnwidth}%
 \MakeFramed {\advance\hsize-\width
   \@totalleftmargin\z@ \linewidth\hsize
   \@setminipage}}%
 {\par\unskip\endMakeFramed%
 \at@end@of@kframe}
\makeatother

\makeatletter
\@ifundefined{Shaded}{
}{\renewenvironment{Shaded}{\begin{kframe}}{\end{kframe}}}
\makeatother


% --- custom blocks ---

% ---- redefine quote format as modern
\setlength{\fboxsep}{.8em}
\usepackage{framed}
\definecolor{quotebarcolor}{rgb}{0.2,0.2,0.2}
\renewenvironment{quote}{\def\FrameCommand{{\color{quotebarcolor}{\vrule width 3pt}}\hspace{10pt}}\MakeFramed{\advance\hsize-\width\FrameRestore}}{\endMakeFramed}
% ----
% ---- tcolobox settings by the Cookbook Sec.9.6.2
\usepackage{tcolorbox}
% \newenvironment{blackbox}{\definecolor{shadecolor}{rgb}{0, 0, 0}\color{white}\begin{shaded}}{\end{shaded}}
% \newtcolorbox{blackbox}{colback=black,colframe=orange,coltext=white,boxsep=5pt,arc=4pt}
\newtcolorbox{greyblock}{colback=gray!20,colframe=orange,coltext=black,boxsep=5pt,arc=4pt}
\newenvironment{infobox}[1]{\begin{itemize}\renewcommand{\labelitemi}{\raisebox{-.7\height}[0pt][0pt]{%
  {\setkeys{Gin}{width=3em,keepaspectratio}\includegraphics{_latex/_img/#1}}}}
  \setlength{\fboxsep}{1em}
  \begin{greyblock}
  \item
  }{\end{greyblock}\end{itemize}
}
% ----

% for block/block2 engine
\newenvironment{memo}{\begin{infobox}{memo}}{\end{infobox}}
\newenvironment{caution}{\begin{infobox}{caution}}{\end{infobox}}
\newenvironment{important}{\begin{infobox}{important}}{\end{infobox}}
\newenvironment{tip}{\begin{infobox}{tip}}{\end{infobox}}
\newenvironment{warning}{\begin{infobox}{warning}}{\end{infobox}}
% ---- custom block over ----

% --- for soft wrapping in code block
\usepackage{fvextra}
\DefineVerbatimEnvironment{Highlighting}{Verbatim}{commandchars=\\\{\},breaklines,breakanywhere}
% ----

% ---- font style-changing macro for convenient ----
\ifdefined\mcfamily
  \newcommand{\textserif}[1]{{\rmfamily\mcfamily #1}}
\else
  \newcommand{\textserif}[1]{{\rmfamily #1}}
\fi
\ifdefined\gtfamily
  \newcommand{\textsans}[1]{{\sffamily\gtfamily #1}}
\else
  \newcommand{\textsans}[1]{{\sffamily #1}}
\fi
% ----

% ---- user-defined preamble here ----
\usepackage{titlesec}
\newcommand{\sectionbreak}{\clearpage}
\newcommand{\chapterbreak}{\cleardoublepage}
\renewcommand{\chaptermark}[1]{\markboth{第\ \thechapter\ 章~#1}{}}
\renewcommand{\sectionmark}[1]{\markright{\thesection\,#1}{}}
\usepackage{makeidx}
\makeindex
\newjfontfamily\fallbacking{DejaVu Sans}
\newcommand{\ragman}{{\fallbacking ﻟﻪﯕﻤﻪﻦ}}
\usepackage{booktabs}
\usepackage{longtable}
\usepackage{array}
\usepackage{multirow}
\usepackage{wrapfig}
\usepackage{float}
\usepackage{colortbl}
\usepackage{pdflscape}
\usepackage{tabu}
\usepackage{threeparttable}
\usepackage{threeparttablex}
\usepackage[normalem]{ulem}
\usepackage{makecell}
\usepackage{xcolor}
% ---- user-defined preamble over ----

\usepackage[]{natbib}
\bibliographystyle{jecon-custom}


\title{ザ・クックブック}
\author{ill-identified\footnote{twitter ID: @ill-identified}}
\date{最終更新時刻 2021/07/11, 初公開: 2021/04/01}
\usepackage{bxtexlogo}
\colorlet{shadecolor}{gray!20}

\usepackage{fmtcount}
\ifdefined\theFancyVerbLine\renewcommand{\theFancyVerbLine}{\small \padzeroes[2]{\decimal{FancyVerbLine}}}\fi % adjust row number position
\IfFileExists{bxcoloremoji.sty}{\usepackage{bxcoloremoji}}{}




\begin{document}
\maketitle

\makeatletter
\def\emptypage@emptypage{%
    \hbox{}%
    \thispagestyle{headings}%
    \newpage%    
}%
\def\cleardoublepage{%
        \clearpage%
        \if@twoside%
            \ifodd\c@page%
                % do nothing
            \else%
                \emptypage@emptypage%
            \fi%
        \fi%
    }%
\makeatother

{
\hypersetup{linkcolor=}
\setcounter{tocdepth}{1}
\tableofcontents
}
\listoffigures

\hypertarget{preface}{%
\chapter*{初めに}\label{preface}}
\addcontentsline{toc}{chapter}{初めに}

\begin{infobox}{memo}
本書は R 言語並びに \textbf{rmarkdown}, \textbf{bookdown},
\textbf{rmdja}パッケージなどを使って書かれています.

\end{infobox}

2020年は外出できなかったのでいろいろ作った.

プロの料理人でも料理道の家元でもないので,
もちろんここで紹介しているものが唯一正統な作り方というわけではない.
うまい料理を見つけ出すのに最後に必要なのはあなたの努力だろう. しかし,
本書があなたの今後の食事を豊かにすることになんらかの形で寄与できることを願っている.

\hypertarget{ux672cux66f8ux306eux8aadux307fux65b9}{%
\chapter*{本書の読み方}\label{ux672cux66f8ux306eux8aadux307fux65b9}}
\addcontentsline{toc}{chapter}{本書の読み方}

各レシピは次のようなセクションに分かれている.

\begin{itemize}
\tightlist
\item
  「難度」は文字通り料理の難度を5段階で評価したものである.
  評価は完全に個人的な経験に基づくものなので,
  読者すべてに当てはまるとは言えない. 必要な材料の入手難度と,
  料理手順の技術的難度について記載している.
\item
  「材料」は文字通り必要な材料とその分量である.
  「適量」と書かれているのは常識の範囲でわかりそうな場合か,
  作者が正確な分量を量ってなかったり覚えてなかったりするもの.
\item
  「道具」は特筆すべき調理器具が必要な場合に記載する.
  フライパンや鍋や包丁などありふれたものは省略するが,
  例えば大きいフライパンや鍋が要求されるばあいは記載する.
\item
  「作り方」も文字通り使い方で,
  基本的には時系列に沿って手順を記載している.
  簡単な注意事項なども補足することがある.
\item
  「補足」はレシピ全般に関する参考情報や,
  他のセクションに書くには長すぎる注意事項などを書いている.
\item
  「参考資料」は文字通り
\end{itemize}

いろいろな国の伝統料理を紹介しているが,
これがどの程度「伝統的」あるいは「正統」なものなのかについては,
参考にしている資料以上の信頼性はないことを断っておく.

\hypertarget{ux51e1ux4f8b}{%
\section*{凡例}\label{ux51e1ux4f8b}}
\addcontentsline{toc}{section}{凡例}

\hypertarget{ux56fdux540dux306eux7565ux8a18}{%
\subsection*{国名の略記}\label{ux56fdux540dux306eux7565ux8a18}}
\addcontentsline{toc}{subsection}{国名の略記}

基本的に1字だが重複があるもの, 紛らわしいものは2字で表記

\begin{itemize}
\tightlist
\item
  亜塞 - アゼルバイジャン
\item
  米 - アメリカ合衆国
\item
  英 - イギリス (現時点ではスコットランド等を区別しない)
\item
  印 - インド
\item
  宇 - ウクライナ
\item
  月 - ウズベキスタン
\item
  韓 - 韓国
\item
  草 - ジョージア
\item
  捷 - チェコ
\item
  中 - 中国
\item
  日 - 日本
\item
  白露 - ベラルーシ
\item
  波 - ポーランド
\item
  露 - ロシア
\end{itemize}

\hypertarget{tenuki}{%
\chapter{手抜き料理編}\label{tenuki}}

難易度の低そうな料理, 比較的簡単に,
一度に大量に作りやすい料理をジャンル等無節操に紹介する.

\hypertarget{peperoncino}{%
\section{\texorpdfstring{ペペロンチーノ (伊: Spaghetti Aglio, Olio e
Peperoncino)\index{うどん!ペペロンチーノ}}{ペペロンチーノ (伊: Spaghetti Aglio, Olio e Peperoncino)}}\label{peperoncino}}

アンチョビすら使わない本場の手抜きレシピを紹介する

\begin{tabular}[t]{rl}
\toprule
 & 難易度\\
\midrule
材料調達 & {\fontspec{Noto Sans CJK JP} ★☆☆☆☆ }\\
調理 & {\fontspec{Noto Sans CJK JP} ★☆☆☆☆ }\\
\bottomrule
\end{tabular}

\hypertarget{ux6750ux6599}{%
\subsection{材料}\label{ux6750ux6599}}

\begin{itemize}
\tightlist
\item
  スパゲッティ 1人前
\item
  塩 大さじ1杯
\item
  水 2-3 l
\item
  オリーブオイル
\item
  ニンニク 1かけ
\item
  唐辛子 1本

  \begin{itemize}
  \tightlist
  \item
    できれば生が良い
  \end{itemize}
\item
  (オプション) イタリアンパセリの葉
\end{itemize}

\hypertarget{ux9053ux5177}{%
\subsection{道具}\label{ux9053ux5177}}

\begin{itemize}
\tightlist
\item
  茹でるための鍋
\item
  トング
\item
  フライパン

  \begin{itemize}
  \tightlist
  \item
    小さいものが使いやすい
  \end{itemize}
\end{itemize}

\hypertarget{ux4f5cux308aux65b9}{%
\subsection{作り方}\label{ux4f5cux308aux65b9}}

\begin{enumerate}
\def\labelenumi{\arabic{enumi}.}
\tightlist
\item
  水に大さじ1杯の塩を加え沸騰させる

  \begin{itemize}
  \tightlist
  \item
    下味を付けるため, これくらい多くても良い
  \end{itemize}
\item
  スパゲッティを入れて説明書にある時間くらい茹でる
\item
  フライパンを弱火にかけ, オリーブオイルをひき,
  刻んだニンニクと唐辛子を炒める

  \begin{itemize}
  \tightlist
  \item
    生の唐辛子は入手しづらい. 乾燥鷹の爪を使う場合,
    焦がさないように少し後で入れる
  \item
    見た目の理由で唐辛子の種は取ったほうが良い.
    辛いのが苦手ならなおさら.
  \end{itemize}
\item
  中火-強火にしてスパゲッティの茹で汁を少しづつ, 大さじ2-3杯くらい
\item
  水分が飛んで油と半々の粘り気のある状態になったら火を止める
\item
  スパゲッティに絡める
\item
  オプションで刻んだパセリをふりかける
\end{enumerate}

\hypertarget{ux88dcux8db3}{%
\subsubsection{補足}\label{ux88dcux8db3}}

パセリで彩りを加えると幸福追求権が多少満たされた感じがする.
昆布出汁で茹でると味にちょっと深みが出る.
しかし和風の味付けが嫌ならやらなくてもよい.

\hypertarget{puttanesca}{%
\section{\texorpdfstring{娼婦風スパゲッティ (伊: Spaghetti Alla
Puttanesca)\index{うどん!娼婦風スパゲッティ}}{娼婦風スパゲッティ (伊: Spaghetti Alla Puttanesca)}}\label{puttanesca}}

ジョジョ四部のあれ.
ペペロンチーノよりも手間を加えて美味しくしたいときに.

\begin{figure}

{\centering \includegraphics[width=1\linewidth,height=1\textheight,keepaspectratio]{img/puttanesca/finished} 

}

\caption{娼婦風スパゲッティ}\label{fig:finished-puttanesca}
\end{figure}

\begin{tabular}[t]{rl}
\toprule
 & 難易度\\
\midrule
材料調達 & {\fontspec{Noto Sans CJK JP} ★★☆☆☆ }\\
調理 & {\fontspec{Noto Sans CJK JP} ★★☆☆☆ }\\
\bottomrule
\end{tabular}

\hypertarget{ux6750ux6599-1}{%
\subsection{材料}\label{ux6750ux6599-1}}

\begin{itemize}
\item
  スパゲッティ 1人前
\item
  塩
\item
  オリーブオイル
\item
  ニンニク 1かけ
\item
  唐辛子 1個

  \begin{itemize}
  \tightlist
  \item
    注意点はペペロンチーノと同じ.
  \end{itemize}
\item
  アンチョビ 3切れ程度

  \begin{itemize}
  \tightlist
  \item
    トマトソースに塩分がない前提
  \end{itemize}
\item
  オリーブの実 (種抜き) 4-5個
\item
  トマトソース 大さじ3杯程度
\item
  (オプション) トッピング

  \begin{itemize}
  \tightlist
  \item
    パセリの葉
  \item
    ケーパーの塩漬け
  \item
    粉チーズ
  \end{itemize}
\end{itemize}

\hypertarget{ux9053ux5177-1}{%
\subsection{道具}\label{ux9053ux5177-1}}

ペペロンチーノと同じ

\hypertarget{ux4f5cux308aux65b9-1}{%
\subsection{作り方}\label{ux4f5cux308aux65b9-1}}

\begin{enumerate}
\def\labelenumi{\arabic{enumi}.}
\item
  スパゲッティを茹でるところまではペペロンチーノと同じ.
\item
  フライパンにオリーブオイルをしき, 弱火でニンニクと唐辛子を入れる.
\item
  香りが立ってきたら刻んだアンチョビも加える.
\item
  さらに少ししたらトマトソースと輪切りにしたオリーブの実を入れ,
  よく混ぜながら加熱する.
\item
  ソースが煮立ってきたら火を止める.
\item
  スパゲッティとソースを混ぜる.
\item
  皿に盛り付けてオプションのトッピングをする.
\end{enumerate}

\hypertarget{ux88dcux8db3-1}{%
\subsection{補足}\label{ux88dcux8db3-1}}

アンチョビの量は, トマトソースに味付けがないという前提である.
もし塩で味付け済みの市販のトマトソースを使うなら,
これより少なくても良い.

\hypertarget{ux30a2ux30f3ux30c1ux30e7ux30d3ux306eux8abfux9054ux65b9ux6cd5}{%
\subsubsection*{アンチョビの調達方法}\label{ux30a2ux30f3ux30c1ux30e7ux30d3ux306eux8abfux9054ux65b9ux6cd5}}
\addcontentsline{toc}{subsubsection}{アンチョビの調達方法}

スーパーでよく売っている小さい缶詰は容量あたりの単価が高い.
よく使うならスカーリア製の3000円程度の瓶詰めが価格と品質のバランスが良い.
ゴムパッキン付きの瓶なので保存容器として再利用できる.
それより安いものは身が崩れていたりするが, 味はそこまで変わらないので,
最初からペーストとして使うならそういうものでも問題ないだろう.

\hypertarget{makaroni-po-flotski}{%
\section{海軍風パスタ (露: макароны
по-флотски)}\label{makaroni-po-flotski}}

ソ連時代に流行した大衆料理.
名前の通りソ連海軍のメニューが発祥と言われている.

\begin{figure}

{\centering \includegraphics[width=1\linewidth,height=1\textheight,keepaspectratio]{img/makaroni-po-flotski/finished} 

}

\caption{海軍風パスタ}\label{fig:finished-makaroni-po-flotski}
\end{figure}

\begin{tabular}[t]{rl}
\toprule
 & 難易度\\
\midrule
材料調達 & {\fontspec{Noto Sans CJK JP} ★☆☆☆☆ }\\
調理 & {\fontspec{Noto Sans CJK JP} ★☆☆☆☆ }\\
\bottomrule
\end{tabular}

\hypertarget{ux6750ux6599-2}{%
\subsection{材料}\label{ux6750ux6599-2}}

\begin{itemize}
\tightlist
\item
  パスタ

  \begin{itemize}
  \tightlist
  \item
    ロシア語ではマカロニ=パスタ類全般なのでなんでもいい
  \item
    ペンネを使う例が多い?
  \end{itemize}
\item
  ノザキコンビーフ缶 1個 (100gくらいのやつ)

  \begin{itemize}
  \tightlist
  \item
    または牛ひき肉
  \end{itemize}
\item
  バター
\item
  玉ねぎ
\item
  塩胡椒
\item
  (オプション) 刻んだパセリや万能ネギ等
\end{itemize}

\hypertarget{ux9053ux5177-2}{%
\subsection{道具}\label{ux9053ux5177-2}}

ペペロンチーノと同じ

\hypertarget{ux4f5cux308aux65b9-1-2ux98dfux5206}{%
\subsection{作り方 (1-2食分)}\label{ux4f5cux308aux65b9-1-2ux98dfux5206}}

\begin{enumerate}
\def\labelenumi{\arabic{enumi}.}
\tightlist
\item
  パスタを茹でる.
\item
  玉ねぎをみじん切りにする
\item
  フライパンを弱火にかけ, バターを溶かす
\item
  玉ねぎを炒める

  \begin{itemize}
  \tightlist
  \item
    焦らず, 焦がさず, 時間をかけて茶色くなるまで炒めるとうまいが,
    適当に切り上げても良い
  \end{itemize}
\item
  コンビーフを混ぜる
\item
  塩胡椒で味付けする
\item
  パスタと絡める
\item
  皿に盛り付けてオプションをトッピングする
\end{enumerate}

\hypertarget{ux88dcux8db3-2}{%
\subsection{補足}\label{ux88dcux8db3-2}}

ロシアで一般的なコンビーフ (тушенка)
は脂身やブイヨンが一緒になった大きな缶詰なので,
大量のパスタに温めたコンビーフを混ぜるだけで作ることができたらしい.

\hypertarget{ux53c2ux8003ux8cc7ux6599}{%
\subsection{参考資料}\label{ux53c2ux8003ux8cc7ux6599}}

\begin{itemize}
\tightlist
\item
  『ロシアの味』日本在住ロシア人による動画 (日本語)
  \url{https://www.youtube.com/watch?v=HXz8deILEe8}
\item
  『ロシア・ビヨンド』での言及
  \url{https://jp.rbth.com/multimedia/video/2015/03/03/52185}
\end{itemize}

\hypertarget{chahan}{%
\section{醤油炒飯 (中)}\label{chahan}}

イマドキは冷凍食品を炒めるだけでもできてしまうので王剛式炒飯を参考にしたレシピを紹介する.
最もシンプルな炒飯で, ほとんど具も入ってない.

\begin{tabular}[t]{rl}
\toprule
 & 難易度\\
\midrule
材料調達 & {\fontspec{Noto Sans CJK JP} ★☆☆☆☆ }\\
調理 & {\fontspec{Noto Sans CJK JP} ★★☆☆☆ }\\
\bottomrule
\end{tabular}

\hypertarget{ux6750ux6599-3}{%
\subsection{材料}\label{ux6750ux6599-3}}

\begin{itemize}
\tightlist
\item
  米飯 1人前

  \begin{itemize}
  \tightlist
  \item
    一度炊いてから完全に冷えて固くなったもの
  \end{itemize}
\item
  卵 1個
\item
  サラダ油
\item
  ニンニク
\item
  ショウガ
\item
  万能ねぎ
\item
  うま味のあるもの

  \begin{itemize}
  \tightlist
  \item
    干瓢とか干し椎茸とか
  \end{itemize}
\item
  醤油
\item
  (オプション) 明油
\item
  (オプション) 片栗粉か小麦粉
\end{itemize}

\hypertarget{ux9053ux5177-3}{%
\subsection{道具}\label{ux9053ux5177-3}}

\begin{itemize}
\tightlist
\item
  ザル
\item
  中華鍋

  \begin{itemize}
  \tightlist
  \item
    あったほうが楽しい
  \end{itemize}
\end{itemize}

\hypertarget{ux4f5cux308aux65b9-2}{%
\subsection{作り方}\label{ux4f5cux308aux65b9-2}}

\begin{enumerate}
\def\labelenumi{\arabic{enumi}.}
\item
  米飯をザルに入れて短時間冷水ですすぎながらほぐし, 水気を良く切る

  \begin{itemize}
  \tightlist
  \item
    水を使わない方法: 片栗粉か小麦粉を軽くまぶして吸収させる\footnote{水でほぐす方法は
      \url{https://mi-journey.jp/foodie/59108/} で紹介されている.
      しかし私の場合は容器に押し固めているのが原因の1つであり,
      冷えた時点でさほど固まっていないならこのようにする必要はないかもしれない.}
  \end{itemize}
\item
  卵を卵白と卵黄に分ける

  \begin{itemize}
  \tightlist
  \item
    \href{https://www.google.com/search?sxsrf=ALeKk00q-gEeJLwrMinjdKzuK0vEwdEO9w:1613567209399\&q=\%E3\%82\%A8\%E3\%83\%83\%E3\%82\%B0\%E3\%82\%BB\%E3\%83\%91\%E3\%83\%AC\%E3\%83\%BC\%E3\%82\%BF\%E3\%83\%BC+\%E3\%81\%B2\%E3\%82\%88\%E3\%81\%93\&sa=X\&ved=2ahUKEwiL-Nii_vDuAhWRyosBHRVKDc4Q1QIoA3oECAwQBA\&biw=1920\&bih=947}{ちょっと悪趣味な器具}を使ってもいいし,
    手でやってもいい.
  \end{itemize}
\item
  卵白をよくかき混ぜる
\item
  ニンニク, ショウガ, その他を細かく切り刻む

  \begin{itemize}
  \tightlist
  \item
    米と同じくらいの大きさだと混ぜやすい
  \end{itemize}
\item
  フライパンまたは中華鍋を温め, 油をひいてなじませる

  \begin{itemize}
  \tightlist
  \item
    テフロンなら焦げ付きは我慢しよう
  \end{itemize}
\item
  弱火で卵白を注ぎ, 焦がさないって程度にゆっくり加熱して固める

  \begin{itemize}
  \tightlist
  \item
    中華鍋ならある程度固まってきたら鍋の上を回すだけで焦げ付かせずに加熱できる
  \end{itemize}
\end{enumerate}

\begin{itemize}
\tightlist
\item
  途中でいちどひっくり返す
\end{itemize}

\begin{enumerate}
\def\labelenumi{\arabic{enumi}.}
\setcounter{enumi}{6}
\item
  一旦卵白を取り出し, 油を多めに追加する
\item
  卵黄を入れ, すぐかき混ぜる
\item
  卵黄が固まってきたら, 卵白とネギ以外の固形物も入れ,
  お玉やしゃもじで叩いてよくほぐす
\item
  中華鍋なら適度に鍋を振り,
  そうでないなら固まったり焦げ付いたりしないようにかき混ぜ続ける
\item
  全体に火が通ってパラパラになってきたら,
  少し火を強くして醤油と油をふりかける

  \begin{itemize}
  \tightlist
  \item
    ネギ油など香りの良いものを使うと良い
  \end{itemize}
\item
  再び水気が飛ぶまで固まらないように, 焦げないように混ぜ続ける
\item
  最後にネギを加える
\item
  パラパラになってきたら盛り付ける
\end{enumerate}

\hypertarget{ux88dcux8db3-3}{%
\subsubsection{補足}\label{ux88dcux8db3-3}}

中華鍋で作ったほうがやりやすい.
卵黄と卵白を分けるのは面倒なように見えるが,
このほうが卵が塊にならず均等に混ざりやすい.

王剛の方法では片栗粉などで米をほぐしているが, ザルさえあれば Foodie
の記事で紹介されているように水ですすいでしまうほうが簡単である.
そしておそらく, 水を吸ったぶん加熱時間を長くする必要が生じる.

また, 平たいフライパンではくっつきやすくなり,
より多くの油が必要になるかもしれない.

\hypertarget{ux53c2ux8003ux8cc7ux6599-1}{%
\subsection{参考資料}\label{ux53c2ux8003ux8cc7ux6599-1}}

\begin{itemize}
\tightlist
\item
  美食作家王剛の醤油炒飯の作り方 (中国語, 英語字幕あり)
  \url{https://www.youtube.com/watch?v=1Q-5eIBfBDQ}
\item
  Foodie の記事 \url{https://mi-journey.jp/foodie/59108/}
\end{itemize}

\hypertarget{friedtomatoegg}{%
\section{西紅柿炒鶏蛋 (中) + 面}\label{friedtomatoegg}}

炒鶏蛋とも. 卵と西紅柿 (トマト) を炒めただけ. なおトマトは番茄ともいう
(主に台湾で使われる表現?).
王剛のレシピのうち最も簡単な「怠け者のレシピ」に基づいたレシピを紹介する.
麺に乗せて食べても良い(図\ref{fig:finished-friedtomatoegg}).

\begin{figure}

{\centering \includegraphics[width=1\linewidth,height=1\textheight,keepaspectratio]{img/friedtomatoegg/finished} 

}

\caption{西紅柿炒鶏蛋麺}\label{fig:finished-friedtomatoegg}
\end{figure}

\begin{tabular}[t]{rl}
\toprule
 & 難易度\\
\midrule
材料調達 & {\fontspec{Noto Sans CJK JP} ★☆☆☆☆ }\\
調理 & {\fontspec{Noto Sans CJK JP} ★★☆☆☆ }\\
\bottomrule
\end{tabular}

\hypertarget{ux6750ux6599-1ux4ebaux524d}{%
\subsection{材料 (1人前)}\label{ux6750ux6599-1ux4ebaux524d}}

\begin{itemize}
\tightlist
\item
  トマト 中サイズ 1個
\item
  卵 2個
\item
  長ネギ 1/4 - 1/3 程度
\item
  塩胡椒 少々
\end{itemize}

\hypertarget{ux4f5cux308aux65b9-3}{%
\subsection{作り方}\label{ux4f5cux308aux65b9-3}}

\begin{enumerate}
\def\labelenumi{\arabic{enumi}.}
\tightlist
\item
  卵を良く混ぜる
\item
  トマトを小さく角切りに, ネギをみじん切りにする
\item
  これらをまぜ, 塩胡椒を少々加える
\item
  フライパンまたは中華鍋に油をひき, よく熱する
\item
  材料を入れる
\item
  ある程度固まるまではかき混ぜすぎず, 軽く揺する程度にする

  \begin{itemize}
  \tightlist
  \item
    早くにかき混ぜると油と混ざり塊にならないが, 好みならそれでも良い
    (図\ref{fig:cooking-friedtomatoegg})
  \end{itemize}
\item
  好みの固さになるまで加熱する
\end{enumerate}

\begin{figure}

{\centering \includegraphics[width=1\linewidth,height=1\textheight,keepaspectratio]{img/friedtomatoegg/cooking} 

}

\caption{スクランブルエッグに近い状態}\label{fig:cooking-friedtomatoegg}
\end{figure}

\hypertarget{ux88dcux8db3-4}{%
\subsection{補足}\label{ux88dcux8db3-4}}

王剛のレシピでは卵4個を使っているが, 一度に大量に作るのは難しく,
また1食としては多すぎるので卵2個, 中サイズのトマト1個とした.
王剛は卵とトマトの量は 4:3 が良いと言っているので,
これを念頭において大きさを選ぶと良い. トマトが多すぎると水っぽくなる.
酢や刻んだニンニクを入れるのも良い.

麺にかけてもよい. インスタント麺でも良いが,
つけ麺や油そば用の太い麺とも合うだろう.

\hypertarget{ux53c2ux8003ux8cc7ux6599-2}{%
\subsection{参考資料}\label{ux53c2ux8003ux8cc7ux6599-2}}

\begin{itemize}
\tightlist
\item
  美食作家王剛による6動画

  \begin{itemize}
  \tightlist
  \item
    通りの作り方の動画,
    冒頭から2分までが「怠け者のレシピ」\url{https://www.youtube.com/watch?v=3gF44PXZXNc}
  \item
    中華鍋で丁寧に作る場合の解説 (日本語)
    \url{https://www.youtube.com/watch?v=-7vw-sHGtDY}
  \item
    水を足してスープにしている「西红柿鸡蛋面」
    \url{https://www.youtube.com/watch?v=PtAhBml14Uw}
  \end{itemize}
\end{itemize}

\hypertarget{shkmerli}{%
\section{\texorpdfstring{シュクメルリ\index{シュクメルリ} (草:
შქმერული\index{შქმერული|see{シュクメルリ}})}{シュクメルリ (草: შქმერული)}}\label{shkmerli}}

最近松屋の限定メニューになったので知名度が上がったが,
松屋ではなく本場ジョージアに近い (と思う) レシピを紹介する
(図\ref{fig:shkmerli-finished}).

\begin{figure}

{\centering \includegraphics[width=1\linewidth,height=1\textheight,keepaspectratio]{img/shkmerli/finished} 

}

\caption{シュクメルリ}\label{fig:shkmerli-finished}
\end{figure}

\begin{tabular}[t]{rl}
\toprule
 & 難易度\\
\midrule
材料調達 & {\fontspec{Noto Sans CJK JP} ★★☆☆☆ }\\
調理 & {\fontspec{Noto Sans CJK JP} ★★☆☆☆ }\\
\bottomrule
\end{tabular}

\hypertarget{ux6750ux6599-1ux4ebaux524d-1}{%
\subsection{材料 (1人前)}\label{ux6750ux6599-1ux4ebaux524d-1}}

\begin{itemize}
\tightlist
\item
  鶏モモ肉 1枚

  \begin{itemize}
  \tightlist
  \item
    本来は丸ごと使うが, 手抜きレシピなので骨抜きのモモ肉でよい
  \item
    丸ごと使うなら 4-5人前くらいになる
  \end{itemize}
\item
  バター 30 g
\item
  ニンニク 2-3かけ

  \begin{itemize}
  \tightlist
  \item
    もう少し多くてもよい?
  \end{itemize}
\item
  牛乳 100-200cc
\item
  (オプション) トッピング

  \begin{itemize}
  \tightlist
  \item
    コリアンダーの葉
  \item
    パプリカパウダー
  \item
    粉チーズ
  \end{itemize}
\end{itemize}

\hypertarget{ux4f5cux308aux65b9-4}{%
\subsection{作り方}\label{ux4f5cux308aux65b9-4}}

\begin{enumerate}
\def\labelenumi{\arabic{enumi}.}
\tightlist
\item
  丸ごとや骨抜きのモモ肉以外の場合, なるべく平たくなるよう切り開く.
\item
  塩を軽く振って下味をつける
\item
  フライパンを熱してバターを溶かし, 鶏肉をのせる.
\item
  上から重しをする

  \begin{itemize}
  \tightlist
  \item
    アルミホイルの上から水を入れた炊飯釜や鍋などを載せるとよい
    (図\ref{fig:shkmerli-press})
  \end{itemize}
\item
  中火で10-15分程度焼く
\item
  裏返して同様に重しをして焼く. 焼き目がついていたら成功
  (図\ref{fig:shkmerli-fried})
\item
  一旦鶏肉を取り出し, 熱に気をつけて食べやすい大きさに切る
  (図\ref{fig:shkmerli-chopped})
\item
  弱火にして, フライパンに刻んだニンニクを入れ,
  バターと溶けた鶏の脂に絡める
\item
  ニンニクの香りが出てきたら再度鶏肉を入れ, 牛乳を注ぐ
\item
  ふつふつと煮立ってくるまで加熱する
\item
  皿にソースと一緒に盛り付けてオプションのトッピングをする.
\end{enumerate}

\begin{figure}

{\centering \includegraphics[width=1\linewidth,height=1\textheight,keepaspectratio]{img/shkmerli/press} 

}

\caption{重しをする}\label{fig:shkmerli-press}
\end{figure}

\begin{figure}

{\centering \includegraphics[width=1\linewidth,height=1\textheight,keepaspectratio]{img/shkmerli/fried} 

}

\caption{片面を焼いた直後 (胸は切除している)}\label{fig:shkmerli-fried}
\end{figure}

\begin{figure}

{\centering \includegraphics[width=1\linewidth,height=1\textheight,keepaspectratio]{img/shkmerli/chopped} 

}

\caption{適当に切り分ける}\label{fig:shkmerli-chopped}
\end{figure}

\hypertarget{ux88dcux8db3-5}{%
\subsection{補足}\label{ux88dcux8db3-5}}

ハチャプリ (作り方は参考資料参照) かパン, そしてワインないしチャチャ
(ჭაჭა) と一緒に食べると良い.
もともとはタパカというローストチキンのレシピをアレンジしたものらしい.

\hypertarget{ux53c2ux8003ux8cc7ux6599-3}{%
\subsection{参考資料}\label{ux53c2ux8003ux8cc7ux6599-3}}

\begin{itemize}
\tightlist
\item
  シュクメリ村での伝統的な作り方, 牛乳の代わりに水を使う (ジョージア語)
  \url{https://www.youtube.com/watch?v=qBpRB3QAjoQ}
\item
  『シュクメルリ異聞』上記の動画に言及した日本語の解説記事
  \url{http://georgia1001.com/2020/02/12/anotherstoryofshkmeruli/}
\item
  『シュクメルリには「茶シュク」と「白シュク」の2つがある。本当のシュクメルリは茶色い料理です。』
  \url{http://jp.ndish.com/diary/20200901_3847/}
\item
  ジョージア大使によるハチャプリの作り方実演動画
  \url{https://www.youtube.com/watch?v=v1fL2_qm4xE}
\end{itemize}

\hypertarget{ux30d1ux30a8ux30eaux30e4-ux897f-paella}{%
\section{\texorpdfstring{パエリヤ (西:
Paella)\index{パエリヤ}\index{paella|see{パエリヤ}}}{パエリヤ (西: Paella)}}\label{ux30d1ux30a8ux30eaux30e4-ux897f-paella}}

パエリヤというとムール貝やエビをふんだんに使ったものを連想しがちだが,
ここでは鶏肉がメインの比較的簡単な paella valenciana
風のレシピを紹介する.

\begin{figure}

{\centering \includegraphics[width=1\linewidth,height=1\textheight,keepaspectratio]{img/paella/finished} 

}

\caption{鶏肉のパエリア}\label{fig:paella-finished}
\end{figure}

\begin{tabular}[t]{rl}
\toprule
 & 難易度\\
\midrule
材料調達 & {\fontspec{Noto Sans CJK JP} ★☆☆☆☆ }\\
調理 & {\fontspec{Noto Sans CJK JP} ★★☆☆☆ }\\
\bottomrule
\end{tabular}

\hypertarget{ux6750ux6599-3-4ux98dfux5206}{%
\subsection{材料 (3-4食分)}\label{ux6750ux6599-3-4ux98dfux5206}}

\begin{itemize}
\item
  米 1合

  \begin{itemize}
  \tightlist
  \item
    できればジャバニカ米がいいが, なければ日本の米でいい.
  \end{itemize}
\item
  鶏肉 またはウサギ肉
\item
  オリーブオイル
\item
  インゲン豆

  \begin{itemize}
  \tightlist
  \item
    さやごと使う.
  \end{itemize}
\item
  日本とスペインで一般的なインゲン豆は品種が違うような気がする
\item
  ニンニク 1個
\item
  玉ねぎ 1玉
\item
  トマトピューレ 50g
\item
  パプリカパウダー 大さじ1杯
\item
  ローズマリー 数本
\item
  サフラン ひとつまみ

  -- 香りが肝だが高価なのでなければないで可
\item
  (オプション) コンソメスープ だいたい米の2倍

  -- コンソメスープの素などなんでも
\end{itemize}

\hypertarget{ux9053ux5177-4}{%
\subsection{道具}\label{ux9053ux5177-4}}

\begin{itemize}
\item
  アルミホイル
\item
  専用フライパンではなく普通のフライパンでもなんとかなる.
\end{itemize}

\hypertarget{ux4f5cux308aux65b9-5}{%
\subsection{作り方}\label{ux4f5cux308aux65b9-5}}

\begin{enumerate}
\def\labelenumi{\arabic{enumi}.}
\item
  サフランをアルミホイルに包み, 熱したフライパンに10秒だけ載せて煎る.
\item
  肉を食べやすい大きさに切る
\item
  玉ねぎとニンニクをみじん切りにする
\item
  フライパンに火をかけオリーブオイルをひく
\item
  鶏肉を焼き目が付くまでよく炒める
\item
  インゲン豆も炒める
\item
  玉ねぎとニンニクも加えて炒める
\item
  よく火が通ったら, パプリカパウダーとトマトピューレを加える
\item
  コンソメスープか, なければお湯を注ぐ
\item
  米を入れる. たぶん研がなくても良い. 米が山になってたらならす
\item
  ローズマリーを添える
\item
  一切かき混ぜずに加熱を続ける.
  加熱時間は米の量や品種次第なので状況判断する
\item
  水かさが減ってきて米が露出してきたら, 蓋をかぶせて弱火にする

  \begin{itemize}
  \tightlist
  \item
    残った水分で蒸すことになる
  \end{itemize}
\end{enumerate}

\hypertarget{ux88dcux8db3-6}{%
\subsubsection{補足}\label{ux88dcux8db3-6}}

欲張って一度に大量に作ろうとすると米の上層に火が通ってないのに底が焦げ付くということが起こりがちである.

参考動画ではアルティチョークを使用しているが,
なかなか手に入らないので私は使ったことがない.

\hypertarget{ux53c2ux8003ux8cc7ux6599-4}{%
\subsection{参考資料}\label{ux53c2ux8003ux8cc7ux6599-4}}

\begin{itemize}
\tightlist
\item
  \url{https://www.youtube.com/watch?v=L_dDUw_QuDU}
\end{itemize}

\hypertarget{ux30b9ux30daux30a4ux30f3ux98a8ux30aaux30e0ux30ecux30c4-ux897f-tortilla-de-patatas}{%
\section{\texorpdfstring{スペイン風オムレツ (西: Tortilla de
Patatas)\index{スペイン風オムレツ}\index{tortilla de patatas|see{スペイン風オムレツ}}}{スペイン風オムレツ (西: Tortilla de Patatas)}}\label{ux30b9ux30daux30a4ux30f3ux98a8ux30aaux30e0ux30ecux30c4-ux897f-tortilla-de-patatas}}

日本のオムレツ (フランス風) と違いしっかり焼き目をつけ,
固くする(図\ref{fig:finished-tornilla}).

\begin{figure}

{\centering \includegraphics[width=1\linewidth,height=1\textheight,keepaspectratio]{img/tortilla/finished} 

}

\caption{スペイン風オムレツ}\label{fig:finished-tornilla}
\end{figure}

\begin{tabular}[t]{rl}
\toprule
 & 難易度\\
\midrule
材料調達 & {\fontspec{Noto Sans CJK JP} ★☆☆☆☆ }\\
調理 & {\fontspec{Noto Sans CJK JP} ★★☆☆☆ }\\
\bottomrule
\end{tabular}

\hypertarget{ux6750ux6599-3-4ux98dfux5206-1}{%
\subsection{材料 (3-4食分)}\label{ux6750ux6599-3-4ux98dfux5206-1}}

\begin{itemize}
\tightlist
\item
  卵 4-5個
\item
  ジャガイモ 1-2個
\item
  玉ねぎ 1個
\item
  ベーコンまたはハム 50-100g?
\item
  塩胡椒
\item
  オリーブオイル
\end{itemize}

\hypertarget{ux9053ux5177-5}{%
\subsection{道具}\label{ux9053ux5177-5}}

\begin{itemize}
\tightlist
\item
  大きめのフライパン

  \begin{itemize}
  \tightlist
  \item
    直径 28cmとかそれくらいの
  \item
    蓋も必要
  \end{itemize}
\item
  大きめの皿
\end{itemize}

\hypertarget{ux4f5cux308aux65b9-6}{%
\subsection{作り方}\label{ux4f5cux308aux65b9-6}}

\begin{enumerate}
\def\labelenumi{\arabic{enumi}.}
\tightlist
\item
  ジャガイモ, 玉ねぎ, ベーコンを小さく切る

  \begin{itemize}
  \tightlist
  \item
    ジャガイモはスライス, それ以外はみじん切りがよい
  \end{itemize}
\item
  フライパンにオリーブオイルを広げ, 上記を炒める

  \begin{itemize}
  \tightlist
  \item
    ジャガイモが焦げ付かないように油は多め, 火は弱く
  \end{itemize}
\item
  塩胡椒で味付けする

  \begin{itemize}
  \tightlist
  \item
    ケチャップ等をかけない前提なのでしっかり味付けする
  \end{itemize}
\item
  火を止めて少し冷ます
\item
  ボウルに卵を全て割り, 溶き卵にする
\item
  卵に炒めた具材を混ぜる
\item
  蓋をして10分寝かせる
\item
  フライパンに注いで再び数分加熱する

  \begin{itemize}
  \tightlist
  \item
    混ぜる必要はないが軽く揺すってくっつかないようにする
  \end{itemize}
\item
  底が十分焼き固まったようなら, 皿をかぶせ, 裏返して皿に載せる

  \begin{itemize}
  \tightlist
  \item
    油が垂れてやけどしないようにすばやくやる
  \end{itemize}
\item
  フライパンに戻して反対側も焼き固める
\item
  ピザのように切り分けて食べる
\end{enumerate}

\hypertarget{ux88dcux8db3-7}{%
\subsection{補足}\label{ux88dcux8db3-7}}

ベーコンはさほど重要ではない. 必須なのはジャガイモ.

\hypertarget{ux53c2ux8003ux8cc7ux6599-5}{%
\subsection{参考資料}\label{ux53c2ux8003ux8cc7ux6599-5}}

\begin{itemize}
\tightlist
\item
  \url{https://www.youtube.com/watch?v=JceGMNG7rpU}
\end{itemize}

\hypertarget{cesnecka}{%
\section{\texorpdfstring{チェスネチカ - (捷:
česnečka)\index{チェスネチカ}\index{česnečka|see{チェスネチカ}}}{チェスネチカ - (捷: česnečka)}}\label{cesnecka}}

名前はそのまま「ニンニク (のスープ)」と言う意味
(図\ref{fig:cesnecka-finished}).

\begin{figure}

{\centering \includegraphics[width=1\linewidth,height=1\textheight,keepaspectratio]{img/cesnecka/finished} 

}

\caption{チェスネチカ}\label{fig:cesnecka-finished}
\end{figure}

\begin{tabular}[t]{rl}
\toprule
 & 難易度\\
\midrule
材料調達 & {\fontspec{Noto Sans CJK JP} ★☆☆☆☆ }\\
調理 & {\fontspec{Noto Sans CJK JP} ★☆☆☆☆ }\\
\bottomrule
\end{tabular}

\hypertarget{ux6750ux6599-4}{%
\subsection{材料}\label{ux6750ux6599-4}}

\begin{itemize}
\tightlist
\item
  バター 20 g
\item
  ニンニク 5-6 個
\item
  玉ねぎ 1個
\item
  ジャガイモ 1個
\item
  水またはブイヨン 適量
\item
  塩胡椒 適量
\item
  キャラウェイシード 少々
\item
  マジョラム 少々
\item
  (オプション) クルトン, あるいは古くなった食パン
\item
  (オプション) 溶き卵
\end{itemize}

\hypertarget{ux4f5cux308aux65b9-7}{%
\subsection{作り方}\label{ux4f5cux308aux65b9-7}}

\begin{enumerate}
\def\labelenumi{\arabic{enumi}.}
\tightlist
\item
  鍋にバターをひく
\item
  玉ねぎをみじん切りにし, 弱火で炒める
\item
  鍋で湯を沸かす

  \begin{itemize}
  \tightlist
  \item
    ブイヨンの素などを溶かすと良い
  \end{itemize}
\item
  ジャガイモの皮を向き, 小さめに切る
\item
  鍋にジャガイモを入れて茹でる
\item
  みじん切りにしたニンニクを入れる
\item
  キャラウェイシードやマジョラムを入れる
\item
  塩胡椒で味を調える
\item
  盛り付け, お好みでクルトンや溶き卵を入れる
\end{enumerate}

\hypertarget{ux88dcux8db3-8}{%
\subsubsection{補足}\label{ux88dcux8db3-8}}

中世の頃から似たような料理が同地に存在するらしい.
現代のチェコ人は二日酔いや風邪気味で調子が悪かったら朝食をこれにするらしい.
ニンニクは最初に入れたり, 2回に分けて入れたりしても良い.
もうすこし食べごたえのあるものにしたければ, ベーコンを入れても良い

\hypertarget{ux53c2ux8003ux8cc7ux6599-6}{%
\subsubsection{参考資料}\label{ux53c2ux8003ux8cc7ux6599-6}}

\begin{itemize}
\tightlist
\item
  czechcookbook の動画 (英語音声)
  \url{https://www.youtube.com/watch?v=vYeoxYUMzME}
\item
  Random Innkeeper の動画 (英語音声, 日本語字幕)
  \url{https://www.youtube.com/watch?v=F_59dZw5G44}
\item
  \citet{faktor2007Traditional} p.~18
\end{itemize}

\hypertarget{ux9d8fux91ceux83dcux5473ux564c-ux65e5}{%
\section{\texorpdfstring{鶏・野菜・味噌
(日)\index{鶏・野菜・味噌}}{鶏・野菜・味噌 (日)}}\label{ux9d8fux91ceux83dcux5473ux564c-ux65e5}}

料理名なのか怪しいが, こういう名前で通っている.
最低限鶏肉と白菜と味噌があれば作れる. 冬に作ると良い.

TODO: 画像

\begin{tabular}[t]{rl}
\toprule
 & 難易度\\
\midrule
材料調達 & {\fontspec{Noto Sans CJK JP} ★☆☆☆☆ }\\
調理 & {\fontspec{Noto Sans CJK JP} ★★☆☆☆ }\\
\bottomrule
\end{tabular}

材料

\begin{itemize}
\tightlist
\item
  味噌 -- 「まつや」の専用味噌は高いので使わない --
  「日本海味噌\footnote{\url{https://www.youtube.com/watch?v=JG_eS7HcuD8}}」あたりが味と価格のバランスが良い
  -- 煮干しや昆布の出汁も合わせるとより良い
\item
  白菜
\item
  鶏肉 -- できればモモ肉
\item
  その他鍋料理に使う食材 -- 椎茸, えのき, 人参, ネギなど
\item
  (オプション) 七味唐辛子
\end{itemize}

\hypertarget{ux4f5cux308aux65b9-8}{%
\subsection{作り方}\label{ux4f5cux308aux65b9-8}}

\begin{enumerate}
\def\labelenumi{\arabic{enumi}.}
\tightlist
\item
  昆布や煮干しの出汁か, 熱湯に味噌を溶く.
\item
  適当に切った食材を煮る

  \begin{itemize}
  \tightlist
  \item
    人参, 肉, 白菜その他の野菜, の順が良いと思う.
  \end{itemize}
\item
  野菜に火が通り柔らかくなってきたら食べる
\item
  七味唐辛子で味付けして食べる
\item
  残った出汁でうどんを煮たり雑炊にしても良い.
\end{enumerate}

\hypertarget{ux88dcux8db3-9}{%
\subsubsection{補足}\label{ux88dcux8db3-9}}

たぶん石川県民しか知らない. 本来は特製の味噌を使う.
\url{https://www.toriyasaimiso.jp/recipe}

\hypertarget{spicy}{%
\chapter{激辛料理編}\label{spicy}}

主に四川料理とインド料理.
冗談抜きで辛いレシピばかりなので辛いのが苦手だと自負する人にはお薦めしない.
四川料理はだいたい「美食作家王剛」の動画を模倣したもの.

\hypertarget{ux8fa3ux5b50ux9d8f-ux4e2d-ux56dbux5ddd}{%
\section{\texorpdfstring{辣子鶏 (中,
四川)\index{辣子鶏}\index{辣子鸡|see{辣子鶏}}}{辣子鶏 (中, 四川)}}\label{ux8fa3ux5b50ux9d8f-ux4e2d-ux56dbux5ddd}}

唐辛子まみれの鶏の唐揚 (図\ref{fig:laziji-finished}).
『中華一番!』にもあるように, 唐辛子は食べなくてもいい.

\begin{figure}

{\centering \includegraphics[width=1\linewidth,height=1\textheight,keepaspectratio]{img/laziji/finished} 

}

\caption{辣子鶏}\label{fig:laziji-finished}
\end{figure}

\begin{tabular}[t]{rl}
\toprule
 & 難易度\\
\midrule
材料調達 & {\fontspec{Noto Sans CJK JP} ★★☆☆☆ }\\
調理 & {\fontspec{Noto Sans CJK JP} ★★★☆☆ }\\
\bottomrule
\end{tabular}

\hypertarget{ux6750ux6599-2ux98dfux5206}{%
\subsection{材料 (2食分)}\label{ux6750ux6599-2ux98dfux5206}}

唐揚げの下処理に必要な材料

\begin{itemize}
\tightlist
\item
  鶏肉 300 g

  \begin{itemize}
  \tightlist
  \item
    若い雄鶏が良いらしい
  \end{itemize}
\item
  塩 小さじ 1 杯
\item
  醤油 小さじ 1 杯

  \begin{itemize}
  \tightlist
  \item
    元のレシピでは生抽
  \end{itemize}
\item
  胡椒 少々
\item
  片栗粉 小さじ2杯
\item
  卵黄 1個分
\item
  料理酒 小さじ 1/2杯
\end{itemize}

唐揚げと炒める材料

\begin{itemize}
\tightlist
\item
  唐辛子 90 g

  \begin{itemize}
  \tightlist
  \item
    元のレシピでは 干七星, 小弾頭,
    燈籠の三種を使うとあるが鷹の爪を使う前提にする
  \end{itemize}
\item
  青花椒 10 g

  \begin{itemize}
  \tightlist
  \item
    王剛のレシピはかなり多いと感じる. 多少すくなくても良い
  \end{itemize}
\item
  ニンニク
\item
  生姜
\item
  ネギ 20 g

  \begin{itemize}
  \tightlist
  \item
    元のレシピでは香葱だが, 日本では入手が難しいのでネギ類なら何でも可
  \end{itemize}
\item
  ピーナッツ 20 g
\item
  白ごま 適量
\item
  胡麻
\item
  揚げ物用のサラダ油 大量に
\item
  ごま油 20 ml
\item
  花椒油 20 ml

  \begin{itemize}
  \tightlist
  \item
    これも苦手ならもっと少なくても良い
  \end{itemize}
\item
  香醋 20 ml

  \begin{itemize}
  \tightlist
  \item
    独特の香りと色が日本で一般的な穀物酢にはないので, 代用できるか怪しい
  \end{itemize}
\item
  砂糖 少々
\item
  うま味調味料 少々

  \begin{itemize}
  \tightlist
  \item
    シャンタンとか
  \end{itemize}
\end{itemize}

\hypertarget{ux9053ux5177-6}{%
\subsection{道具}\label{ux9053ux5177-6}}

\begin{itemize}
\tightlist
\item
  中華鍋

  \begin{itemize}
  \tightlist
  \item
    フライパンでもさほど難しくないと思われる
  \end{itemize}
\item
  中華お玉
\item
  ジャーレン
\item
  ボウルといくつかの小皿
\end{itemize}

\hypertarget{ux4f5cux308aux65b9-9}{%
\subsection{作り方}\label{ux4f5cux308aux65b9-9}}

\begin{enumerate}
\def\labelenumi{\arabic{enumi}.}
\item
  鶏肉を通常の唐揚げよりかなり小さく, 一口サイズにぶつ切りにする

  \begin{itemize}
  \tightlist
  \item
    だいたい 1-2cm 四方くらい
  \item
    小さく切らないと高温かつ短時間で揚げることができない
  \end{itemize}
\item
  \begin{enumerate}
  \def\labelenumii{(\arabic{enumii})}
  \tightlist
  \item
    に塩, 胡椒, 料理酒 醤油, 卵黄を混ぜてよく揉む
  \end{enumerate}
\item
  片栗粉を加え良く混ぜる
\item
  10 ml の植物油を加えよく混ぜる
\item
  ニンニク, 生姜, ネギをみじん切りにする
\item
  唐辛子を適当な大きさに切る (図\ref{fig:laziji-ingredients})

  \begin{itemize}
  \tightlist
  \item
    辛いのが苦手ならこのとき種を全て捨てる, 種の有無で辛さはかなり変わる
  \item
    ジャーレンをふるいの代わりにすると簡単に種と実を選り分けられる
  \end{itemize}
\item
  揚げ物用の油を 210度C まで加熱する (図\ref{fig:laziji-fried})

  \begin{itemize}
  \tightlist
  \item
    家庭用のコンロでは高温の維持が難しいので気休め
  \end{itemize}
\item
  鶏肉を1分程度揚げる

  \begin{itemize}
  \tightlist
  \item
    家庭用のコンロではすぐ温度が下がるので実際は3-5分程度
  \item
    肉どうしがくっつかないようにほぐす
  \end{itemize}
\item
  一旦鶏肉を取り出し, 揚げカスを取り除く
\item
  240度Cまで上げて, 再度1分程度揚げる

  \begin{itemize}
  \tightlist
  \item
    この温度も実際には維持が難しい
  \item
    表面をカリカリにするという意図
  \end{itemize}
\item
  鶏肉を取り出し, 油もよける
\item
  油を 20ml 引き, 加熱する
\item
  ニンニク, 生姜, ネギ, 唐辛子, 花椒を入れて炒める

  \begin{itemize}
  \tightlist
  \item
    唐辛子はかさばるので溢れたり, 焦げ付かせたりしないように注意
    (図\ref{fig:laziji-fry-wtih-chille})
  \end{itemize}
\item
  香りが立ってきたら鶏肉を入れ, 2分ほど炒める

  \begin{itemize}
  \tightlist
  \item
    じっくりと炒め, 辛味を鶏肉に移す
  \end{itemize}
\item
  うま味調味料と砂糖を少々, ピーナッツと胡麻,
  そして残りの香り付け用の油と酢をふりかける
\item
  これもよくかき混ぜる
\end{enumerate}

\begin{figure}

{\centering \includegraphics[width=1\linewidth,height=1\textheight,keepaspectratio]{img/laziji/ingredients} 

}

\caption{揚げる直前の材料}\label{fig:laziji-ingredients}
\end{figure}

\begin{figure}

{\centering \includegraphics[width=1\linewidth,height=1\textheight,keepaspectratio]{img/laziji/fried} 

}

\caption{揚げる量に対して十分な大きさの中華鍋を用意すること}\label{fig:laziji-fried}
\end{figure}

\begin{figure}

{\centering \includegraphics[width=1\linewidth,height=1\textheight,keepaspectratio]{img/laziji/fry-with-chille} 

}

\caption{焦げ付かないよう絶えず鍋を振る}\label{fig:laziji-fry-wtih-chille}
\end{figure}

\hypertarget{ux88dcux8db3-10}{%
\subsection{補足}\label{ux88dcux8db3-10}}

鶏肉を小さく切る,
ピーナッツを入れるといったやり方は宮保鶏丁のレシピを部分的に取り入れた王剛のアレンジだと思われる.
ピーナッツや胡麻, 揚げ物用以外の油は主に香りを付けるためにある.
エッセンシャルなのは下味を付けた一口サイズの唐揚げ, ネギ, 唐辛子,
そして花椒である.
なお食べ終わった後に大量に残る唐辛子は残念ながら捨てなければならない.

\hypertarget{ux53c2ux8003ux8cc7ux6599-7}{%
\subsection{参考資料}\label{ux53c2ux8003ux8cc7ux6599-7}}

\begin{itemize}
\tightlist
\item
  美食作家王剛の動画 \url{https://www.youtube.com/watch?v=NqHI1CJU-Rg}
\item
  同, 火鍋の素を入れるなどアレンジしたバージョン
  \url{https://www.youtube.com/watch?v=ZjnOTA65DPo}
\item
  『メシ通』の記事, フライパンを使用
  \url{https://www.hotpepper.jp/mesitsu/entry/noriki-washiya/19-00032}
\end{itemize}

\hypertarget{rosu}{%
\section{\texorpdfstring{魚香肉絲\index{魚香肉絲}/青椒肉絲\index{青椒肉絲}
(中, 四川)}{魚香肉絲/青椒肉絲 (中, 四川)}}\label{rosu}}

レシピが似ているので2つまとめて紹介する. 四川料理の中ではさほど辛くない.
魚香肉絲の「魚香」は「魚料理風の味付け」という程度の意味なので魚は使わない.

\begin{tabular}[t]{rl}
\toprule
 & 難易度\\
\midrule
材料調達 & {\fontspec{Noto Sans CJK JP} ★★★☆☆ }\\
調理 & {\fontspec{Noto Sans CJK JP} ★★☆☆☆ }\\
\bottomrule
\end{tabular}

\hypertarget{ux6750ux6599-2ux98df}{%
\subsection{材料 (2食)}\label{ux6750ux6599-2ux98df}}

肉と下処理に必要な材料

\begin{itemize}
\tightlist
\item
  豚肉 200-300g
\item
  塩・胡椒 少々
\item
  酒 小さじ1
\item
  醤油 小さじ 1/2
\item
  卵白 1個分
\item
  中華風スープ 少量

  \begin{itemize}
  \tightlist
  \item
    湯に溶いたシャンタンでも何でも良い
  \end{itemize}
\item
  片栗粉 小さじ1
\end{itemize}

魚香に必要な材料

\begin{itemize}
\tightlist
\item
  キクラゲ 40 g
\item
  タケノコ 50 g

  \begin{itemize}
  \tightlist
  \item
    本場では蜀漢笋を使うがなくてもよい
  \end{itemize}
\item
  白ネギ 30 g
\item
  泡辣椒 30 g
\item
  ニンニク 20g
\item
  生姜 20g
\item
  合わせ調味料

  \begin{itemize}
  \tightlist
  \item
    塩 少々
  \item
    うま味調味料 少々
  \item
    砂糖 大さじ1
  \item
    料理酒 小さじ2
  \item
    醤油 小さじ1
  \item
    醋 小さじ2

    \begin{itemize}
    \tightlist
    \item
      なければ酢で良い
    \end{itemize}
  \end{itemize}
\end{itemize}

青椒に必要な材料

\begin{itemize}
\tightlist
\item
  ピーマンやししとう, 青唐辛子など 適量

  \begin{itemize}
  \tightlist
  \item
    日本の青唐辛子は辛いので少しだけにしておく
  \end{itemize}
\item
  醤油 小さじ1
\item
  塩 少量
\item
  うま味調味料 少量
\item
  油 適量
\end{itemize}

\hypertarget{ux9053ux5177-7}{%
\subsection{道具}\label{ux9053ux5177-7}}

\begin{itemize}
\tightlist
\item
  例によってできれば中華鍋があるとよい
\end{itemize}

\hypertarget{ux4f5cux308aux65b9-10}{%
\subsection{作り方}\label{ux4f5cux308aux65b9-10}}

\begin{enumerate}
\def\labelenumi{\arabic{enumi}.}
\tightlist
\item
  豚肉の下処理
\item
  肉を細く切る
\item
  片栗粉を除く上記の調味料を加え, 2分間かき混ぜ続ける
\item
  片栗粉をさらに加え,よく混ぜる
\item
  野菜の下処理
\item
  キクラゲを細く切る (巻いてから輪切りにするとやりやすい)
\item
  タケノコも細く切る
\item
  ネギも細く切る
\item
  青椒の場合はピーマン等を細く切る
\item
  ニンニク・生姜・泡辣椒をみじん切りにする
\item
  合わせ調味料の作成
\item
  上記の材料を予め混ぜておく
\item
  調理

  \begin{enumerate}
  \def\labelenumii{\arabic{enumii}.}
  \tightlist
  \item
    熱湯に塩を少々入れ, タケノコとキクラゲを数分だけ下茹でする
  \item
    鍋に油を多く入れ, 120度Cまで加熱してから火を止め, 余熱で肉を加熱する
    (油通し)

    \begin{itemize}
    \tightlist
    \item
      表面が白くなる程度まで加熱すれば十分
    \end{itemize}
  \item
    鍋をきれいにし, ニンニク・生姜・泡辣椒を加えて弱火で炒める
  \item
    香りが立ってきたら肉を入れて強火で炒める
  \item
    キクラゲとタケノコを加えて馴染むまで炒める
  \item
    合わせ調味料も加えて炒める
  \item
    最後にネギと少量の油を加えて炒める

    \begin{itemize}
    \tightlist
    \item
      ネギを焦がさない程度でやめる
    \end{itemize}
  \end{enumerate}
\end{enumerate}

青椒肉絲の場合は, 肉の油通しの後, 代わりに野菜を先に炒め,
火が通ってきたら肉と調味料を投入して炒める.

\hypertarget{ux88dcux8db3-11}{%
\subsection{補足}\label{ux88dcux8db3-11}}

ちょうどいい火加減はよくわからないので,
フライパンを置いて料理する場合は注意.

王剛は切った肉を水でゆすぐと肉が柔らかくなるとしているが,
私は違いを感じられなかったので省略している.

\hypertarget{ux53c2ux8003ux8cc7ux6599-8}{%
\subsection{参考資料}\label{ux53c2ux8003ux8cc7ux6599-8}}

\begin{itemize}
\tightlist
\item
  美食作家王剛の料理動画 (中国語)
  \url{https://www.youtube.com/watch?v=TpR9-9CNxAY}
\item
  同, 古いバージョン \url{https://www.youtube.com/watch?v=68v5mGdE978}
\item
  同, 青椒肉絲の料理動画 (中国語)
  \url{https://www.youtube.com/watch?v=wn0_7SMpw6o}
\end{itemize}

\hypertarget{ux6c34ux716eux8089ux7247-ux4e2d-ux56dbux5ddd}{%
\section{\texorpdfstring{水煮肉片 (中,
四川)\index{水煮肉片}}{水煮肉片 (中, 四川)}}\label{ux6c34ux716eux8089ux7247-ux4e2d-ux56dbux5ddd}}

水煮という名に反して赤黒い液体で煮込まれた四川料理
(図\ref{fig:finished-mizuni}).

\begin{figure}

{\centering \includegraphics[width=1\linewidth,height=1\textheight,keepaspectratio]{img/mizuni/finished} 

}

\caption{水煮肉片}\label{fig:finished-mizuni}
\end{figure}

\begin{tabular}[t]{rl}
\toprule
 & 難易度\\
\midrule
材料調達 & {\fontspec{Noto Sans CJK JP} ★★★☆☆ }\\
調理 & {\fontspec{Noto Sans CJK JP} ★★☆☆☆ }\\
\bottomrule
\end{tabular}

\hypertarget{ux6750ux6599-2ux98dfux5206-1}{%
\subsection{材料 (2食分)}\label{ux6750ux6599-2ux98dfux5206-1}}

下処理に必要な材料

\begin{itemize}
\tightlist
\item
  豚ロース 300g

  \begin{itemize}
  \tightlist
  \item
    バラやヒレでも可
  \end{itemize}
\item
  塩 小さじ 1/2
\item
  胡椒 少々
\item
  片栗粉 小さじ1
\item
  醤油 小さじ1
\item
  料理酒 小さじ1
\item
  卵白 1個分
\item
  片栗粉 小さじ2
\item
  植物油 大さじ1
\end{itemize}

炒める材料

\begin{itemize}
\tightlist
\item
  もやし 100g
\item
  レタス 数枚
\item
  ニンニク 2-3個
\item
  トッピング用ニンニク 2-3個
\item
  生姜 1かけ
\item
  泡辣椒 20 g
\item
  刀口辣椒 適量
\item
  乾燥唐辛子 10 g
\item
  青花椒 5 g
\item
  豆板醤 10 g
\item
  葉ニンニク

  \begin{itemize}
  \tightlist
  \item
    なければニンニクの芽やネギ
  \end{itemize}
\end{itemize}

\hypertarget{ux4f5cux308aux65b9-11}{%
\subsection{作り方}\label{ux4f5cux308aux65b9-11}}

\begin{enumerate}
\def\labelenumi{\arabic{enumi}.}
\item
  豚肉を薄切りにする
\item
  塩小さじ1杯, 胡椒少々, 醤油小さじ1杯, 調理酒小さじ1杯を入れる
\item
  2分間かき混ぜよくなじませる
\item
  卵白に片栗粉を小さじ2杯加えてよく混ぜる
\item
  \begin{enumerate}
  \def\labelenumii{(\arabic{enumii})}
  \setcounter{enumii}{3}
  \tightlist
  \item
    を肉とよく混ぜる
  \end{enumerate}
\item
  レタス, 葉ニンニクを適当な大きさに切る
\item
  ニンニク, 生姜, 泡辣椒をみじん切りにする
\item
  中華鍋に油をひき, 170度まで熱する
\item
  乾燥唐辛子と青花椒を入れ爆する

  \begin{itemize}
  \tightlist
  \item
    刀口辣椒と同様, 短時間だけ加熱し焦がさないようにする
  \end{itemize}
\item
  野菜を投入する
\item
  塩を少々入れて味付けする
\item
  そのまま20秒程度爆する

  \begin{itemize}
  \tightlist
  \item
    軽く火が通ってしんなりする程度で良い
  \end{itemize}
\item
  皿によそう
\item
  ニンニク, 生姜, 泡辣椒を鍋に入れる
\item
  軽く炒めたら豆板醤もいれ, よく溶かす
\item
  水を入れ, 強火にする
\item
  スープが煮え立ってきたら塩, 砂糖, うま味調味料を少々入れる
\item
  水溶き片栗粉も入れてとろみをつける
\item
  中火にし, とろみが取れないよう静かに肉を入れて煮る
\item
  肉に火が通ったら皿に肉を盛り付け, スープも適量注ぐ
\item
  最後に刀口辣椒とみじん切りにしたニンニクとネギをふりかける

  \begin{itemize}
  \tightlist
  \item
    ニンニクは刀口辣椒の後にかける
  \end{itemize}
\item
  210度に熱した油を上からかける
\end{enumerate}

\hypertarget{ux88dcux8db3-12}{%
\subsection{補足}\label{ux88dcux8db3-12}}

正確にはレタスではなく油麦菜 (チシャ) を使う. また,
王剛のやり方では芹菜も使うが日本では入手しづらいので省略した.

\hypertarget{ux53c2ux8003ux8cc7ux6599-9}{%
\subsection{参考資料}\label{ux53c2ux8003ux8cc7ux6599-9}}

\begin{itemize}
\tightlist
\item
  美食作家王剛の料理動画
  \url{https://www.youtube.com/watch?v=hyiGCNwCMxU}
\end{itemize}

\hypertarget{ux9ebbux5a46ux8c46ux8150-ux4e2d-ux56dbux5ddd}{%
\section{\texorpdfstring{麻婆豆腐 (中,
四川)\index{麻婆豆腐}}{麻婆豆腐 (中, 四川)}}\label{ux9ebbux5a46ux8c46ux8150-ux4e2d-ux56dbux5ddd}}

国内メーカーのインスタント食品と比べると味の濃さも油っこさも段違いである.

\begin{figure}

{\centering \includegraphics[width=1\linewidth,height=1\textheight,keepaspectratio]{img/mabo/finished} 

}

\caption{麻婆豆腐}\label{fig:finished-mabo}
\end{figure}

\begin{tabular}[t]{rl}
\toprule
 & 難易度\\
\midrule
材料調達 & {\fontspec{Noto Sans CJK JP} ★★★☆☆ }\\
調理 & {\fontspec{Noto Sans CJK JP} ★★☆☆☆ }\\
\bottomrule
\end{tabular}

\hypertarget{ux6750ux6599-2ux98dfux5206-2}{%
\subsection{材料 (2食分)}\label{ux6750ux6599-2ux98dfux5206-2}}

\begin{itemize}
\tightlist
\item
  豆腐 1丁 (約 400 g)

  \begin{itemize}
  \tightlist
  \item
    味がとても濃いのでバランスを取るため木綿が良い
  \end{itemize}
\item
  牛挽き肉 50-100g

  \begin{itemize}
  \tightlist
  \item
    脂身の多い部位が望ましい
  \item
    安い部位を買ってきて包丁で粗挽き肉にするのも一興
  \end{itemize}
\item
  ニンニク 4-5 欠片
\item
  生姜 20 g
\item
  豆板醤 大さじ1杯
\item
  泡辣椒 5-6本
\item
  豆豉 大さじ1杯
\item
  刀口辣椒 大さじ1杯 + 1食ごとに1杯

  \begin{itemize}
  \tightlist
  \item
    刀口辣椒の作り方は別項参照
  \end{itemize}
\item
  中華風スープまたは熱湯 300 ml

  \begin{itemize}
  \tightlist
  \item
    私は口水鶏の茹で汁を良く使う
  \item
    怠け者向け: 豆腐の茹で汁を使う
  \end{itemize}
\item
  (オプション) 老抽

  \begin{itemize}
  \tightlist
  \item
    たまり醤油に近いもの
  \end{itemize}
\item
  砂糖 大さじ1杯
\item
  片栗粉 大さじ2杯
\item
  ラードまたはバター 30g
\item
  ニンニクの芽

  \begin{itemize}
  \tightlist
  \item
    なければネギ
  \item
    元のレシピでは葉ニンニクを使う
  \end{itemize}
\item
  ネギ油 50 ml
\end{itemize}

\hypertarget{ux9053ux5177-8}{%
\subsection{道具}\label{ux9053ux5177-8}}

\begin{itemize}
\tightlist
\item
  鍋
\item
  中華鍋

  \begin{itemize}
  \tightlist
  \item
    おそらくフライパンでも代用可
  \end{itemize}
\item
  中華お玉

  \begin{itemize}
  \tightlist
  \item
    200 ml くらいのものが使いやすい
  \end{itemize}
\item
  ジャーレン
\end{itemize}

\hypertarget{ux4f5cux308aux65b9-12}{%
\subsection{作り方}\label{ux4f5cux308aux65b9-12}}

\begin{enumerate}
\def\labelenumi{\arabic{enumi}.}
\tightlist
\item
  ニンニク, 生姜, 泡辣椒はみじん切りにする
\item
  鍋に油をはり, 大さじ1杯の塩を入れて軽く豆腐を茹でる
\item
  中華鍋に油をひき, さらにラードまたはバターを溶かす
\item
  肉を入れて強火でよく混ぜる
\item
  肉に火が通ってきたらニンニクと生姜と泡辣椒, そして豆板醤を入れる
\item
  全体が赤くなるまでかき混ぜつつ炒める
\item
  豆豉と刀口辣椒を入れ, 少し炒めたらすぐに中華スープまたは熱湯を入れる

  \begin{itemize}
  \tightlist
  \item
    水で味が薄まるので熱湯の場合はシャンタンやウェイパーなど中華風うま味調味料で味を補強すると良い
  \end{itemize}
\item
  中火にして砂糖と老抽を入れる
\item
  さいの目切りにした豆腐を入れる
\item
  3-4分ふつふつと沸き立つ程度に加熱する
\item
  弱火にし, 水に溶いた片栗粉を3回に分けて入れる

  \begin{itemize}
  \tightlist
  \item
    50 ml の水に対して, 小さじ半, 小さじ1, 小さじ1.5 の順で分けて入れる
  \item
    豆腐の容器を捨てずにここで使うと楽
  \end{itemize}
\item
  最後にネギ油をふりかける
\item
  盛り付ける際には刻んだニンニクの芽と刀口辣椒をトッピングする
\end{enumerate}

\hypertarget{ux88dcux8db3-13}{%
\subsection{補足}\label{ux88dcux8db3-13}}

王剛のレシピでは肉は50gとしているが,
何品も作るの労力を考え一品で腹持ちを良くするため多めにしている.

老抽はたまり醤油に近い製法の, 中国の醤油である.
比較的マイルドな味であり, ここではおもに色を濃くするのが目的のため,
省いても味は大きく変化しない.

体感での辛さは主に豆板醤と (青花椒を含んだ) 刀口辣椒の量に依存する.
例えば市販のインスタント食品『陳麻婆豆腐』などは花椒が多いため辛く感じるが,
このレシピならばむしろ味の濃さが印象付けられ相対的に辛さを感じにくいだろう.

あまり鍋を振る必要がないので, フライパンでもさほど支障がないと思われる.

\hypertarget{ux53c2ux8003ux8cc7ux6599-10}{%
\subsection{参考資料}\label{ux53c2ux8003ux8cc7ux6599-10}}

\begin{itemize}
\tightlist
\item
  美食作家王剛による料理動画
  \url{https://www.youtube.com/watch?v=USoC8AqirVA}
\item
  『美味四川』の記事「\href{https://sisen-recipe.com/sichuan/2016/03/1.html}{永久保存版!本場の四川で食べる麻婆豆腐の作り方}」
\item
  『メシ通』の記事「\href{https://www.hotpepper.jp/mesitsu/entry/noriki-washiya/18-00431}{「麻婆豆腐の作り方」を四川料理のスゴイ人に教わったら、目からウロコが3回落ちた}」
\end{itemize}

\hypertarget{ux56deux934bux8089-ux4e2d-ux56dbux5ddd}{%
\section{\texorpdfstring{回鍋肉 (中, 四川)
\index{回鍋肉}}{回鍋肉 (中, 四川) }}\label{ux56deux934bux8089-ux4e2d-ux56dbux5ddd}}

これも本場式の辛いレシピを紹介する.

TODO: 画像

\begin{tabular}[t]{rl}
\toprule
 & 難易度\\
\midrule
材料調達 & {\fontspec{Noto Sans CJK JP} ★★★☆☆ }\\
調理 & {\fontspec{Noto Sans CJK JP} ★★★☆☆ }\\
\bottomrule
\end{tabular}

\hypertarget{ux6750ux6599-5}{%
\subsection{材料}\label{ux6750ux6599-5}}

\begin{itemize}
\tightlist
\item
  ニンニク 3欠片
\item
  生姜
\item
  長ネギ 1本

  \begin{itemize}
  \tightlist
  \item
    葉ニンニクがあればつかう. ニンニクの芽でもよい
  \end{itemize}
\item
  豆豉 大さじ1

  \begin{itemize}
  \tightlist
  \item
    永川豆豉がよいらしい
  \end{itemize}
\item
  豆板醤 大さじ1

  \begin{itemize}
  \tightlist
  \item
    郫県豆板醤がよいらしい
  \end{itemize}
\item
  甜麺醤 小さじ1
\item
  醤油 小さじ1
\item
  泡辣椒 適量
\item
  砂糖 適量
\item
  うま味調味料 適量
\item
  酢 適量
\item
  豚バラ肉 300 g

  \begin{itemize}
  \tightlist
  \item
    塊のものを用意する
  \end{itemize}
\item
  下処理に必要な薬味

  \begin{itemize}
  \tightlist
  \item
    生姜 1欠片
  \item
    ネギ 適量
  \item
    料理酒 30cc
  \end{itemize}
\end{itemize}

\hypertarget{ux9053ux5177-9}{%
\subsection{道具}\label{ux9053ux5177-9}}

中華鍋でやったほうがそれっぽい

\hypertarget{ux4f5cux308aux65b9-13}{%
\subsection{作り方}\label{ux4f5cux308aux65b9-13}}

\begin{enumerate}
\def\labelenumi{\arabic{enumi}.}
\tightlist
\item
  鍋に水を張り, 薬味を加えて豚肉を20分下茹でする
\item
  肉を取り出し, 冷水で冷やす
\item
  肉を薄切りにする

  \begin{itemize}
  \tightlist
  \item
    肉の繊維にそって切るとやりやすい
  \end{itemize}
\item
  ネギを半分または4分割してぶつ切りにする

  \begin{itemize}
  \tightlist
  \item
    ニンニクの芽ならたたいて繊維をほぐし, ぶつ切りにする
  \item
    葉ニンニクなら単にぶつ切りするだけで良い
  \end{itemize}
\item
  生姜と泡辣椒を刻む
\item
  甜麺醤と醤油を混ぜておく (合わせ調味料)
\item
  鍋に油を引き, 最大火力で肉を炒める

  \begin{itemize}
  \tightlist
  \item
    これは余分な脂を落とす意図がある
  \end{itemize}
\item
  肉が焼けて丸まってきたら, 火からおろし豆板醤と豆豉を入れる
\item
  肉に全体的に色が移るまでよく混ぜながら弱火で炒める
\item
  刻んだ生姜, 泡辣椒を加え再度強火で炒める
\item
  酢, 合わせ調味料, 砂糖やうま味調味料で味を調整する
\item
  ニンニクの芽を入れて最大火力で炒める
\end{enumerate}

\hypertarget{ux88dcux8db3-14}{%
\subsection{補足}\label{ux88dcux8db3-14}}

最初の下茹でで十分に火が通っており, また薄切りにしているため,
2度目に加熱する際は焦がさないよう火加減に注意して行うとよい.
特に最後の工程は太いネギやニンニクの芽を使うと火が通る前に肉が焦げてしまうため,
切り方にも注意する.

王剛の動画では皮付きの豚肉を扱っているが,
日本ではあまり売ってないのでその処理は省略した.

\hypertarget{ux53c2ux8003ux8cc7ux6599-11}{%
\subsection{参考資料}\label{ux53c2ux8003ux8cc7ux6599-11}}

\begin{itemize}
\tightlist
\item
  王剛の料理動画

  \begin{itemize}
  \tightlist
  \item
    その1 \url{https://www.youtube.com/watch?v=v72yoabCHXA}
  \item
    その2 \url{https://www.youtube.com/watch?v=DIMwUp9UqB4}
  \end{itemize}
\end{itemize}

\hypertarget{ux9178ux6e6fux80a5ux725b-ux4e2d-ux56dbux5ddd}{%
\section{\texorpdfstring{酸湯肥牛 (中, 四川)
\index{酸湯肥牛}}{酸湯肥牛 (中, 四川) }}\label{ux9178ux6e6fux80a5ux725b-ux4e2d-ux56dbux5ddd}}

花椒の麻味と漬け物から染み出した酸味が強いので暑い日によいかもしれない

TODO: 画像

\begin{tabular}[t]{rl}
\toprule
 & 難易度\\
\midrule
材料調達 & {\fontspec{Noto Sans CJK JP} ★★☆☆☆ }\\
調理 & {\fontspec{Noto Sans CJK JP} ★★☆☆☆ }\\
\bottomrule
\end{tabular}

\hypertarget{ux6750ux6599-6}{%
\subsection{材料}\label{ux6750ux6599-6}}

\begin{itemize}
\tightlist
\item
  泡酸菜や泡酸萝卜

  \begin{itemize}
  \tightlist
  \item
    なければ高菜漬けや野沢菜漬など乳酸発酵による漬け物
    (糠臭くないものや麹発酵してない漬け物ならだいたい可).
  \end{itemize}
\item
  牛切り落とし

  \begin{itemize}
  \tightlist
  \item
    薄く脂身の多い牛肉ならなんでも可. 安いものでいい.
  \end{itemize}
\item
  えのき
\item
  泡辣椒
\item
  ニンニク
\item
  生姜

  \begin{itemize}
  \tightlist
  \item
    できれば泡子姜がいいが, 日本で売ってるのを見たことがない.
  \end{itemize}
\item
  料理酒
\item
  酢
\item
  青唐辛子
\item
  花椒
\end{itemize}

\hypertarget{ux4f5cux308aux65b9-14}{%
\subsection{作り方}\label{ux4f5cux308aux65b9-14}}

\begin{enumerate}
\def\labelenumi{\arabic{enumi}.}
\tightlist
\item
  ニンニクと生姜と青唐辛子をみじん切りにする
\item
  えのきの石づきを切り取り, 適当にほぐす
\item
  湯を沸かし, 水と油を少々入れてえのきを湯通しする
\item
  同じ湯で肉も湯通しする
\item
  フライパンにラードをひき, ニンニク, 生姜, 漬け物を炒める
\item
  先ほどの湯を適量加える

  \begin{itemize}
  \tightlist
  \item
    味が薄いと感じたらうま味調味料を足す
  \end{itemize}
\item
  漬け物を取り除く
\item
  肉を入れて煮る
\item
  料理酒と酢を少々入れる
\item
  泡酸菜を煮込んでだしをとる
\item
  泡酸菜を除き, えのきをいれて茹でる
\item
  肉を茹でる
\item
  皿に盛り付け, 上から青唐辛子や花椒をふりかける
\item
  少々の油を強く熱して青唐辛子にかける
\end{enumerate}

\hypertarget{ux88dcux8db3-15}{%
\subsection{補足}\label{ux88dcux8db3-15}}

この料理では漬け物をだしを取るためだけに使う.
もったいないと思ったら食べても構わない.

私はこの料理を都内の料理店「\href{http://www.tenfufan.com/}{天府舫}」で知った.
ここ提供されるものは花椒の実がまるごと浮かんでおり, 非常に辛かった.

\hypertarget{ux53c2ux8003ux8cc7ux6599-12}{%
\subsection{参考資料}\label{ux53c2ux8003ux8cc7ux6599-12}}

\begin{itemize}
\tightlist
\item
  『美味四川』の記事「\href{https://sisen-recipe.com/sichuan/2018/07/2358.html}{暑い日に食べたい酸っぱ辛い四川料理「酸湯肥牛」の作り方}」
\item
  『美食台』の動画 (中国語)
  \url{https://www.youtube.com/watch?v=8SlM9nacW0g}
\end{itemize}

\hypertarget{ux53e3ux6c34ux9d8f-ux4e2d-ux56dbux5ddd}{%
\section{\texorpdfstring{口水鶏 (中,
四川)\index{口水鶏}\index{口水雞|see{口水鶏}}\index{口水鸡|see{口水鶏}}}{口水鶏 (中, 四川)}}\label{ux53e3ux6c34ux9d8f-ux4e2d-ux56dbux5ddd}}

いわゆる「よだれ鶏」\index{よだれ鶏|see{口水鶏}}日本で出回っているものはちょっと辛い棒々鶏
(棒々鶏も本場のものは辛い) みたいなのが多いが,
本場では赤黒いソースをかけた見るからに辛そうな料理である
(図\ref{fig:yodare-finished}).

\begin{figure}

{\centering \includegraphics[width=1\linewidth,height=1\textheight,keepaspectratio]{img/yodare/finished} 

}

\caption{モモ肉の口水鶏}\label{fig:yodare-finished}
\end{figure}

\begin{tabular}[t]{rl}
\toprule
 & 難易度\\
\midrule
材料調達 & {\fontspec{Noto Sans CJK JP} ★★★☆☆ }\\
調理 & {\fontspec{Noto Sans CJK JP} ★★☆☆☆ }\\
\bottomrule
\end{tabular}

\hypertarget{ux6750ux6599-2ux4ebaux524d}{%
\subsection{材料 (2人前)}\label{ux6750ux6599-2ux4ebaux524d}}

鶏の下処理に必要な材料

\begin{itemize}
\tightlist
\item
  鶏もも肉 2枚

  \begin{itemize}
  \tightlist
  \item
    骨付きのほうが見た目がかっこいいが食べやすさを考えると骨抜きが楽
  \end{itemize}
\item
  生姜 1かけ
\item
  料理酒 50 cc
\item
  ネギ 1/2本
\end{itemize}

ソースに必要な材料

\begin{itemize}
\tightlist
\item
  ニンニク 20 g
\item
  生姜 10 g
\item
  唐辛子 10 g
\item
  刀口辣椒 大さじ1杯
\item
  香辣紅油 50 cc
\item
  醤油 少々
\item
  塩 少々
\item
  ネギまたはコリアンダーの葉
\end{itemize}

\hypertarget{ux9053ux5177-10}{%
\subsection{道具}\label{ux9053ux5177-10}}

鶏を1羽丸ごと茹でる場合は,
十分に大きな寸胴鍋を用意しないと均等に加熱できない(図\ref{fig:yodare-smaller-pot})

\begin{figure}

{\centering \includegraphics[width=1\linewidth,height=1\textheight,keepaspectratio]{img/yodare/smaller} 

}

\caption{鍋が小さいと完全に入り切らない}\label{fig:yodare-smaller-pot}
\end{figure}

\hypertarget{ux4f5cux308aux65b9-15}{%
\subsection{作り方}\label{ux4f5cux308aux65b9-15}}

鶏の下処理

\begin{enumerate}
\def\labelenumi{\arabic{enumi}.}
\tightlist
\item
  鍋で湯を沸かす
\item
  鶏を下茹でし, 灰汁を取る
\item
  鶏を取り出し湯を捨て, 再度湯を沸かす
\item
  臭い取りのため生姜, ネギ, 料理酒を入れる
\item
  鶏肉を入れ, 茹でる

  \begin{itemize}
  \tightlist
  \item
    沸騰してから10分くらい?
  \end{itemize}
\item
  肉を取り出し, 粗熱をとる

  \begin{itemize}
  \tightlist
  \item
    冷水で冷ましても良い
  \end{itemize}
\end{enumerate}

ソースを作る

\begin{enumerate}
\def\labelenumi{\arabic{enumi}.}
\tightlist
\item
  ニンニク, 生姜, 唐辛子をみじん切りにする, もしくはすりおろす
\item
  熱した香辣紅油をかける
\item
  茹で汁の一部, 醤油を加え, 塩で味を調える
\end{enumerate}

\begin{itemize}
\tightlist
\item
  オイスターソースを隠し味にするのも良い
\end{itemize}

盛り付け

\begin{enumerate}
\def\labelenumi{\arabic{enumi}.}
\tightlist
\item
  鶏肉を適当な大きさに切り分ける
\item
  皿に盛り付け, 上からソースをかける
\item
  お好みでネギやコリアンダーの葉をふりかける
\end{enumerate}

\hypertarget{ux88dcux8db3-16}{%
\subsection{補足}\label{ux88dcux8db3-16}}

灰汁をとるのに2度茹でるのが面倒で,
灰汁が気にならないなら一度目から下味つけをしてもよい

生の唐辛子の入手が難しいなら, 泡辣椒でもよい.

ソースは別の作り方もある. これは以下の『简单 家常
美味!』の動画で紹介されたものに基づいている.
必要な調味料の種類が多いが,
調理自体はほぼ混ぜるだけなのでこっちのほうが簡単である.
この場合は以下を混ぜて作る.

\begin{itemize}
\tightlist
\item
  サラダ油 60 ml
\item
  ニンニク 20 g
\item
  唐辛子または青唐辛子, ししとう 10 g
\item
  万能ネギ 少々
\item
  ピーナッツ 10 g
\item
  醤油 大さじ 3
\item
  辣油 大さじ 3
\item
  ごま油 大さじ 1
\item
  花椒油 小さじ 1
\item
  蚝油 (オイスターソース) 小さじ 1
\end{itemize}

\begin{enumerate}
\def\labelenumi{\arabic{enumi}.}
\item
  ニンニク, ネギ, 唐辛子, ピーナッツを細かく刻む
\item
  油を130度くらいまで加熱する
\item
  \begin{enumerate}
  \def\labelenumii{(\arabic{enumii})}
  \tightlist
  \item
    に油を注ぐ
  \end{enumerate}

  \begin{itemize}
  \tightlist
  \item
    これでニンニクが加熱される
  \item
    熱すぎると唐辛子が焦げるので注意
  \end{itemize}
\item
  残りの材料をよく混ぜる
\end{enumerate}

\hypertarget{ux53c2ux8003ux8cc7ux6599-13}{%
\subsection{参考資料}\label{ux53c2ux8003ux8cc7ux6599-13}}

\begin{itemize}
\item
  鶏を〆る方法含んだ王剛の料理動画 (中国語, 英語字幕あり)
  \url{https://www.youtube.com/watch?v=99nspRiav-A}
\item
  王剛の香辣红油の作成動画 (中国語, 英語字幕あり)
  \url{https://www.youtube.com/watch?v=6wlDqKt2ADo}
\item
  『简单 家常 美味!』の動画
  \url{https://www.youtube.com/watch?v=laB1e6fWbmA}
\end{itemize}

\hypertarget{todo-ux30a4ux30f3ux30c9ux30abux30ecux30fc-ux5370}{%
\section{(TODO) インドカレー
(印)}\label{todo-ux30a4ux30f3ux30c9ux30abux30ecux30fc-ux5370}}

TODO

\begin{tabular}[t]{rl}
\toprule
 & 難易度\\
\midrule
材料調達 & {\fontspec{Noto Sans CJK JP} ★★★★☆ }\\
調理 & {\fontspec{Noto Sans CJK JP} ★★☆☆☆ }\\
\bottomrule
\end{tabular}

\hypertarget{ch-carbon}{%
\chapter{炭水化物編}\label{ch-carbon}}

ジャガイモや小麦粉を大量に使う料理.

\hypertarget{ux30d6ux30e9ux30f3ux30dcux30e9ux30fcux30af-ux6377-bramboruxe1ky}{%
\section{\texorpdfstring{ブランボラーク (捷:
Bramboráky)\index{ブランボラーク}\index{bramboráky|see{ブランボラーク}}}{ブランボラーク (捷: Bramboráky)}}\label{ux30d6ux30e9ux30f3ux30dcux30e9ux30fcux30af-ux6377-bramboruxe1ky}}

チェコの伝統的なファストフード. ジャガイモのかき揚げ.

\begin{figure}

{\centering \includegraphics[width=1\linewidth,height=1\textheight,keepaspectratio]{img/bramborak/finished} 

}

\caption{ブランボラーク}\label{fig:bramborak-finished}
\end{figure}

\begin{tabular}[t]{rl}
\toprule
 & 難易度\\
\midrule
材料調達 & {\fontspec{Noto Sans CJK JP} ★★☆☆☆ }\\
調理 & {\fontspec{Noto Sans CJK JP} ★★☆☆☆ }\\
\bottomrule
\end{tabular}

\hypertarget{ux6750ux6599-5-6ux679a}{%
\subsection{材料 (5-6枚)}\label{ux6750ux6599-5-6ux679a}}

\begin{itemize}
\tightlist
\item
  ジャガイモ
\item
  小麦粉
\item
  卵
\item
  ニンニク
\item
  (オプション) マジョラムまたはパセリ
\item
  (オプション) キャラウェイシード
\item
  塩胡椒
\item
  ラード 100g 程度

  \begin{itemize}
  \tightlist
  \item
    なければバターやサラダ油
  \end{itemize}
\end{itemize}

\hypertarget{ux9053ux5177-11}{%
\subsection{道具}\label{ux9053ux5177-11}}

\begin{itemize}
\tightlist
\item
  ジャガイモをすりおろすスライサー
\item
  ボウル
\item
  お玉
\item
  大きめのフライパン
\item
  揚げ終わった後で油を取る手段

  \begin{itemize}
  \tightlist
  \item
    新聞紙でもいい
  \end{itemize}
\item
  フライ返しと菜箸, あるいはフォークなどがあるとよい
\end{itemize}

\hypertarget{ux4f5cux308aux65b9-16}{%
\subsection{作り方}\label{ux4f5cux308aux65b9-16}}

\begin{enumerate}
\def\labelenumi{\arabic{enumi}.}
\tightlist
\item
  ジャガイモの皮をむいて細かくすりおろす

  \begin{itemize}
  \tightlist
  \item
    粒度は好み次第. 細切れにするとかき揚げのようになる
  \item
    軽く手で抑えて水気を切ると良い
  \end{itemize}
\item
  小麦粉, 卵, みじん切りにしたニンニク,
  スパイス類を加えてよく混ぜる(図\ref{fig:bramborak-dough})
\item
  フライパンを中火にかけ, ラードまたは油を多めにひく

  \begin{itemize}
  \tightlist
  \item
    生地が半分浸かるくらいは必要
  \item
    天ぷらのように丸ごと浸かる必要はない
  \end{itemize}
\item
  お好み焼きの容量で生地をお玉ですくって垂らし, 軽く押して平たくする
\item
  片面が固まって焼き目が付いたら油ハネに注意してひっくり返す

  \begin{itemize}
  \tightlist
  \item
    フライ返しと菜箸をそれぞれ両手に持ってやると丁寧に返せる
  \end{itemize}
\item
  反対側も同じように焼けたら取り出し, 油を切る
\item
  やけどしないように食べる
\end{enumerate}

\begin{figure}

{\centering \includegraphics[width=1\linewidth,height=1\textheight,keepaspectratio]{img/bramborak/dough} 

}

\caption{揚げる直前}\label{fig:bramborak-dough}
\end{figure}

\hypertarget{ux88dcux8db3-17}{%
\subsection{補足}\label{ux88dcux8db3-17}}

チェコ語でジャガイモのことをブランボル (brambor) というので,
ブランボラークという名前はストレートにジャガイモ料理であることを示唆している.
サワークリームを添えて食べることもある. ラトケ (latke)
やウクライナのデルニー (деруни), ベラルーシのドラニキ (драники)
など類似料理がいくつかある.
デルニーは生地にもサワークリームやすり下ろした玉ねぎを混ぜる.

\hypertarget{ux53c2ux8003ux8cc7ux6599-14}{%
\subsection{参考資料}\label{ux53c2ux8003ux8cc7ux6599-14}}

\begin{itemize}
\tightlist
\item
  動画 (英語音声) \url{https://www.youtube.com/watch?v=pTInrevl54Q}
\item
  Václav Frič による料理動画 (チェコ語音声)
  \url{https://www.youtube.com/watch?v=lUGyggN9ygY}
\end{itemize}

\hypertarget{ux30afux30cdux30c9ux30eaux30fcux30af-ux6377-knedluxedky}{%
\section{\texorpdfstring{クネドリーク (捷: Knedlíky)
\index{クネドリーク}\index{knedlíky|see{クネドリーク}}}{クネドリーク (捷: Knedlíky) }}\label{ux30afux30cdux30c9ux30eaux30fcux30af-ux6377-knedluxedky}}

ドイツのゼンメルクネーデル\index{ゼンメルクネーデル|see{クネドリーク}}\index{semmelknödel|see{ゼンメルクネーデル}}と関連があるが,
チェコのクネドリークはジャガイモがメインだったりパンくずを整形したり作り方がいろいろある.
ここでは最も簡単と思われる, 発酵もさせず,
主にジャガイモを使うクネドリークを紹介する.

\begin{tabular}[t]{rl}
\toprule
 & 難易度\\
\midrule
材料調達 & {\fontspec{Noto Sans CJK JP} ★★☆☆☆ }\\
調理 & {\fontspec{Noto Sans CJK JP} ★★☆☆☆ }\\
\bottomrule
\end{tabular}

\hypertarget{ux6750ux6599-7}{%
\subsection{材料}\label{ux6750ux6599-7}}

\begin{itemize}
\tightlist
\item
  ジャガイモ (男爵いもが良い) 3-4 個
\item
  小麦粉 50 cc
\item
  卵 1個
\item
  塩 少々
\end{itemize}

\hypertarget{ux9053ux5177-12}{%
\subsection{道具}\label{ux9053ux5177-12}}

ラップだけだとほどけることがあるのでラップの上から形状維持しやすいアルミホイルで包むと良い.

\begin{itemize}
\tightlist
\item
  ラップ
\item
  アルミホイル
\end{itemize}

\hypertarget{ux4f5cux308aux65b9-17}{%
\subsection{作り方}\label{ux4f5cux308aux65b9-17}}

\begin{itemize}
\tightlist
\item
  じゃがいもの皮をむき, 下茹でする

  \begin{itemize}
  \tightlist
  \item
    弱火で40分程度かかるだろう
  \end{itemize}
\item
  ポテトピューレのように念入りに潰す
\item
  粗熱が取れるまで待つ
\item
  小麦粉と卵と塩を加え, よく混ぜてこねる
\item
  整形して, ラップで包み, さらにアルミホイルで包む

  \begin{itemize}
  \tightlist
  \item
    よくあるパターンは, (1) 細長く (2) 丸く (3) ボート型
  \end{itemize}
\item
  10分ほど茹でる

  \begin{itemize}
  \tightlist
  \item
    適切な時間は当然ながら大きさに依存する
  \end{itemize}
\end{itemize}

\hypertarget{ux88dcux8db3-18}{%
\subsection{補足}\label{ux88dcux8db3-18}}

オーストリアや南ドイツ,
あるいはチェコ国内のカルロヴィ・ヴァリではパンを細かくちぎったものを混ぜることもあるらしい.

グラーシュ (セクション\ref{gulas}) の付け合せに, あるいは vepřo knedlo
zelo を作ってビールと一緒に食べるとよい.

\hypertarget{ux53c2ux8003ux8cc7ux6599-15}{%
\subsection{参考資料}\label{ux53c2ux8003ux8cc7ux6599-15}}

\begin{itemize}
\tightlist
\item
  \url{http://www.czechfriend.jp/event/040826.htm}
\item
  ``Kuchyně Lidlu'' による標準的なクネドリークの料理動画 (チェコ語)
  \url{https://www.youtube.com/watch?v=rgouyUynrlA}
\item
  ``Kuchyně Lidlu'' による「パンくずのクネドリーク」の料理動画
  (チェコ語) \url{https://www.youtube.com/watch?v=4_KnXyIiZds}
\item
  ``Recepty.cz'' による「カルロヴィ・ヴァリ風」クネドリーク (チェコ語)
  \url{https://www.recepty.cz/recept/karlovarsky-knedlik-140268}
\item
  ``Kuchyně Lidlu'' のカルロヴィ・ヴァリ風の作成動画 (チェコ語)
  \url{https://www.youtube.com/watch?v=SIee8PWWSbI}
\end{itemize}

\hypertarget{ux30d4ux30a8ux30edux30ae-ux6ce2-pierogi}{%
\section{\texorpdfstring{ピエロギ (波:
Pierogi)\index{ピエロギ}\index{pierogi|see{ピエロギ}}}{ピエロギ (波: Pierogi)}}\label{ux30d4ux30a8ux30edux30ae-ux6ce2-pierogi}}

餃子と似ているがひき肉を入れることはあまりない. 最もポピュラーな Pierogi
Ruskie と Pierogi kapustą i grzybami (ザワークラウトとキノコのピエロギ)
の例を紹介する.

\begin{figure}

{\centering \includegraphics[width=1\linewidth,height=1\textheight,keepaspectratio]{img/pierogi/finished} 

}

\caption{ピエロギ}\label{fig:finished-pierogi}
\end{figure}

\begin{tabular}[t]{rl}
\toprule
 & 難易度\\
\midrule
材料調達 & {\fontspec{Noto Sans CJK JP} ★★★☆☆ }\\
調理 & {\fontspec{Noto Sans CJK JP} ★★☆☆☆ }\\
\bottomrule
\end{tabular}

\hypertarget{ux6750ux6599-20ux500bux7a0bux5ea6}{%
\subsection{材料 (20個程度?)}\label{ux6750ux6599-20ux500bux7a0bux5ea6}}

\begin{itemize}
\tightlist
\item
  小麦粉 600cc
\item
  塩 小さじ1/2
\item
  水 150cc
\item
  室温に戻した卵 1 個
\end{itemize}

詰め物 (Ruskie の場合)

\begin{itemize}
\tightlist
\item
  ジャガイモ 2-3 個
\item
  チーズ 100g

  \begin{itemize}
  \tightlist
  \item
    リコッタかカッテージチーズが良い
  \end{itemize}
\item
  玉ねぎ 1/2 個
\item
  塩 小さじ1/2
\item
  胡椒
\end{itemize}

詰め物 (キノコとキャベツの場合)

\begin{itemize}
\item
  ザワークラウト 適量
\item
  キノコ 適量
\item
  玉ねぎ 1/2 個
\item
  (オプション) トッピング・付け合せ

  \begin{itemize}
  \tightlist
  \item
    玉ねぎ
  \item
    バター
  \item
    サワークリーム
  \end{itemize}
\end{itemize}

\hypertarget{ux9053ux5177-13}{%
\subsection{道具}\label{ux9053ux5177-13}}

\begin{itemize}
\tightlist
\item
  大きめのボウル
\item
  延し台
\end{itemize}

\hypertarget{ux4f5cux308aux65b9-18}{%
\subsection{作り方}\label{ux4f5cux308aux65b9-18}}

\begin{enumerate}
\def\labelenumi{\arabic{enumi}.}
\tightlist
\item
  生地の材料をよく混ぜてこねる

  \begin{itemize}
  \tightlist
  \item
    こねすぎてもいけない. 50回程度が良いらしい
  \end{itemize}
\item
  布かラップを被せ10分寝かせる
\item
  直径7cm程度の薄い円形に延ばす
\item
  大さじ1杯程度の詰め物を乗せて包む

  \begin{itemize}
  \tightlist
  \item
    だいたい餃子と同じ要領で良い
  \item
    縁にフォークを押し付けても良い
  \end{itemize}
\item
  3分煮る

  \begin{itemize}
  \tightlist
  \item
    煮たピエロギを並べるとき, トレーに油を塗っておくと良い
  \end{itemize}
\item
  さらにこの後, バターで軽く焼き目を付けても良い
\item
  サワークリームや時間を掛けて炒めた玉ねぎを添えて食べる
\end{enumerate}

Ruskie の場合の詰め物は,
マッシュポテトにリコッタチーズ・みじん切りにして炒めた玉ねぎ・塩小さじ半を混ぜる
キャベツとキノコの場合は, 材料を (ザワークラウトを特に念入りに)
みじん切りにした上で炒め, 塩胡椒で味を調える.

\hypertarget{ux88dcux8db3-19}{%
\subsection{補足}\label{ux88dcux8db3-19}}

果物やジャムを入れることもある. スモモジャム (ポヴィドル) がポピュラー.
17世紀のポーランドの料理本 ``Compendium Ferculorum''
には現存する最も古いピエロギのレシピが記載されており,
当時から甘い物や肉など様々な具材を入れていたことが分かっている.
ロシアやウクライナにもピローク (пирог) というパイ料理があり,
名称は似ているが作り方は異なる.

\hypertarget{ux53c2ux8003ux8cc7ux6599-16}{%
\subsection{参考資料}\label{ux53c2ux8003ux8cc7ux6599-16}}

\begin{itemize}
\tightlist
\item
  The Polish Chef による Ruskie の作り方
  \url{https://www.youtube.com/watch?v=GQ0GiTKzu38}
\item
  日本語のレシピ \url{https://jpya.or.jp/ja/2018/12/pierogi/}
\item
  \citet{czerniecki1682Compendium} pp.86-87 (ほとんど読めない)
\end{itemize}

\hypertarget{ux4e88ux5b9a-ux30daux30eaux30e1ux30cb-ux9732-ux43fux435ux43bux44cux43cux435ux43dux438}{%
\section{\texorpdfstring{(予定) ペリメニ (露:
Пельмени)\index{ペリメニ}\index{пельмени|see{ペリメニ}}}{(予定) ペリメニ (露: Пельмени)}}\label{ux4e88ux5b9a-ux30daux30eaux30e1ux30cb-ux9732-ux43fux435ux43bux44cux43cux435ux43dux438}}

ゲニスとワイリは「怠け者のための料理」と呼んでいるが,
皮を作るのが少し面倒なので手抜きリストには入れなかった.

TODO: 画像

\begin{tabular}[t]{rl}
\toprule
 & 難易度\\
\midrule
材料調達 & {\fontspec{Noto Sans CJK JP} ★★☆☆☆ }\\
調理 & {\fontspec{Noto Sans CJK JP} ★★☆☆☆ }\\
\bottomrule
\end{tabular}

\hypertarget{ux6750ux6599-8}{%
\subsection{材料}\label{ux6750ux6599-8}}

\begin{itemize}
\tightlist
\item
  小麦粉
\item
  水
\item
  卵
\item
  塩
\item
  牛または豚の挽き肉
\end{itemize}

\hypertarget{ux4f5cux308aux65b9-19}{%
\subsection{作り方}\label{ux4f5cux308aux65b9-19}}

バター

\begin{enumerate}
\def\labelenumi{\arabic{enumi}.}
\tightlist
\item
  茹でたらバターを入れたボウルに投入し, 余熱でバターを溶かす
\item
  ボウルをよく振ってバターを絡める
\item
  例によってパセリやディルやサワークリームなどを添えて食べる.
\end{enumerate}

\hypertarget{ux88dcux8db3-20}{%
\subsection{補足}\label{ux88dcux8db3-20}}

餃子より一回り小さい. 市販の餃子の皮は大きすぎるので向かない

\begin{itemize}
\tightlist
\item
  茹でる直前の状態で,
  くっつかないように小分けして冷凍しておけば好きな時に食べられる
\item
  ロシアでは家庭でペリメニを一気に大量に作るため,
  ハニカム構造の型が流通している

  \begin{itemize}
  \tightlist
  \item
    型抜きで生地を分けるなら,
    ウォッカ用のショットグラスがちょうどいい大きさになる.
  \item
    参考: \url{https://www.youtube.com/watch?v=_o9934oSYqQ}
  \end{itemize}
\end{itemize}

\hypertarget{ux53c2ux8003ux8cc7ux6599-17}{%
\subsection{参考資料}\label{ux53c2ux8003ux8cc7ux6599-17}}

\begin{itemize}
\tightlist
\item
  Коллекция Рецептов の料理動画
  \url{https://www.youtube.com/watch?v=N9aS8jBaTS8}
\item
  \citet{OginoNumano2017} p.~54
\item
  \citet{boumei} Ch. 27, および付録のサラファン式レシピ
\end{itemize}

\hypertarget{ux30d2ux30f3ux30abux30ea-ux8349-ux10eeux10d8ux10dcux10d9ux10d0ux10daux10d8}{%
\section{\texorpdfstring{ヒンカリ (草:
ხინკალი)\index{ヒンカリ}\index{ხინკალი|see{ヒンカリ}}}{ヒンカリ (草: ხინკალი)}}\label{ux30d2ux30f3ux30abux30ea-ux8349-ux10eeux10d8ux10dcux10d9ux10d0ux10daux10d8}}

小籠包に似たジョージア料理. 餃子や小籠包より一回り大きく, 皮が厚い.

\begin{figure}

{\centering \includegraphics[width=1\linewidth,height=1\textheight,keepaspectratio]{img/khinkali/finished} 

}

\caption{ヒンカリ}\label{fig:khinkali-finished}
\end{figure}

\begin{tabular}[t]{rl}
\toprule
 & 難易度\\
\midrule
材料調達 & {\fontspec{Noto Sans CJK JP} ★★★☆☆ }\\
調理 & {\fontspec{Noto Sans CJK JP} ★★☆☆☆ }\\
\bottomrule
\end{tabular}

\hypertarget{ux6750ux6599-9}{%
\subsection{材料}\label{ux6750ux6599-9}}

生地

皮が薄い市販の餃子の皮は向かない.

\begin{itemize}
\tightlist
\item
  小麦粉
\item
  水
\item
  (オプション) 卵

  \begin{itemize}
  \tightlist
  \item
    使っても使わなくてもいい
  \item
    使う場合は水を少し減らす
  \end{itemize}
\item
  塩
\end{itemize}

以下は詰め物

\begin{itemize}
\tightlist
\item
  ひき肉 300 g

  \begin{itemize}
  \tightlist
  \item
    牛, 鶏, 羊, 豚などなんでもいい
  \item
    合い挽きでも可
  \end{itemize}
\item
  玉ねぎ 1個
\item
  ニンニク 3-4個
\item
  唐辛子 1本
\item
  コリアンダーの葉

  \begin{itemize}
  \tightlist
  \item
    種でも可
  \end{itemize}
\item
  ブイヨン または水
\item
  塩胡椒
\end{itemize}

\hypertarget{ux9053ux5177-14}{%
\subsection{道具}\label{ux9053ux5177-14}}

\begin{itemize}
\tightlist
\item
  ボウル
\item
  のし台
\item
  鍋
\end{itemize}

\hypertarget{ux4f5cux308aux65b9-20}{%
\subsection{作り方}\label{ux4f5cux308aux65b9-20}}

\begin{enumerate}
\def\labelenumi{\arabic{enumi}.}
\tightlist
\item
  ニンニク, 玉ねぎ, 唐辛子, コリアンダーの葉を刻む
\item
  挽き肉と混ぜる

  \begin{itemize}
  \tightlist
  \item
    伝統的なやり方にならって野菜と肉の塊と一緒に斧や重い剣で叩いてもよい
    (図\ref{fig:khinkali-minced})
  \end{itemize}
\item
  塩胡椒を加えてよく混ぜる
\item
  水を加えてよく混ぜる
\item
  生地を延ばして切り分ける

  \begin{itemize}
  \tightlist
  \item
    包み方のため, きれいに円形にならなくともなんとかなる
  \end{itemize}
\item
  大さじ1-1.5杯程度を包む (図\ref{fig:khinkali-pre-boil},
  \ref{fig:khinkali-wrapping})

  \begin{itemize}
  \tightlist
  \item
    包み方は動画等参照
  \end{itemize}
\item
  鍋に湯を沸かし, 塩を大さじ1杯入れる
\item
  ヒンカリを入れ茹でる

  \begin{itemize}
  \tightlist
  \item
    張り付かないようにかき混ぜて水流を作ると良い
  \item
    沸騰しすぎると良くない, 少し泡立つ程度で良い
  \end{itemize}
\item
  数分で浮かんでくるので取りだす (図\ref{fig:khinkali-boiling})
\end{enumerate}

\begin{figure}

{\centering \includegraphics[width=1\linewidth,height=1\textheight,keepaspectratio]{img/khinkali/minced} 

}

\caption{野菜とともにミンチにする}\label{fig:khinkali-minced}
\end{figure}

\begin{figure}

{\centering \includegraphics[width=1\linewidth,height=1\textheight,keepaspectratio]{img/khinkali/pre-boil} 

}

\caption{茹でる直前}\label{fig:khinkali-pre-boil}
\end{figure}

\begin{figure}

{\centering \includegraphics[width=1\linewidth,height=1\textheight,keepaspectratio]{img/khinkali/wrapping} 

}

\caption{包んでいる途中}\label{fig:khinkali-wrapping}
\end{figure}

\begin{figure}

{\centering \includegraphics[width=1\linewidth,height=1\textheight,keepaspectratio]{img/khinkali/boiling} 

}

\caption{ローレルは必須ではない}\label{fig:khinkali-boiling}
\end{figure}

\hypertarget{ux88dcux8db3-21}{%
\subsection{補足}\label{ux88dcux8db3-21}}

ヘタ部分をつまんで食べる. ヘタ部分は食べない.
これもタレはないので詰め物にしっかり味付けする.
胡椒などを軽く振ることはある.

現地人の動画では延ばしてから型抜きで切り分けているが,
台所がせまい場合は餃子のように生地を細長くしてちぎってから延ばすやり方でも良い.

\hypertarget{ux53c2ux8003ux8cc7ux6599-18}{%
\subsection{参考資料}\label{ux53c2ux8003ux8cc7ux6599-18}}

\begin{itemize}
\tightlist
\item
  伝統的な道具を使った料理法 (ジョージア語音声)
  \url{https://www.youtube.com/watch?v=YRNOKkblZzc}
\item
  日本在住のロシア人による料理動画 (日本語音声)
  \url{https://www.youtube.com/watch?v=-v2sk3QxVDc}
\item
  ジョージア在住の日本人による解説動画, 肉ではなくジャガイモ入り
  (日本語, ロシア語音声)
  \url{https://www.youtube.com/watch?v=Z7HJ1f-Qp0}
\end{itemize}

\hypertarget{ux4e88ux5b9a-ux30ddux30f3ux30c1ux30ad}{%
\section{(予定) ポンチキ}\label{ux4e88ux5b9a-ux30ddux30f3ux30c1ux30ad}}

\begin{tabular}[t]{rl}
\toprule
 & 難易度\\
\midrule
材料調達 & {\fontspec{Noto Sans CJK JP} ★★★★☆ }\\
調理 & {\fontspec{Noto Sans CJK JP} ★★☆☆☆ }\\
\bottomrule
\end{tabular}

\hypertarget{ux30e9ux30b0ux30deux30f3-ux30a6ux30a4ux30b0ux30eb}{%
\section{\texorpdfstring{ラグマン (ウイグル:
\ragman)\index{うどん!ラグマン}}{ラグマン (ウイグル:  )}}\label{ux30e9ux30b0ux30deux30f3-ux30a6ux30a4ux30b0ux30eb}}

ウイグル族をはじめ中央アジア各所で見られる料理.
中国語の「拉麺」の由来説がある\footnote{日本語のラーメンは拉麺説以外に異説があるらしいが詳しくは知らない.}.
日本のラーメンと違い, 本来の語義通り手延べ麺であることが特徴.
スープに漬けるものもあれば, 「混ぜそば」「焼きそば」風のものもあるが,
今回は簡単な混ぜそば風のものにする.

\begin{figure}

{\centering \includegraphics[width=1\linewidth,height=1\textheight,keepaspectratio]{img/laghman/finished} 

}

\caption{ラグマン}\label{fig:laghman-finished}
\end{figure}

\begin{tabular}[t]{rl}
\toprule
 & 難易度\\
\midrule
材料調達 & {\fontspec{Noto Sans CJK JP} ★★★★☆ }\\
調理 & {\fontspec{Noto Sans CJK JP} ★★★★★ }\\
\bottomrule
\end{tabular}

\hypertarget{ux6750ux6599-10}{%
\subsection{材料}\label{ux6750ux6599-10}}

麺に必要なもの

卵, 水, 小麦粉 400g

小麦粉と水は 1:4 くらい?

20分寝かせる

1.5 - 2mm に伸ばす 麺は切る派と伸ばしに伸ばして1本の長い麺にする派がいる

\begin{itemize}
\tightlist
\item
  小麦粉 400 g
\item
  水 小麦粉に対して 1/4 程度

  \begin{itemize}
  \tightlist
  \item
    卵で補っても良い?
  \end{itemize}
\item
  塩 ひとつまみ
\item
  植物油
\end{itemize}

具に必要なもの

\begin{itemize}
\tightlist
\item
  羊肉または鶏肉

  \begin{itemize}
  \tightlist
  \item
    牛肉でも良い
  \end{itemize}
\item
  玉ねぎ
\item
  人参
\item
  トマト
\item
  パプリカ
\item
  ニンニク
\item
  生姜
\item
  羊脂または植物油
\item
  塩胡椒
\item
  クミンシード
\end{itemize}

茄子やネギやいんげんを入れても良い

\hypertarget{ux9053ux5177-15}{%
\subsection{道具}\label{ux9053ux5177-15}}

\begin{itemize}
\tightlist
\item
  のし台
\item
  大きめの皿, またはフライパン
\end{itemize}

\hypertarget{ux4f5cux308aux65b9-21}{%
\subsection{作り方}\label{ux4f5cux308aux65b9-21}}

手延べ麺の作成

\begin{enumerate}
\def\labelenumi{\arabic{enumi}.}
\tightlist
\item
  小麦粉と水, 塩を混ぜてよくこねる
\item
  ふきんをかぶせて1時間ほど寝かせる
\item
  棒状に延ばす

  \begin{itemize}
  \tightlist
  \item
    延ばす際は打ち粉ではなく油を塗ってくっつかないようにすること
  \item
    この時点では太くても良い
  \item
    いくつかに分割してやると場所を取らない
  \end{itemize}
\item
  表面に油を塗る
\item
  もうすこし細長く延ばし, 渦巻き状にして皿に置く
  (図\ref{fig:laghman-spiral})
\item
  乾かないように表面に油を塗り, 1時間ほど寝かせる
\item
  先端から引っ張ってさらに少しづつ細く延ばす
\item
  うどんみたいに両手に巻き付けて振って延ばす
\item
  コップに水をいれておく
\item
  熱湯で茹でる, 激しく茹だったらコップ半分だけ室温に戻した水を入れる
\item
  もう一度茹だったら麺を取り出し, 冷水で締める
\end{enumerate}

\begin{itemize}
\tightlist
\item
  もう一度水を入れて, 三度目の沸騰で取りだすやり方もあるらしい
\end{itemize}

\begin{figure}

{\centering \includegraphics[width=1\linewidth,height=1\textheight,keepaspectratio]{img/laghman/spiral} 

}

\caption{作成中の麺}\label{fig:laghman-spiral}
\end{figure}

具の作成 (麺を寝かせている間に作るとよいだろう)

\begin{enumerate}
\def\labelenumi{\arabic{enumi}.}
\tightlist
\item
  肉と野菜を小さく切る
\item
  フライパンに油をひき, 炒める
\item
  塩胡椒とクミンシードで味付けする
\end{enumerate}

最後に麺の上に具を盛り付ける

\hypertarget{ux88dcux8db3-22}{%
\subsection{補足}\label{ux88dcux8db3-22}}

ロシアではヴォク (中国語の鑊が由来か)
と言う名前で野菜と鶏肉と太い麺を炒め,
アメリカのドラマでよく見かける箱に入れたファストフードが頻繁に見られた.
ラグマンと関係があるのかは不明

\hypertarget{ux53c2ux8003ux8cc7ux6599-19}{%
\subsection{参考資料}\label{ux53c2ux8003ux8cc7ux6599-19}}

\begin{itemize}
\tightlist
\item
  小泉武夫『食マガジン』での紹介
  \url{https://koizumipress.com/archives/6314}
\item
  中央アジア雑貨輸入販売業者『シルクロードキャラバン』によるレシピ『ウズベク料理
  ラグメンのレシピ』 \url{http://sr-caravan.com/?mode=f10}
\item
  ``UYGHUR HAND-PULLED NOODLES \textbar{} LAGHMAN'' (英語)
  \url{https://www.youtube.com/watch?v=XhrxYD4Zahg}
\item
  ``How to make Lagman Noodles Homemade Hand-pulled Noodle Uyghur
  Laghman'' (英語) \url{https://www.youtube.com/watch?v=AiMGPbMcAw0}
\item
  ロシア国内で知られているラグマンは手延ではないことが多く,
  またジャガイモやディルを使うなどロシア的ローカライズがなされている

  \begin{itemize}
  \tightlist
  \item
    「ウズベキスタン風」ラグマン (ロシア語)
    \url{https://www.youtube.com/watch?v=4QdaJoJpt5s}
  \item
    Всегда Вкусно! による「ウズベキスタン風」ラグマン (ロシア語)
    \url{https://www.youtube.com/watch?v=pF0MiqYUxLw}
  \end{itemize}
\end{itemize}

\hypertarget{ux30d7ux30edux30d5ux30d1ux30e9ux30d5-ux9732-ux43fux43bux43eux432ux6708-palov}{%
\section{\texorpdfstring{プロフ/パラフ (露: Плов/月: Palov)
\index{плов|see{プロフ}}
\index{palo|see{プロフ}}}{プロフ/パラフ (露: Плов/月: Palov)  }}\label{ux30d7ux30edux30d5ux30d1ux30e9ux30d5-ux9732-ux43fux43bux43eux432ux6708-palov}}

オシュ (Osh) とも. おそらくピラフと起源を同じにする,
ウズベキスタン周辺の中央アジア起源の料理だが,
ソ連時代にロシアに普及した. 真面目に作ろうとすると,
材料と器具の調達の点で炒飯やピラフより大変である.
ウズベキスタン国内でもタシケント式, サマルカンド式, フェルガナ式,
ブハラ式など地域によって差異がある.
東北地方の芋煮会めいてコミュニティで共同して作る習慣があるので一度に大量に作る前提のレシピになっている.
(現地にはプロフを作るための専用施設もある)
ロシアではスタローバヤの定番料理のため,
大衆向けの安価な食事というイメージらしい.

\begin{figure}

{\centering \includegraphics[width=1\linewidth,height=1\textheight,keepaspectratio]{img/palov/finished} 

}

\caption{プロフとアチク=チュチュク}\label{fig:finished-palov}
\end{figure}

\begin{tabular}[t]{rl}
\toprule
 & 難易度\\
\midrule
材料調達 & {\fontspec{Noto Sans CJK JP} ★★★★☆ }\\
調理 & {\fontspec{Noto Sans CJK JP} ★★★☆☆ }\\
\bottomrule
\end{tabular}

\hypertarget{ux6750ux6599-11}{%
\subsection{材料}\label{ux6750ux6599-11}}

\begin{itemize}
\tightlist
\item
  羊肉 200g

  \begin{itemize}
  \tightlist
  \item
    イスラム教徒が大半なので豚は使わない
  \end{itemize}
\item
  人参 1本

  \begin{itemize}
  \tightlist
  \item
    現地では黄色い人参がよく使われる. 沖縄の島にんじん的な?
  \end{itemize}
\item
  玉ねぎ 1個
\item
  ニンニク

  \begin{itemize}
  \tightlist
  \item
    皮も含めて丸ごと
  \end{itemize}
\item
  クミンシード 小さじ1
\item
  塩胡椒 適量
\item
  米 2合

  \begin{itemize}
  \tightlist
  \item
    ジャポニカ米は使われないようだが気にしない
  \end{itemize}
\item
  水 通常の炊飯と同じ量
\end{itemize}

以上は多くの地域で共通するもの. その他,
地域によって添えるものが変わってくるオプション

\begin{itemize}
\tightlist
\item
  干しぶどう
\item
  ひよこ豆
\item
  鶏やウズラのゆで卵
\item
  鶏肉や羊肉, 馬肉ソーセージなど複数種類の肉を入れることもある
\end{itemize}

\hypertarget{ux9053ux5177-16}{%
\subsection{道具}\label{ux9053ux5177-16}}

\begin{itemize}
\tightlist
\item
  カザン (大型で丸底の鉄鍋) --
  なければ中華鍋や底の深い平底鍋でもできなくはない
\item
  蓋になる大きめのボウル --
  現地の人がやってるように皿を載せるだけで密閉できなくともわりとなんとかなる
\end{itemize}

\hypertarget{ux4f5cux308aux65b9-22}{%
\subsection{作り方}\label{ux4f5cux308aux65b9-22}}

\begin{enumerate}
\def\labelenumi{\arabic{enumi}.}
\tightlist
\item
  鍋に洋脂を溶かす

  \begin{itemize}
  \tightlist
  \item
    肉から脂身を切り取って使う
  \item
    なければ植物油
  \end{itemize}
\item
  千切りにした玉ねぎと人参を入れ, 弱火でじっくり炒める
\item
  玉ねぎと人参はよく火が通り, 水分が抜けるまで炒める

  \begin{itemize}
  \tightlist
  \item
    焦げないように130度程度を保つと良い
  \end{itemize}
\item
  手頃な大きさに切った肉を入れる
\item
  ある程度火が通ったら, 米と水を入れる

  \begin{itemize}
  \tightlist
  \item
    米は事前に研がなくてもよい
  \end{itemize}
\item
  米は水平になるように均し, 水位が米と同じくらいの高さになるまで待つ
\item
  唐辛子, ニンニク, 荒く挽いたクミンシード, そして塩を入れる

  \begin{itemize}
  \tightlist
  \item
    クミンシードは手のひらこすり合わせるようにしてすりつぶすことができる
  \item
    肌が弱い人はやらないほうがいいが,
    慣れると意外と簡単に手で挽くことができる
  \end{itemize}
\item
  ボウルで蓋をする場合, 米を覆えるように鍋の中央に盛り上げる

  \begin{itemize}
  \tightlist
  \item
    底のほうにある野菜や肉は混ぜ返さないようにする
  \end{itemize}
\item
  水分が抜けやすいように長い菜箸かフォークで数カ所に穴を開ける
\item
  蓋をして20分程度待つ
\item
  火を止めて10分程度蒸らす
\item
  鍋底をさらうようによくかき混ぜてから盛り付ける
\item
  オプションでゆで卵や干しブドウを添える
\end{enumerate}

\hypertarget{ux88dcux8db3-23}{%
\subsection{補足}\label{ux88dcux8db3-23}}

米が丸底鍋の底に触れていると焦げやすいため,
下敷きになるように野菜は十分に入れる必要がある.

余裕があれば輪切りにした玉ねぎとトマトにコリアンダーと塩胡椒を振ったサラダ
(アチク=チュチュク) を作って一緒に食べる

材料を入れる順番はかなり差異がある.

\hypertarget{ux53c2ux8003ux8cc7ux6599-20}{%
\subsection{参考資料}\label{ux53c2ux8003ux8cc7ux6599-20}}

\begin{itemize}
\tightlist
\item
  \citet{OginoNumano2017} p.~83
\item
  ウズベク風プロフ (ロシア語)
  \url{https://www.youtube.com/watch?v=J-D3eZgos3I}
\item
  フェルガナ式プロフ (ロシア語)
  \url{https://www.youtube.com/watch?v=LDSO3W88QvU}
\item
  サマルカンド式 (ロシア語)
  \url{https://www.youtube.com/watch?v=6gnzLZdpxQs}
\item
  タシュケント式 (ロシア語)
  \url{https://www.youtube.com/watch?v=oYqqHNdoGMk}
\item
  炊飯器で作る方法 (日本語)
  \url{https://www.youtube.com/watch?v=ZVTa4OQ0atk}
\item
  アゼルバイジャン人による料理動画
  \url{https://www.youtube.com/watch?v=oAOYV9Y6MVc}
\item
  現地の「プロフセンター」の風景
\item
  e-food.jp の簡略化したレシピ (日本語)
  \url{https://e-food.jp/recipe/asia/samalkandplov/}
\end{itemize}

\hypertarget{ux30e1ux30c9ux30a5ux30efux30c0-ux5370-ux30bfux30dfux30ebux30bbux30a4ux30edux30f3}{%
\section{メドゥ・ワダ (印,
タミル/セイロン)}\label{ux30e1ux30c9ux30a5ux30efux30c0-ux5370-ux30bfux30dfux30ebux30bbux30a4ux30edux30f3}}

南インドの豆のドーナツ.
映画『バーフバリ』で一瞬ドーナツみたいな料理が皿に山盛りになってるシーンがあったので気になって調べたら見つけた.

\begin{figure}

{\centering \includegraphics[width=1\linewidth,height=1\textheight,keepaspectratio]{img/medu-vada/finished} 

}

\caption{メドゥ・ワダ}\label{fig:medu-vada-finished}
\end{figure}

\begin{tabular}[t]{rl}
\toprule
 & 難易度\\
\midrule
材料調達 & {\fontspec{Noto Sans CJK JP} ★★★★☆ }\\
調理 & {\fontspec{Noto Sans CJK JP} ★★★★☆ }\\
\bottomrule
\end{tabular}

\hypertarget{ux6750ux6599-12}{%
\subsection{材料}\label{ux6750ux6599-12}}

\begin{itemize}
\tightlist
\item
  ウラド・ダル (ケツルアズキ)
\item
  カレーリーフ
\item
  唐辛子
\item
  油
\item
  塩
\end{itemize}

\hypertarget{ux9053ux5177-17}{%
\subsection{道具}\label{ux9053ux5177-17}}

ミキサーがあったほうがよい

\hypertarget{ux4f5cux308aux65b9-23}{%
\subsection{作り方}\label{ux4f5cux308aux65b9-23}}

\begin{enumerate}
\def\labelenumi{\arabic{enumi}.}
\tightlist
\item
  ウラド・ダルを水に浸けて柔らかくする
\item
  水気を切って細かく潰す

  \begin{itemize}
  \tightlist
  \item
    ミキサーを使ったほうがいい
  \end{itemize}
\item
  冷蔵庫で2時間寝かせる
\item
  塩, カレーリーフ, 少量の唐辛子を混ぜる
\item
  小さく丸め, 指で穴を空けてドーナツ状にする
\item
  油で揚げる
\end{enumerate}

\hypertarget{ux88dcux8db3-24}{%
\subsection{補足}\label{ux88dcux8db3-24}}

ミキサーがないと困難. 粘りが出ずにすぐ崩れる.
小麦粉等をつなぎにするしかない.

\hypertarget{ux53c2ux8003ux8cc7ux6599-21}{%
\subsection{参考資料}\label{ux53c2ux8003ux8cc7ux6599-21}}

\begin{itemize}
\tightlist
\item
  \url{https://www.youtube.com/watch?v=Zjm9fQBBHiM}
\end{itemize}

\hypertarget{meet}{%
\chapter{肉編}\label{meet}}

\hypertarget{todo-ux30dfux30c6ux30a3ux30c6ux30a3ux3068ux30adux30e7ux30d5ux30c6}{%
\section{(TODO)
ミティティとキョフテ}\label{todo-ux30dfux30c6ux30a3ux30c6ux30a3ux3068ux30adux30e7ux30d5ux30c6}}

\begin{tabular}[t]{rl}
\toprule
 & 難易度\\
\midrule
材料調達 & {\fontspec{Noto Sans CJK JP} ★★☆☆☆ }\\
調理 & {\fontspec{Noto Sans CJK JP} ★★☆☆☆ }\\
\bottomrule
\end{tabular}

\hypertarget{ux6750ux6599-13}{%
\subsection{材料}\label{ux6750ux6599-13}}

\begin{itemize}
\tightlist
\item
  牛豚合い挽き肉

  \begin{itemize}
  \tightlist
  \item
    もちろん, キョフテには豚肉を使わない
  \end{itemize}
\item
  ベーキングパウダー
\item
  玉ねぎ
\item
  塩胡椒
\item
  クミンシード
\item
  パプリカ
\end{itemize}

\hypertarget{ux9053ux5177-18}{%
\subsection{道具}\label{ux9053ux5177-18}}

\begin{itemize}
\tightlist
\item
  オーブントースターまたは魚焼きグリル
\end{itemize}

\hypertarget{ux30d7ux30ebux30c9ux30ddux30fcux30af-ux5317ux7c73-pulled-pork}{%
\section{\texorpdfstring{プルドポーク (北米: pulled pork)
\index{プルドポーク}\index{pulled pork|see{プルドポーク}}}{プルドポーク (北米: pulled pork) }}\label{ux30d7ux30ebux30c9ux30ddux30fcux30af-ux5317ux7c73-pulled-pork}}

北米のバーベキュー料理の一種. じっくり火を通してほぐした豚の肩肉.

\begin{figure}

{\centering \includegraphics[width=1\linewidth,height=1\textheight,keepaspectratio]{img/pulled-pork/finished} 

}

\caption{プルドポークバーガー}\label{fig:pulled-pork-finished}
\end{figure}

\begin{tabular}[t]{rl}
\toprule
 & 難易度\\
\midrule
材料調達 & {\fontspec{Noto Sans CJK JP} ★★★★☆ }\\
調理 & {\fontspec{Noto Sans CJK JP} ★★☆☆☆ }\\
\bottomrule
\end{tabular}

\hypertarget{ux6750ux6599-14}{%
\subsection{材料}\label{ux6750ux6599-14}}

\begin{itemize}
\tightlist
\item
  豚ショルダー

  \begin{itemize}
  \tightlist
  \item
    赤身肉が適している
  \item
    脂身の多いバラ肉などは適していない
  \end{itemize}
\item
  砂糖

  \begin{itemize}
  \tightlist
  \item
    できればブラウンシュガー
  \end{itemize}
\item
  塩
\item
  ガーリックパウダー
\item
  唐辛子の粉末
\item
  パプリカパウダー
\item
  マスタード
\item
  バーベキューソース
\end{itemize}

\hypertarget{ux9053ux5177-19}{%
\subsection{道具}\label{ux9053ux5177-19}}

\begin{itemize}
\tightlist
\item
  オーブン (オーブントースター可)
\item
  燻製器
\end{itemize}

\hypertarget{ux4f5cux308aux65b9-24}{%
\subsection{作り方}\label{ux4f5cux308aux65b9-24}}

\begin{enumerate}
\def\labelenumi{\arabic{enumi}.}
\tightlist
\item
  塩, 砂糖, ガーリックパウダー, 唐辛子の粉末,
  パプリカパウダーを良く混ぜる
\item
  肉の表面の水分を良く拭き取る
\item
  肉によく塗る
\item
  6時間温燻または熱燻する

  \begin{itemize}
  \tightlist
  \item
    表面が黒く固くなるまで
  \item
    器具がなければ残りの材料を塗って鍋で柔らかくなるまで煮込む
  \end{itemize}
\item
  バーベキューソースとマスタードをたっぷり塗り, アルミホイルで包む
\item
  さらに20分ほど加熱する
\item
  フォークでよくほぐす
\item
  ハンバーガーなどにして食べる
\end{enumerate}

\hypertarget{ux88dcux8db3-25}{%
\subsection{補足}\label{ux88dcux8db3-25}}

燻製にする場合, 生のニンニクなど水分のあるものは肉に塗るべきではない.
燻製にしないなら気にしなくても良い.

\hypertarget{ux53c2ux8003ux8cc7ux6599-22}{%
\subsection{参考資料}\label{ux53c2ux8003ux8cc7ux6599-22}}

\hypertarget{ux30f4ux30a7ux30d7ux30b7ux30e7ux30afux30cdux30c9ux30edux30bcux30ed-ux6377-vepux159o-knedlo-zelo}{%
\section{\texorpdfstring{ヴェプショ=クネドロ=ゼロ (捷:
Vepřo-Knedlo-Zelo)\index{ヴェプショ=クネドロ=ゼロ}\index{vepřo-knedlo-zelo|see{ヴェプショ=クネドロ=ゼロ}}}{ヴェプショ=クネドロ=ゼロ (捷: Vepřo-Knedlo-Zelo)}}\label{ux30f4ux30a7ux30d7ux30b7ux30e7ux30afux30cdux30c9ux30edux30bcux30ed-ux6377-vepux159o-knedlo-zelo}}

「豚肉・クネドリーク・キャベツ」という意味になる. ローストポーク,
クネドリーク,
甘酸っぱく炒めたザワークラウトを並べたチェコの居酒屋の定番メニュー.

TODO: 画像

\begin{tabular}[t]{rl}
\toprule
 & 難易度\\
\midrule
材料調達 & {\fontspec{Noto Sans CJK JP} ★★★☆☆ }\\
調理 & {\fontspec{Noto Sans CJK JP} ★★★☆☆ }\\
\bottomrule
\end{tabular}

\hypertarget{ux6750ux6599-2-3ux98df}{%
\subsection{材料 (2-3食)}\label{ux6750ux6599-2-3ux98df}}

\begin{itemize}
\tightlist
\item
  豚肉 300g

  \begin{itemize}
  \tightlist
  \item
    ショルダー, モモなど
  \end{itemize}
\item
  キャラウェイシード 適量
\item
  タイム 適量
\item
  玉ねぎ 1個
\item
  塩・胡椒
\item
  砂糖
\item
  ニンニク
\item
  ザワークラウト
\item
  クネドリーク @ref(\#knedliky) 参照 数切れ
\item
  ラード 適量

  \begin{itemize}
  \tightlist
  \item
    なければバターや植物油
  \end{itemize}
\item
  (オプション) 白ワインまたはワインビネガーまたは 酢 適量
\end{itemize}

\hypertarget{ux9053ux5177-20}{%
\subsection{道具}\label{ux9053ux5177-20}}

\begin{itemize}
\tightlist
\item
  できればロースト用のオーブン

  \begin{itemize}
  \tightlist
  \item
    なければ圧力鍋等で
  \end{itemize}
\end{itemize}

\hypertarget{ux4f5cux308aux65b9-25}{%
\subsection{作り方}\label{ux4f5cux308aux65b9-25}}

\begin{enumerate}
\def\labelenumi{\arabic{enumi}.}
\tightlist
\item
  ローストポークを作る

  \begin{enumerate}
  \def\labelenumii{\arabic{enumii}.}
  \tightlist
  \item
    豚肉に塩胡椒を振り, 塩胡椒とすりおろしたニンニクで下味を付ける
  \item
    みじん切りにした玉ねぎ1/2・水・タイム・キャラウェイシード・ラードを加えてオーブンでローストする

    \begin{itemize}
    \tightlist
    \item
      代替案: 圧力鍋で加熱する
    \item
      単に豚肉のステーキにする
    \end{itemize}
  \end{enumerate}
\item
  ザワークラウトのペーストを作る

  \begin{enumerate}
  \def\labelenumii{\arabic{enumii}.}
  \tightlist
  \item
    ザワークラウトを細かく切る
  \item
    鍋にラードを引き, 玉ねぎ1/2を弱火で炒める
  \item
    色がついたらザワークラウトと汁・キャラウェイシード・ベイリーフを入れて炒める
  \item
    しんなりしてきたら水を加え, 蓋をして20分煮込む
  \item
    砂糖, あるいは好みでワインビネガーや酢も使って甘酸っぱく調える
  \end{enumerate}
\item
  ローストポークの肉汁に水と小麦粉を適量補いとろみをつけてソースにする
\item
  ローストポークを切り分け,
  クネドリークとザワークラウトとソースと一緒に盛り付ける
\end{enumerate}

\hypertarget{ux53c2ux8003ux8cc7ux6599-23}{%
\subsection{参考資料}\label{ux53c2ux8003ux8cc7ux6599-23}}

\begin{itemize}
\tightlist
\item
  \citet{faktor2007Traditional} p.65
\item
  ``Recepty bez brepty'' の動画
  \url{https://www.youtube.com/watch?v=rXudmMKbInQ}
\item
  ``Czechcookbook'' の動画
  \url{https://www.youtube.com/watch?v=tkCtjcHuJ-s}
\end{itemize}

\hypertarget{ux30b9ux30daux30a4ux30f3ux98a8ux30dfux30fcux30c8ux30dcux30fcux30ebux30ebux30fcux30e9ux30fcux30c7ux30f3-ux6377-ux161panux11blskuxfd-ptuxe1ux10dekux72ec-rouladen}{%
\section{\texorpdfstring{「スペイン風」ミートボール/ルーラーデン (捷:
španělský ptáček/独:
Rouladen)\index{スペイン風ミートボール}\index{španělský ptáček|see{スペイン風ミートボール}}\index{ルーラーデン|see{スペイン風ミートボール}}\index{rouladen|see{ルーラーデン}}}{「スペイン風」ミートボール/ルーラーデン (捷: španělský ptáček/独: Rouladen)}}\label{ux30b9ux30daux30a4ux30f3ux98a8ux30dfux30fcux30c8ux30dcux30fcux30ebux30ebux30fcux30e9ux30fcux30c7ux30f3-ux6377-ux161panux11blskuxfd-ptuxe1ux10dekux72ec-rouladen}}

内部にゆで卵やソーセージ,
ピクルスを詰め込んだ大きめのミートボール.「スペイン風」とあるが,
チェコの料理である.

\begin{figure}

{\centering \includegraphics[width=1\linewidth,height=1\textheight,keepaspectratio]{img/spanelsky-ptacek/finished} 

}

\caption{「スペイン風」ミートボール, クネドリークと米を添えて}\label{fig:spanelsky-ptacek-finished}
\end{figure}

\begin{tabular}[t]{rl}
\toprule
 & 難易度\\
\midrule
材料調達 & {\fontspec{Noto Sans CJK JP} ★★★☆☆ }\\
調理 & {\fontspec{Noto Sans CJK JP} ★★★★☆ }\\
\bottomrule
\end{tabular}

\hypertarget{ux6750ux6599-15}{%
\subsection{材料}\label{ux6750ux6599-15}}

\begin{itemize}
\tightlist
\item
  牛スネ肉

  \begin{itemize}
  \tightlist
  \item
    もしくは他の柔らかい部位
  \item
    なるべく大きい塊肉がよい
  \end{itemize}
\item
  ソーセージ
\item
  ピクルス
\item
  ゆで卵
\item
  ベーコン
\item
  マスタード
\item
  玉ねぎ
\item
  ラードまたはバター
\item
  小麦粉
\end{itemize}

\hypertarget{ux9053ux5177-21}{%
\subsection{道具}\label{ux9053ux5177-21}}

\begin{itemize}
\tightlist
\item
  料理用のタコ糸

  \begin{itemize}
  \tightlist
  \item
    または爪楊枝や竹串
  \end{itemize}
\end{itemize}

\hypertarget{ux4f5cux308aux65b9-26}{%
\subsection{作り方}\label{ux4f5cux308aux65b9-26}}

\begin{enumerate}
\def\labelenumi{\arabic{enumi}.}
\tightlist
\item
  牛肉を薄く切り, 叩いて延ばす
\item
  牛肉に少量の塩胡椒とマスタードを塗って下味をつける
\item
  玉ねぎ, ゆで卵, ソーセージ, ピクルスを小さく切る
\item
  肉で上記を包み, タコ糸等で固定する
\item
  外側に再度少量のマスタードを塗る
\item
  鍋を熱してラードを溶かし,
  みじん切りにした50gのベーコンと玉ねぎ1/2を中火で炒める
\item
  火が通って色が変わってきたら, ミートボールを入れ表面を焼く
\item
  鍋に温めたブイヨンを入れる. 肉が浸かる位の量が必要
\item
  鍋に蓋をして2時間ほど煮る (圧力鍋を使って短縮しても良い)
\item
  肉を取り出し, 茹で汁は固形物を濾し取ってから小麦粉でとろみを付け,
  ソースにする
\item
  皿にミートボールとソースを盛り付ける
\item
  クネドリークまたは米と合わせて食べる
\end{enumerate}

\begin{figure}

{\centering \includegraphics[width=1\linewidth,height=1\textheight,keepaspectratio]{img/spanelsky-ptacek/tied} 

}

\caption{タコ糸, マスタードを塗る}\label{fig:spanelsky-ptacek-tied}
\end{figure}

\hypertarget{ux88dcux8db3-26}{%
\subsection{補足}\label{ux88dcux8db3-26}}

ヨーロッパのマスタードは全般的に日本の練からしよりだいぶ辛味がマイルドなので,
どちらを使うかで使用量に注意する.

柔らかい部位の肉を使う場合, 茹で時間は少ないほうが良いかもしれない.

ルーラーデンは類似のドイツ料理である.
こちらも薄く延ばした肉で具材を巻いて焼くという点が同じだが,
使われる材料は異なる.「小鳥」と訳したらチェコ人に間違えを指摘された.
ptáček は辞書には小鳥とあるが, ミートボールというニュアンスもあるので,
ここは素直に「ミートボール」と訳す.

この料理の起源は神聖ローマ皇帝ルドルフ2世時代にあると言われている.
皇母マリアはスペイン出身であり, その影響でスペイン人のシェフがおり,
その影響でスペイン風の料理として考案された.
一応スペインにも肉を巻いて焼く料理があるらしい.
チェコ料理ではあまり米が使われることがないが,
このミートボールにはしばしば米が添えられる.
地中海を想起される食材なのかもしれない.

\hypertarget{ux53c2ux8003ux8cc7ux6599-24}{%
\subsection{参考資料}\label{ux53c2ux8003ux8cc7ux6599-24}}

\begin{itemize}
\tightlist
\item
  \citet{faktor2007Traditional} p.54
\item
  ``video-recepty.com'' の動画
  \url{https://www.youtube.com/watch?v=T2hfkBYQjkc}
\item
  Wikipedia の記事
  \url{https://cs.wikipedia.org/w/index.php?title=\%C5\%A0pan\%C4\%9Blsk\%C3\%BD_pt\%C3\%A1\%C4\%8Dek\&oldid=19363197}
\end{itemize}

\hypertarget{ux30b4ux30a6ux30a9ux30f3ux30d7ux30ad-ux6ce2-goux142ux105bki}{%
\section{\texorpdfstring{ゴウォンプキ (波: Gołąbki)
\index{ゴウォンプキ}\index{gołąbki|see{ゴウォンプキ}}}{ゴウォンプキ (波: Gołąbki) }}\label{ux30b4ux30a6ux30a9ux30f3ux30d7ux30ad-ux6ce2-goux142ux105bki}}

ポーランドのロールキャベツ. 日本のと似ているが, 米を多く混ぜる,
トマトソースと一緒に煮込む, という特徴がある. ロシアの
\aruby{голубцы}{ガルプツィ} \index{голубцы|see{ゴウォンプキ}}
もほぼ同じ. \index{ガルプツィ|see{ゴウォンプキ}}

\begin{figure}

{\centering \includegraphics[width=1\linewidth,height=1\textheight,keepaspectratio]{img/golabki/finished} 

}

\caption{ゴウォンプキ}\label{fig:finished-golabki}
\end{figure}

\begin{tabular}[t]{rl}
\toprule
 & 難易度\\
\midrule
材料調達 & {\fontspec{Noto Sans CJK JP} ★★☆☆☆ }\\
調理 & {\fontspec{Noto Sans CJK JP} ★★★★☆ }\\
\bottomrule
\end{tabular}

\hypertarget{ux6750ux6599-2-3ux98dfux5206}{%
\subsection{材料 (2-3食分)}\label{ux6750ux6599-2-3ux98dfux5206}}

\begin{itemize}
\tightlist
\item
  キャベツ1玉
\item
  ひき肉 600g
\item
  玉ねぎ半分-1個
\item
  米 お椀1杯\textasciitilde 大盛り程度

  \begin{itemize}
  \tightlist
  \item
    できればインディカ米
  \end{itemize}
\item
  パン
\item
  牛乳
\item
  塩
\item
  卵
\item
  パプリカパウダー
\item
  トマトピューレ
\end{itemize}

\hypertarget{ux4f5cux308aux65b9-27}{%
\subsection{作り方}\label{ux4f5cux308aux65b9-27}}

\begin{enumerate}
\def\labelenumi{\arabic{enumi}.}
\tightlist
\item
  米を炊く
\item
  キャベツの芯をくり抜き, 葉が柔らかくなるまで丸ごと茹でる

  \begin{itemize}
  \tightlist
  \item
    別に手でちぎってもよい
  \end{itemize}
\item
  玉ねぎを炒める
\item
  玉ねぎ, 米, ひき肉, 卵, 牛乳ひたしたパンをまぜる
\item
  塩胡椒, パプリカパウダー, パセリ粉末, トマトピューレを混ぜる

  \begin{itemize}
  \tightlist
  \item
    塩胡椒の代わりにコンソメ粉末とかでもいい
  \end{itemize}
\item
  ロールキャベツと同様に肉を巻く.
\item
  鍋に詰め込んで熱湯を注ぎ, コンソメ粉末,
  オールスパイスとローレルの葉を加えて蒸す
\item
  十分に蒸したら水を切る
\item
  茹で汁は捨てずに小麦粉とトマトピューレ,
  そして刻んだディルを加えて加熱し, ソースを作る
\item
  ゴウォンプキの入った鍋にソースをかける
\end{enumerate}

\hypertarget{ux53c2ux8003ux8cc7ux6599-25}{%
\subsection{参考資料}\label{ux53c2ux8003ux8cc7ux6599-25}}

\begin{itemize}
\tightlist
\item
  ポーランド在住日本人による解説付きの動画 (日本語)
  \url{https://www.youtube.com/watch?v=GYV1bVHyglQ}
\item
  Dorotki による料理動画 (ポーランド語, 英語字幕あり)
  \url{https://www.youtube.com/watch?v=LFc-PO0fu7A}
\item
  \citet{OginoNumano2017} pp.~40-41
\end{itemize}

\hypertarget{ux30adux30e0ux30c1ux30c1ux30e0-ux97d3-uxae40uxce58uxcc1c}{%
\section{キムチチム (韓:
김치찜)}\label{ux30adux30e0ux30c1ux30c1ux30e0-ux97d3-uxae40uxce58uxcc1c}}

キムチチムの調理自体は簡単だが,
よく発酵したキムチの入手または作成が大変である.
キムチの自作方法は別項参照.

\begin{figure}

{\centering \includegraphics[width=1\linewidth,height=1\textheight,keepaspectratio]{img/kimchichim/finished} 

}

\caption{キムチチム}\label{fig:kimchichim-finished}
\end{figure}

\begin{tabular}[t]{rl}
\toprule
 & 難易度\\
\midrule
材料調達 & {\fontspec{Noto Sans CJK JP} ★★★★★ }\\
調理 & {\fontspec{Noto Sans CJK JP} ★★☆☆☆ }\\
\bottomrule
\end{tabular}

\hypertarget{ux6750ux6599-4-5ux98dfux5206}{%
\subsection{材料 (4-5食分)}\label{ux6750ux6599-4-5ux98dfux5206}}

\begin{itemize}
\tightlist
\item
  豚バラ肉ブロック 1 kg

  \begin{itemize}
  \tightlist
  \item
    肉の選び方は補足の部分参照
  \end{itemize}
\item
  白菜キムチ 1.3 kg (白菜1/4株相当)

  \begin{itemize}
  \tightlist
  \item
    発酵が進み酸味が強くなったものが良い
  \end{itemize}
\item
  ニンニク 3-4かけら
\item
  生姜
\item
  玉ねぎ
\item
  長ねぎ
\item
  唐辛子の粉 大さじ 2-3杯

  \begin{itemize}
  \tightlist
  \item
    キムチ用のさほど辛くないものを使う
  \item
    カイエンヌペッパーのような特別に辛い種類のものは適さない
  \item
    辛いのが苦手ならば一部をパプリカパウダーに置き換えてもよい
  \end{itemize}
\item
  (オプション) 生の唐辛子
\item
  (オプション) 隠し味

  \begin{itemize}
  \tightlist
  \item
    テンジャン (甜麺醤などでも良いかもしれない?)
  \item
    砂糖
  \item
    酢
  \item
    醤油またはエクチョッなどキムチに使う調味料
  \end{itemize}
\end{itemize}

\hypertarget{ux4f5cux308aux65b9-28}{%
\subsection{作り方}\label{ux4f5cux308aux65b9-28}}

\begin{enumerate}
\def\labelenumi{\arabic{enumi}.}
\tightlist
\item
  豚肉を取り分けやすい大きさに切る
\item
  野菜を適当な大きさに切る
\item
  鍋底にキムチをしく
\item
  その上に肉とネギ, 玉ねぎ, ニンニク, 生姜, 生の唐辛子をふりかける
\item
  肉が隠れる程度まで水を入れる (図\ref{fig:kimchichim-pot})
\item
  唐辛子の粉をかける

  \begin{itemize}
  \tightlist
  \item
    これは水で色や味が薄まることに対する補強である
  \end{itemize}
\item
  軽く沸き立つまで加熱する
\item
  ある程度煮立ったらオプションの調味料で味を調整する

  \begin{itemize}
  \tightlist
  \item
    酸味が足りなかったら酢を入れる
  \item
    味が薄かったら醤油とかを入れる
  \item
    発酵しすぎて匂いが強ければ砂糖を入れる
  \end{itemize}
\item
  肉が十分に柔らかくなるまで煮込み続ける

  \begin{itemize}
  \tightlist
  \item
    1時間半程度
  \item
    圧力鍋を使って短縮しても良い
  \item
    だいたい角煮と同じ要領で良い
  \end{itemize}
\end{enumerate}

\begin{figure}

{\centering \includegraphics[width=1\linewidth,height=1\textheight,keepaspectratio]{img/kimchichim/pot} 

}

\caption{煮込む直前}\label{fig:kimchichim-pot}
\end{figure}

\hypertarget{ux88dcux8db3-27}{%
\subsection{補足}\label{ux88dcux8db3-27}}

激辛料理の章に掲載しようか迷ったが,
唐辛子をよほど入れない限りそこまで辛くはならないと思われる.
多くのレシピでは豚バラ肉を指定し, 実際豚バラ肉は柔らかくするのが簡単で,
キムチチムは基本的には発酵したキムチの酸味と豚肉の脂身の組み合わせを楽しむ料理だが,
ものによっては脂身が多すぎることもある.
ペク・ジョンウォンがすすめるようにショルダーを使ったり,
脂身の少ない肉にラードを継ぎ足すのも良いかもしれない.
国内で流通しているキムチの多くは長期熟成を想定していないようなので,
衛生面の問題は自己責任で.

\hypertarget{ux53c2ux8003ux8cc7ux6599-26}{%
\subsection{参考資料}\label{ux53c2ux8003ux8cc7ux6599-26}}

\begin{itemize}
\tightlist
\item
  ペク・ジョンウォンの料理動画
  \url{https://www.youtube.com/watch?v=RVfSeUZ8XkY}
\end{itemize}

\hypertarget{ux30adux30a8ux30d5ux98a8ux30abux30c4ux30ecux30c4-ux9732-ux43aux43eux442ux43bux435ux442ux430-ux43fux43e-ux43aux438ux435ux432ux441ux43aux438ux43aux43eux442ux43bux435ux442ux430-ux434ux435-ux432ux430ux43bux44fux439}{%
\section{\texorpdfstring{キエフ風カツレツ (露: котлета
по-киевски/котлета де-валяй)
\index{キエフ風カツレツ}\index{котлета по-киевски|see{キエフ風カツレツ}}\index{котлета де-валяй|see{キエフ風カツレツ}}}{キエフ風カツレツ (露: котлета по-киевски/котлета де-валяй) }}\label{ux30adux30a8ux30d5ux98a8ux30abux30c4ux30ecux30c4-ux9732-ux43aux43eux442ux43bux435ux442ux430-ux43fux43e-ux43aux438ux435ux432ux441ux43aux438ux43aux43eux442ux43bux435ux442ux430-ux434ux435-ux432ux430ux43bux44fux439}}

古典的な котлета де-валяй はいわゆるメンチカツ的なレシピだったらしいが,
ここで紹介するのはソ連時代の一般的な 作り方で,
以前ブログで言及したものと同じ

\begin{figure}

{\centering \includegraphics[width=1\linewidth,height=1\textheight,keepaspectratio]{img/kotlet-kiev/finished} 

}

\caption{キエフ風カツレツとポテトピューレ}\label{fig:finished-kotlet-kiev}
\end{figure}

\begin{tabular}[t]{rl}
\toprule
 & 難易度\\
\midrule
材料調達 & {\fontspec{Noto Sans CJK JP} ★★★★☆ }\\
調理 & {\fontspec{Noto Sans CJK JP} ★★★★★ }\\
\bottomrule
\end{tabular}

材料 (4人前)

\begin{itemize}
\tightlist
\item
  鶏丸ごと 1羽

  \begin{itemize}
  \tightlist
  \item
    ローストチキン用に頭や内臓を除いたものが良い
  \item
    使うのは胸・ささみ・手羽元にあたる部分のみ
  \end{itemize}
\item
  バター 100 g
\item
  イタリアンパセリ
\item
  ディルの葉
\item
  レモンの絞り汁 少量
\item
  塩胡椒
\item
  小麦粉 100 ml
\item
  卵 2個
\item
  牛乳 50 ml
\item
  パン粉 100 g

  \begin{itemize}
  \tightlist
  \item
    できれば細かいもの
  \end{itemize}
\item
  揚げ物用の油
\item
  (オプション) ニンニク
\item
  (オプション) ポテトピューレまたはフライドポテト
\end{itemize}

\hypertarget{ux9053ux5177-22}{%
\subsection{道具}\label{ux9053ux5177-22}}

\begin{itemize}
\tightlist
\item
  肉を叩く棒

  \begin{itemize}
  \tightlist
  \item
    トゲのあるものは良くない, ワインボトルとかでもいい
  \end{itemize}
\item
  揚げ物ができる鍋
\end{itemize}

\hypertarget{ux4f5cux308aux65b9-29}{%
\subsection{作り方}\label{ux4f5cux308aux65b9-29}}

\begin{enumerate}
\def\labelenumi{\arabic{enumi}.}
\item
  バターを室温で柔らかくする
\item
  刻んだパセリとディル, レモンの絞り汁と小さじ半分程度の塩を混ぜる
\item
  袋かラップに包み, 棒状に形成して冷蔵庫に入れて冷やし固める.
\item
  肉を切り出す (詳細は後述)
\item
  胸肉を半分にする. なるべく均等な厚さになるように
\item
  胸肉とささみをそれぞれ, ラップを被せて叩いて薄く延ばす

  \begin{itemize}
  \tightlist
  \item
    まっすぐ打ち下ろすのではなく,
    やや中央から外側に力を掛けて均等に丸く広がるように
  \item
    ささみは多少破れても問題ない
  \end{itemize}
\item
  ささみを半分に切る
\item
  バターを取り出し, 適当な大きさに切る
\item
  バターをささみで包む
\item
  さらに胸肉で包み, ボート型になるよう押さえる
\item
  小麦粉, 卵, パン粉で衣を付ける
\item
  卵とパン粉は二度付ける
\item
  油で揚げる
\item
  好みでついでにフライドポテトも作る
\item
  さらに表面をサクサクにしたい場合, オーブンで5-10分加熱する
\item
  ポテトピューレかフライドポテトと一緒に皿に盛り付ける
\item
  鶏の腹に包丁を入れ皮を切る
\item
  手でひっぱって皮をはぎ, さらに腹の正面を走っている軟骨 (竜骨突起)
  に沿って包丁を入れる
\item
  一般的な方法ではここでささみ (小胸筋) を分離するが,
  今回は切り離さずそのまま軟骨にそって肉を切り離す
\item
  手羽側は肩の関節で繋がっているため,
  先に尻側から切って最後に肩を取り外すのがやりやすい
\item
  モモ肉と胸肉は皮でつながっているだけなので皮を剥がせば下側は難なく外れる
\item
  手羽の周りだけで繋がっている状態になったら,
  包丁で関節を探り当ててそこから切断する
\item
  無理に骨を切断しようとすると鋭利な破片ができて危険であるので,
  関節を外すようにする
\item
  あとは皮で繋がってるだけなので手で外せる
\item
  切り離した胸肉に手羽元と, ひだのようなささみが繋がっていれば成功
\item
  これを両胸で行う
\end{enumerate}

\hypertarget{ux88dcux8db3-28}{%
\subsection{補足}\label{ux88dcux8db3-28}}

中に入っているのがバターなので, 時間が経つと漏れ出してしまう.
揚げたてを食べるのが良い.

余った部分はシュクメルリとか口水鶏とか辣子鶏とかに使うとよいだろう.

\hypertarget{ux53c2ux8003ux8cc7ux6599-27}{%
\subsection{参考資料}\label{ux53c2ux8003ux8cc7ux6599-27}}

\begin{itemize}
\tightlist
\item
  Коллекция Рецептов によるソ連時代の作り方 (ロシア語, 英語字幕あり)
  \url{https://www.youtube.com/watch?v=C-nt0yqz-0g}
\item
  上記を参考にしたブログ記事
  \url{https://under-identified.hatenablog.com/entry/2020/12/30/233933}
\item
  より簡単な作り方の紹介 (日本語)
  \url{https://www.youtube.com/watch?v=MKhA7T9nzbE}
\item
  もっと簡単な作り方の紹介
  \url{https://note.com/dogirls_ua/n/n72220579a52b}
\end{itemize}

\hypertarget{ux30c9ux30ebux30deux306eux4e09ux59c9ux59b9-ux4e9cux585e-uxfcuxe7-bacux131-dolmasux131}{%
\section{ドルマの三姉妹 (亜塞: Üç-Bacı
Dolması)}\label{ux30c9ux30ebux30deux306eux4e09ux59c9ux59b9-ux4e9cux585e-uxfcuxe7-bacux131-dolmasux131}}

\begin{figure}

{\centering \includegraphics[width=1\linewidth,height=1\textheight,keepaspectratio]{img/dolma/dolma-finished} 

}

\caption{ドルマの三姉妹}\label{fig:dolma-finished}
\end{figure}

アゼルバイジャン料理のユーチューブチャンネルで見つけた.
トルコやイランにもあるかもしれない.

\begin{tabular}[t]{rl}
\toprule
 & 難易度\\
\midrule
材料調達 & {\fontspec{Noto Sans CJK JP} ★★★★☆ }\\
調理 & {\fontspec{Noto Sans CJK JP} ★★★☆☆ }\\
\bottomrule
\end{tabular}

\hypertarget{ux6750ux6599-3ux98dfux5206}{%
\subsection{材料 (3食分)}\label{ux6750ux6599-3ux98dfux5206}}

\begin{itemize}
\tightlist
\item
  挽き肉 400 - 500 g

  \begin{itemize}
  \tightlist
  \item
    羊肉または鶏肉, あるいは牛肉
  \end{itemize}
\item
  ピーマン 適量
\item
  トマト 3個
\item
  茄子 3-4本
\item
  植物油
\item
  玉ねぎ 1個
\item
  塩 小さじ2杯
\item
  胡椒 少々
\item
  パプリカパウダー 小さじ 1杯
\item
  ターメリック 小さじ 1杯
\item
  ニンニク 2-3欠片
\item
  コリアンダーの葉 1束
\item
  水 50 cc
\item
  (オプション) マッシュルーム
\end{itemize}

\hypertarget{ux9053ux5177-23}{%
\subsection{道具}\label{ux9053ux5177-23}}

\begin{itemize}
\tightlist
\item
  フライパンと蓋

  \begin{itemize}
  \tightlist
  \item
    野菜の大きさによっては底の深い鍋を使うべき
  \end{itemize}
\end{itemize}

\hypertarget{ux4f5cux308aux65b9-30}{%
\subsection{作り方}\label{ux4f5cux308aux65b9-30}}

\begin{enumerate}
\def\labelenumi{\arabic{enumi}.}
\tightlist
\item
  茄子はへたを切り取り, 切れ目を入れて塩を刷り込む
\item
  ピーマンは上のほうに切り込みを入れて芯と種を取りだす
\item
  トマトも同様にする. おそらくスプーンを使うと簡単

  \begin{itemize}
  \tightlist
  \item
    中身は捨てずに後で使う
  \end{itemize}
\item
  茄子を水でゆすいで塩分をとる
\item
  フライパンに油をひき, 茄子を柔らかくなるまで軽く素揚げする

  \begin{itemize}
  \tightlist
  \item
    人によってはピーマンとトマトも同様にすることもある
  \item
    茹でるケースもある
  \end{itemize}
\item
  胡椒, パプリカパウダー, ターメリック, ニンニクを混ぜてよくすりつぶす

  \begin{itemize}
  \tightlist
  \item
    なければカレー粉とかフメリ・スネリとかでもいいんじゃない
  \end{itemize}
\item
  フライパンで玉ねぎを炒める
\item
  挽き肉と上記のスパイスを混ぜる

  \begin{itemize}
  \tightlist
  \item
    コリアンダーの葉を混ぜても良い
  \end{itemize}
\item
  さらにトマトの中身も合わせて炒める
\item
  炒めた肉を野菜に詰める
\item
  フライパンに並べ, 少量の水を加え,
  蓋をして野菜が柔らかくなるまで蒸し焼きにする
\item
  (オプション) 炒めたマッシュルームや刻んだコリアンダーの葉を添える
\end{enumerate}

\hypertarget{ux88dcux8db3-29}{%
\subsection{補足}\label{ux88dcux8db3-29}}

おそらく最も難しいのは羊の挽き肉を入手または作成することだろう.
それ以外は原始的な調理器具でも対処できる. 多くの場合,
トマトとピーマンを事前に加熱する必要はない.
特にトマトはもろく崩れやすくなるため注意 (図\ref{fig:dolma-failed}).
荻野のドルマのレシピで紹介されているように, 米を入れたり,
茄子も中をくり抜いたりする動画も見られた.

\begin{figure}

{\centering \includegraphics[width=1\linewidth,height=1\textheight,keepaspectratio]{img/dolma/dolma-failed} 

}

\caption{加熱のし過ぎでトマトが崩れる}\label{fig:dolma-failed}
\end{figure}

\hypertarget{ux53c2ux8003ux6587ux732e}{%
\subsection{参考文献}\label{ux53c2ux8003ux6587ux732e}}

\begin{itemize}
\tightlist
\item
  (たぶんアゼリー語, ロシア語, 英語字幕あり)
  \url{https://www.youtube.com/watch?v=GUfGmjOgCZE}
\item
  (たぶんアゼリー語) \url{https://www.youtube.com/watch?v=bohT4pGV7rc}
\item
  (たぶんアゼリー語) \url{https://www.youtube.com/watch?v=eNj9yJ0_1NI}
\item
  荻野恭子『ピーマンとなすの肉詰め煮~ドルマ~』キューピー3分クッキング
  \url{https://www.ntv.co.jp/3min/recipe/20070901/}
\end{itemize}

\hypertarget{ux8c5aux8db3ux306eux30a2ux30b9ux30d4ux30c3ux30af-ux6377-huspenina}{%
\section{\texorpdfstring{豚足のアスピック (捷: Huspenina)
\index{アスピック}\index{huspenina|see{アスピック}}\index{にこごり@煮凝り|see{アスピック}}}{豚足のアスピック (捷: Huspenina) }}\label{ux8c5aux8db3ux306eux30a2ux30b9ux30d4ux30c3ux30af-ux6377-huspenina}}

要するに煮凝り (図\ref{fig:huspenina-finished}).

\begin{figure}

{\centering \includegraphics[width=1\linewidth,height=1\textheight,keepaspectratio]{img/huspenina/finished} 

}

\caption{フスペニナ}\label{fig:huspenina-finished}
\end{figure}

\begin{tabular}[t]{rl}
\toprule
 & 難易度\\
\midrule
材料調達 & {\fontspec{Noto Sans CJK JP} ★★★★☆ }\\
調理 & {\fontspec{Noto Sans CJK JP} ★★☆☆☆ }\\
\bottomrule
\end{tabular}

\hypertarget{ux6750ux6599-ux5927ux52e2ux3067ux98dfux3079ux3089ux308cux308b}{%
\subsection{材料
(大勢で食べられる)}\label{ux6750ux6599-ux5927ux52e2ux3067ux98dfux3079ux3089ux308cux308b}}

\begin{itemize}
\tightlist
\item
  豚足 4本

  \begin{itemize}
  \tightlist
  \item
    血抜きや体毛の下処理はしてあるものとする
  \end{itemize}
\item
  水 2l 程度
\item
  玉ねぎ
\item
  人参
\item
  パセリ根
\item
  セロリ

  \begin{itemize}
  \tightlist
  \item
    あればセルリアックを使う
  \end{itemize}
\item
  ニンニク
\item
  ゆで卵
\item
  塩・胡椒
\item
  オールスパイス
\item
  ベイリーフ
\end{itemize}

\hypertarget{ux9053ux5177-24}{%
\subsection{道具}\label{ux9053ux5177-24}}

\begin{itemize}
\tightlist
\item
  Huspenina を固める型

  \begin{itemize}
  \tightlist
  \item
    ババロア用の型など, 耐熱性があり底の深い容器ならなんでもよい
  \end{itemize}
\item
  大きめの鍋, 2つあると良い
\end{itemize}

\hypertarget{ux4f5cux308aux65b9-31}{%
\subsection{作り方}\label{ux4f5cux308aux65b9-31}}

\begin{enumerate}
\def\labelenumi{\arabic{enumi}.}
\tightlist
\item
  豚足をよく水で洗い汚れを落とす
\item
  鍋に豚足を入れ, 完全に浸かるまで十分に水を注ぐ
\item
  しばらく下茹でしアク取りをする
\item
  アクが取れたら玉ねぎ, 塩胡椒, オールスパイス, ベイリーフを入れる
\item
  弱火で1時間45分茹で続ける
\item
  皮を剥いた人参とパセリ根とセロリを入れ, 15分茹でる

  \begin{itemize}
  \tightlist
  \item
    小さく切っても良い
  \end{itemize}
\item
  豚足と人参とパセリ根を取りだす
\item
  粗熱が取れてから, 豚足の骨から皮と肉を外す

  \begin{itemize}
  \tightlist
  \item
    ひづめ周辺の小さい骨を残さないように注意
  \end{itemize}
\item
  皮と肉をミンチにする. 柔らかいので包丁でも可
\item
  取り出した根菜を小さく角切りにする
\item
  肉と根菜を新しい鍋に入れ, さらに茹で汁を濾したものを注ぐ
\item
  軽く湯だつまで加熱する
\item
  型に切ったゆで卵と上記を注ぎ, 冷蔵庫で一晩冷やす
\end{enumerate}

\hypertarget{ux88dcux8db3-30}{%
\subsection{補足}\label{ux88dcux8db3-30}}

元のレシピでは豚の頭も使用しているが,
入手が難しかったため使用しなかった.
赤身肉の少なさが気になるなら適当に肉を継ぎ足してもよいだろう.

豚足4本でできる Huspenina はかなり多く,
またこの水の量ではだいぶ濃い色になる. もっと水が多くても問題ないだろう.

おそらくは, 豚を解体し, 肉は塩漬けに,
血と腸はソーセージにして残った頭と足も食べるために考案された料理だろう.
このような豚の屠殺を zabíjačka といい,
現在でもチェコの田舎では中世の農村さながらの豚の解体祭りをするらしい.
煮凝りにせず, 単に柔らかくなるまで塩水で煮ただけで食べる ovar
という料理もある\citep{faktor2007Traditional}.

周辺各国にも類似した料理が存在する. ロシア (Холодец, студень),
ベラルーシ, ウクライナ (холодець),
ポーランドなどでも類似した料理が存在する. これらは豚の頭や足を使うこと,
卵や野菜を入れる点で共通している.

\hypertarget{ux53c2ux8003ux8cc7ux6599-28}{%
\subsection{参考資料}\label{ux53c2ux8003ux8cc7ux6599-28}}

\begin{itemize}
\tightlist
\item
  Random Innkeeper の動画
  \url{https://www.youtube.com/watch?v=7Eegww-atzU}
\item
  \citet{faktor2007Traditional}, p.~58
\end{itemize}

\hypertarget{ux30cfux30ebux30c1ux30e7ux30fc-ux30b4ux30fcux30df-ux8349-ux10eeux10d0ux10e0ux10e9ux10dd-ux9732-ux445ux430ux440ux447ux43e}{%
\section{ハルチョー(+ ゴーミ) (草: ხარჩო, 露:
харчо)}\label{ux30cfux30ebux30c1ux30e7ux30fc-ux30b4ux30fcux30df-ux8349-ux10eeux10d0ux10e0ux10e9ux10dd-ux9732-ux445ux430ux440ux447ux43e}}

ジョージアのスープ. どちらかというと材料調達の難易度が高い
(図\ref{fig:kharcho-finished}). \citet{boumei}
で紹介されているレシピは豊富な種類の果物やスパイスを使用しているが,
ここではもう少し簡単に,
かつジョージアでよく見られる鶏肉を使用したレシピを提案する

\begin{figure}

{\centering \includegraphics[width=1\linewidth,height=1\textheight,keepaspectratio]{img/kharcho/finished} 

}

\caption{羊肉のハルチョーとゴーミ}\label{fig:kharcho-finished}
\end{figure}

\begin{tabular}[t]{rl}
\toprule
 & 難易度\\
\midrule
材料調達 & {\fontspec{Noto Sans CJK JP} ★★★★★ }\\
調理 & {\fontspec{Noto Sans CJK JP} ★★★☆☆ }\\
\bottomrule
\end{tabular}

\hypertarget{ux6750ux6599-16}{%
\subsection{材料}\label{ux6750ux6599-16}}

\begin{itemize}
\tightlist
\item
  鶏肉 600 g
\item
  セロリの茎葉 1本
\item
  月桂樹の葉 1枚
\item
  クルミ 200g
\item
  ニンニク 3欠片
\item
  コリアンダーの葉
\item
  トマトまたはトマトピューレ
\item
  アンズまたはザクロ, なければ柑橘類など酸味のある果物
\item
  フメリ・スネリ (後述)
\end{itemize}

フメリ・スネリはロシア雑貨を扱っている店などで買えるが,
ジョージアのものとは違うかもしれない.
今回は次のような簡易バージョンを提案する.

\begin{itemize}
\tightlist
\item
  クレイジーソルト
\item
  バジルの葉
\item
  コリアンダーの葉
\item
  サフラン
\item
  マジョラム
\item
  パプリカ
\item
  胡椒
\item
  鷹の爪
\item
  カイエンヌパウダー
\end{itemize}

\hypertarget{ux4f5cux308aux65b9-32}{%
\subsection{作り方}\label{ux4f5cux308aux65b9-32}}

\begin{enumerate}
\def\labelenumi{\arabic{enumi}.}
\tightlist
\item
  キンザ, ディル, ニンニク, セロリ, ネギ を細かく刻む
\item
  肉を茹でる
\item
  アクをあらかた除いたら, セロリとベイリーフを入れる

  \begin{enumerate}
  \def\labelenumii{\arabic{enumii}.}
  \tightlist
  \item
    ニンジンや玉ねぎを入れても良い
  \end{enumerate}
\item
  そのまま1時間ほど弱火で茹でてブイヨンを作る
\item
  固形物を取り除く. 肉は食べやすい大きさに切って残す
\item
  ソース(アディカ)を作る

  \begin{enumerate}
  \def\labelenumii{\arabic{enumii}.}
  \tightlist
  \item
    クルミを細かく砕く
  \item
    フメリ・スネリまたは上記のハーブ・スパイスを細かく刻んで入れる
  \item
    ペースト状になる程度の量に調整してブイヨンの一部を入れる.
  \end{enumerate}
\item
  鍋にバターを溶かし, 肉と玉ねぎを炒める
\item
  タマネギをみじん切りにして油で炒める
\item
  トマトピューレかホールトマトを混ぜてさらに炒める
\item
  アディカを入れてしばらく煮込む
\item
  最後に果汁をふりかける
\end{enumerate}

\hypertarget{ux88dcux8db3-31}{%
\subsection{補足}\label{ux88dcux8db3-31}}

ロシアでも広く知られているが, ジョージア本来のもととは多少異なるようだ.
ロシアでは牛肉, ジョージアでは鶏肉を使うことが多い.
ロシアではあまりクルミをあまり入れず赤っぽくなるが,
ジョージアでは砕いたクルミを大量に入れて茶色っぽくなる. また,
ロシアのスープの常として, ディルが入れられることもある.
しかしながらジョージア国内でも地域によっても差があるようにみえる.

さらに大きな違いとして, ジョージアでは \aruby{ღომი}{ゴーミ}
というトウモロコシ粉のポリッジにチーズを乗せた料理と一緒に食べることが多い.

図\ref{fig:kharcho-finished}は粟で同様のものが作れないか試した際のものである.
粟に対して3倍程度の水を加え,
1時間ほど浸してから弱火で加熱しながらかき混ぜる. 途中で小麦粉を足す.
最後にスライスしたチーズを乗せる.

沸騰して跳ねることがあるため暑い日でも服を着て調理すべきである.

\hypertarget{ux53c2ux8003ux8cc7ux6599-29}{%
\subsection{参考資料}\label{ux53c2ux8003ux8cc7ux6599-29}}

ジョージア語はほとんどわからないので見様見真似である

\begin{itemize}
\tightlist
\item
  Apolines cuisine の動画, くるみのペーストに近い見た目
  (ジョージア語音声) \url{https://www.youtube.com/watch?v=DeLvJ7V-30k}
\item
  Arkadi Kalantarov の動画 (ジョージア語字幕)
  \url{https://www.youtube.com/watch?v=kcV9GgPqcoU}
\item
  1TV (ジョージアのテレビ局) の紹介動画, 後半は「ハルチョー風パスタ」?
  (ジョージア語音声) \url{https://www.youtube.com/watch?v=kcV9GgPqcoU}
\item
  Спасибо Шеф の動画 (ロシア語)
  \url{https://www.youtube.com/watch?v=mRybqZc_gZk}
\item
  \citet{boumei} 10章
\item
  CNN の記事
  \url{https://edition.cnn.com/travel/article/georgia-best-food-tbilisi/index.html}
\item
  「ゴーミ」の料理動画 (英語)
  \url{https://www.youtube.com/watch?v=76jFKW23X_Y}
\end{itemize}

\hypertarget{summer}{%
\chapter{夏の料理編}\label{summer}}

暑い季節に食べやすいものを紹介する

\hypertarget{ux9b8eux306eux5869ux713cux304d-ux65e5}{%
\section{\texorpdfstring{鮎の塩焼き (日)
\index{鮎の塩焼き}}{鮎の塩焼き (日) }}\label{ux9b8eux306eux5869ux713cux304d-ux65e5}}

鮎といえば6月だがいまは天然物が貴重でスーパーで売ってるのはだいたい養殖物なのであまり関係ない気もする.

\begin{figure}

{\centering \includegraphics[width=1\linewidth,height=1\textheight,keepaspectratio]{img/ayu/finished} 

}

\caption{鮎の塩焼き}\label{fig:ayu-finished}
\end{figure}

\begin{tabular}[t]{rl}
\toprule
 & 難易度\\
\midrule
材料調達 & {\fontspec{Noto Sans CJK JP} ★★★☆☆ }\\
調理 & {\fontspec{Noto Sans CJK JP} ★★★☆☆ }\\
\bottomrule
\end{tabular}

\hypertarget{ux6750ux6599-17}{%
\subsection{材料}\label{ux6750ux6599-17}}

\begin{itemize}
\tightlist
\item
  鮎 1尾
\item
  塩
\item
  割り箸 1本

  \begin{itemize}
  \tightlist
  \item
    竹串があれば使って良い
  \end{itemize}
\item
  (オプション) すだち
\end{itemize}

\hypertarget{ux9053ux5177-25}{%
\subsection{道具}\label{ux9053ux5177-25}}

魚焼きグリル

\hypertarget{ux4f5cux308aux65b9-33}{%
\subsection{作り方}\label{ux4f5cux308aux65b9-33}}

\begin{enumerate}
\def\labelenumi{\arabic{enumi}.}
\tightlist
\item
  尻ビレ付近の肛門の手前を指で押し, 残ったフンを押し出す
\item
  鮎を冷水で洗い, 表面のぬめりを取る
\item
  表面の水を良く拭き取る
\item
  塩を振る
\item
  ヒレには多めに塩をなすりつける
\item
  串を打つ

  \begin{itemize}
  \tightlist
  \item
    割り箸の場合はカッターナイフや包丁で削って尖らせる
  \item
    のぼり串/踊り串に挑戦するのも一興
  \end{itemize}
\item
  グリルで焼く
\end{enumerate}

\hypertarget{ux88dcux8db3-32}{%
\subsection{補足}\label{ux88dcux8db3-32}}

尾ヒレを取り,
全身をかるく箸で押してほぐしてからえらの手前でちぎると背骨が抜けやすい.

\hypertarget{ukha}{%
\section{\texorpdfstring{ウハー (露:
Уха)\index{ウハー}\index{уха|see{ウハー}}}{ウハー (露: Уха)}}\label{ukha}}

\begin{quote}
言うまでもなく, ウハーとは儀式である.

\begin{flushright}
--- A. ゲニス \& P. ワイリ

\end{flushright}
\end{quote}

川魚を使ったスープ. \citet{boumei} によると,
川で釣った魚をその場で食べるためのレシピとのこと.
基本的に彼らの記述に則したレシピ (図\ref{fig:ykha-finished}).

\begin{figure}

{\centering \includegraphics[width=1\linewidth,height=1\textheight,keepaspectratio]{img/ukha/finished} 

}

\caption{鮭のウハー}\label{fig:ykha-finished}
\end{figure}

\begin{tabular}[t]{rl}
\toprule
 & 難易度\\
\midrule
材料調達 & {\fontspec{Noto Sans CJK JP} ★★★☆☆ }\\
調理 & {\fontspec{Noto Sans CJK JP} ★★☆☆☆ }\\
\bottomrule
\end{tabular}

\hypertarget{ux6750ux6599-18}{%
\subsection{材料}\label{ux6750ux6599-18}}

\begin{itemize}
\tightlist
\item
  川魚

  \begin{itemize}
  \tightlist
  \item
    鮭, タラ (日本での入手方法は不明), ニジマス, チョウザメなど
  \item
    チョウザメのウハーが最も上等とされる
  \end{itemize}
\item
  人参
\item
  ジャガイモ
\item
  塩胡椒
\item
  パセリの葉
\item
  ローレルの葉
\item
  万能ねぎ
\item
  (オプション) 煮干し
\item
  (オプション) サフラン
\item
  (オプション) ウォッカ

  \begin{itemize}
  \tightlist
  \item
    隠し味に
  \end{itemize}
\item
  (オプション) ディルの葉
\end{itemize}

\hypertarget{ux4f5cux308aux65b9-34}{%
\subsection{作り方}\label{ux4f5cux308aux65b9-34}}

\begin{enumerate}
\def\labelenumi{\arabic{enumi}.}
\tightlist
\item
  湯を沸かし, 煮干しで出汁をとる (なければ湯を沸かすだけ)
\item
  出汁から煮干しの残りカスを濾し取り, 捨てる
\item
  出汁を弱火にかけ, みじん切りにした玉ねぎ, セロリ, パセリ, ローレル,
  塩胡椒を加える
\item
  20-30分ほど煮て香りを引き出す
\item
  細切れにした人参とジャガイモを入れて5-6分加熱する
\item
  魚の切り身を入れ, 数分煮る. あればウォッカを少し入れる
\item
  火を止める
\item
  カップにサフランを1つまみ入れ, 煮えた出汁を1すくいかけ,
  これを鍋に入れる.
\item
  刻んだディルの葉などを添えて食べる

  \begin{itemize}
  \tightlist
  \item
    飲むためのウォッカも用意する
  \end{itemize}
\end{enumerate}

\hypertarget{ux88dcux8db3-33}{%
\subsection{補足}\label{ux88dcux8db3-33}}

玉ねぎを入れる人も多いが, \citet{boumei}
ウハーは透き通ったスープであることが大切なので,
スープが濁る玉ねぎは使うなと主張する. また, タラゴンやバジルなど,
ハーブは種類が多ければ多いほどよいとも書いている.
私は試したことがないが,
ヴィクトリアショップではロシア製のウハー用ミックススパイスを扱っている\footnote{\url{https://victoriashop.jp/collections/\%E3\%82\%B9\%E3\%83\%91\%E3\%82\%A4\%E3\%82\%B9\%E9\%A1\%9E/products/\%E3\%82\%A6\%E3\%83\%8F\%E3\%83\%BC-\%E3\%83\%AD\%E3\%82\%B7\%E3\%82\%A2\%E9\%A2\%A8\%E9\%AD\%9A\%E3\%81\%AE\%E3\%82\%B9\%E3\%83\%BC\%E3\%83\%97-\%E3\%83\%9F\%E3\%83\%83\%E3\%82\%AF\%E3\%82\%B9\%E3\%82\%B9\%E3\%83\%91\%E3\%82\%A4\%E3\%82\%B9}}.

\hypertarget{ux53c2ux8003ux8cc7ux6599-30}{%
\subsection{参考資料}\label{ux53c2ux8003ux8cc7ux6599-30}}

\begin{itemize}
\tightlist
\item
  \citet{boumei} Ch. 20
\end{itemize}

\hypertarget{ux30b7ux30e3ux30ebux30c6ux30a3ux30d0ux30ebux30b7ux30e5ux30c1ux30e3ux30a4-ux7acb-ux161altibarux161ux10diai}{%
\section{\texorpdfstring{シャルティバルシュチャイ (立: Šaltibarščiai)
\index{シャルティバルシュチャイ}\index{Šaltibarščiai|see{シャルティバルシュチャイ}}}{シャルティバルシュチャイ (立: Šaltibarščiai) }}\label{ux30b7ux30e3ux30ebux30c6ux30a3ux30d0ux30ebux30b7ux30e5ux30c1ux30e3ux30a4-ux7acb-ux161altibarux161ux10diai}}

ボルシチに似ているが肉を使わない,
冷たい料理.「液体のサラダ」とでもいうべき料理(図\ref{fig:saltibarsciai-finished})

\begin{figure}

{\centering \includegraphics[width=1\linewidth,height=1\textheight,keepaspectratio]{img/saltibarsciai/finished} 

}

\caption{ふかし芋入りのシャルティバルシュチャイ}\label{fig:saltibarsciai-finished}
\end{figure}

\begin{tabular}[t]{rl}
\toprule
 & 難易度\\
\midrule
材料調達 & {\fontspec{Noto Sans CJK JP} ★★★★☆ }\\
調理 & {\fontspec{Noto Sans CJK JP} ★★☆☆☆ }\\
\bottomrule
\end{tabular}

\hypertarget{ux6750ux6599-3-4ux98df}{%
\subsection{材料 (3-4食)}\label{ux6750ux6599-3-4ux98df}}

\begin{itemize}
\tightlist
\item
  ビーツ 1 個
\item
  ケフィア 400 g

  \begin{itemize}
  \tightlist
  \item
    ヨーグルトやサワークリームも可
  \end{itemize}
\item
  卵 1個
\item
  塩胡椒 少々
\item
  キュウリ 1本
\item
  ディルの葉
\item
  万能ねぎ
\item
  (オプション) ジャガイモ 1個
\end{itemize}

\hypertarget{ux4f5cux308aux65b9-35}{%
\subsection{作り方}\label{ux4f5cux308aux65b9-35}}

\begin{enumerate}
\def\labelenumi{\arabic{enumi}.}
\tightlist
\item
  ビーツをすりおろす
\item
  キュウリを細かく切る
\item
  ビーツ, キュウリ, ケフィアを良く混ぜ, 塩で味を調える
\item
  ゆで卵, ふかし芋, 刻んだディルの葉や万能ねぎを添える
\end{enumerate}

\hypertarget{ux88dcux8db3-34}{%
\subsection{補足}\label{ux88dcux8db3-34}}

リトアニア語で「冷たいボルシチ」という意味だが,
ロシアではあまり食べられていないようだ\footnote{ただしロシアにもボトヴィニヤやオクローシカといった多様な冷製スープのレシピが存在するので,
  そのうちどれかと関係があるかもしれない.}. ポーランドの
chłodnik\index{chłodni|see{シャルティバルシュチャイ}}, ベラルーシの
халаднік\index{халаднік|see{シャルティバルシュチャイ}}, ウクライナの
холодник \index{холодник|see{シャルティバルシュチャイ}}もほぼ同じ料理.

\hypertarget{ux53c2ux8003ux8cc7ux6599-31}{%
\subsection{参考資料}\label{ux53c2ux8003ux8cc7ux6599-31}}

\begin{itemize}
\tightlist
\item
  \citet{OginoNumano2017} p.14
\item
  (たぶんリトアニア語, 英語字幕)
  \url{https://www.youtube.com/watch?v=sL8Q05qt56c}
\end{itemize}

\hypertarget{ux30cbux30b8ux30deux30b9ux306eux30d0ux30bfux30fcux30b0ux30eaux30eb-ux6377-peux10denuxfd-pstruh-na-muxe1sle}{%
\section{\texorpdfstring{ニジマスのバターグリル (捷, pečený pstruh na
másle)
\index{pečený pstruh na másle|see{ニジマスのバターグリル}}}{ニジマスのバターグリル (捷, pečený pstruh na másle) }}\label{ux30cbux30b8ux30deux30b9ux306eux30d0ux30bfux30fcux30b0ux30eaux30eb-ux6377-peux10denuxfd-pstruh-na-muxe1sle}}

図\ref{fig:trout-finished} のようにレモンの薄切りを乗せることが多い.

\begin{figure}

{\centering \includegraphics[width=1\linewidth,height=1\textheight,keepaspectratio]{img/trout/finished} 

}

\caption{ザワークラウトは不要だった}\label{fig:trout-finished}
\end{figure}

\begin{tabular}[t]{rl}
\toprule
 & 難易度\\
\midrule
材料調達 & {\fontspec{Noto Sans CJK JP} ★★☆☆☆ }\\
調理 & {\fontspec{Noto Sans CJK JP} ★★☆☆☆ }\\
\bottomrule
\end{tabular}

\hypertarget{ux6750ux6599-19}{%
\subsection{材料}\label{ux6750ux6599-19}}

\begin{itemize}
\tightlist
\item
  ニジマス 2 匹
\item
  レモン
\item
  バター
\item
  キャラウェイシード
\item
  パセリ
\end{itemize}

\hypertarget{ux9053ux5177-26}{%
\subsection{道具}\label{ux9053ux5177-26}}

オーブンがあると良いが, なければフライパンや魚焼きグリルで代用する

\hypertarget{ux4f5cux308aux65b9-36}{%
\subsection{作り方}\label{ux4f5cux308aux65b9-36}}

\begin{enumerate}
\def\labelenumi{\arabic{enumi}.}
\tightlist
\item
  ニジマスの腹を開き, ワタを取り, 腹の中を水で洗う
\item
  水気を拭き取る
\item
  酢を軽くふりかける
\item
  小さく切ったバターを腹に詰める
\item
  軽く塩をして10分寝かせる
\item
  キャラウェイシードもふりかける
\item
  表面に小麦粉をまぶす
\item
  オーブンで焼く
\item
  輪切りにしたレモンやパセリを乗せて盛り付ける
\end{enumerate}

\hypertarget{ux88dcux8db3-35}{%
\subsection{補足}\label{ux88dcux8db3-35}}

茹でたジャガイモとともに食べることが多い.

マヨネーズやディジョンマスタードをかけることも多い

ニンニクの欠片を入れてもよいかもしれない.

\hypertarget{ux53c2ux8003ux8cc7ux6599-32}{%
\subsection{参考資料}\label{ux53c2ux8003ux8cc7ux6599-32}}

\begin{itemize}
\tightlist
\item
  \citet{faktor2007Traditional} p.24
\item
  \url{https://www.youtube.com/watch?v=RNjsLl5w5-E}
\end{itemize}

\hypertarget{winter}{%
\chapter{冬の料理編}\label{winter}}

基本的に熱いうちに食べるとうまいスープや煮込み料理.

\hypertarget{gulas}{%
\section{グラーシュ (捷: Guláš)}\label{gulas}}

もとはハンガリー料理のグヤーシュ guyás だが,
これが元になったチェコ料理のほうが本家より有名になっている.
現地でも様々な作り方があるため,
このレシピが唯一の正統なものではないことに注意する
(どのラーメンが正しいか論争するようなもの).

\begin{figure}

{\centering \includegraphics[width=1\linewidth,height=1\textheight,keepaspectratio]{img/gulas/finished} 

}

\caption{牛肉のグラーシュ}\label{fig:finished-gulas}
\end{figure}

\begin{tabular}[t]{rl}
\toprule
 & 難易度\\
\midrule
材料調達 & {\fontspec{Noto Sans CJK JP} ★★★☆☆ }\\
調理 & {\fontspec{Noto Sans CJK JP} ★★★☆☆ }\\
\bottomrule
\end{tabular}

\hypertarget{ux6750ux6599-20}{%
\subsection{材料}\label{ux6750ux6599-20}}

\begin{itemize}
\tightlist
\item
  牛肉 200 g

  \begin{itemize}
  \tightlist
  \item
    スネ肉など煮込みに適した部位がよい
  \end{itemize}
\item
  ラードまたはバター
\item
  パプリカパウダー 大さじ2
\item
  玉ねぎ 大1個
\item
  トマトピューレ
\item
  塩胡椒 適量
\item
  ニンニク 数欠片
\item
  キャラウェイシード 小さじ1
\item
  マジョラム 1つまみ
\item
  小麦粉
\end{itemize}

\hypertarget{ux4f5cux308aux65b9-37}{%
\subsection{作り方}\label{ux4f5cux308aux65b9-37}}

\begin{enumerate}
\def\labelenumi{\arabic{enumi}.}
\tightlist
\item
  鍋を熱してラードを溶かす
\item
  玉ねぎをみじん切りにして焦がさないよう弱火で炒める
\item
  色がついてきたらパプリカパウダーを入れてよくかき混ぜる
\item
  肉を入れて炒める
\item
  トマトピューレを入れる
\item
  水を加える
\item
  塩胡椒で味付けする
\item
  小麦粉でとろみをつける
\item
  みじん切りまたはすりおろしたニンニクとキャラウェイシードを入れる
\item
  パセリや薄く輪切りにした玉ねぎを添えても良い
\end{enumerate}

\hypertarget{ux88dcux8db3-36}{%
\subsection{補足}\label{ux88dcux8db3-36}}

定番の隠し味として, ビールを入れる,
あるいは肉をビールにつけて柔らかくするというものがある.

広く普及している料理なので作り方のバリエーションが多い.

\hypertarget{ux53c2ux8003ux8cc7ux6599-33}{%
\subsection{参考資料}\label{ux53c2ux8003ux8cc7ux6599-33}}

\begin{itemize}
\tightlist
\item
  \citet{faktor2007Traditional} p.50
\item
  \url{https://www.toprecepty.cz/recept/4007-klasicky-hospodsky-gulas-vidensky/}
\end{itemize}

\hypertarget{ux30dcux30ebux30b7ux30c1-ux5b87ux9732-ux431ux43eux440ux449}{%
\section{\texorpdfstring{ボルシチ (宇/露:
Борщ)\index{ボルシチ}\index{Борщ|see{ボルシチ}}}{ボルシチ (宇/露: Борщ)}}\label{ux30dcux30ebux30b7ux30c1-ux5b87ux9732-ux431ux43eux440ux449}}

\begin{quote}
「ボルジチ」じゃあない「ボルシチ」だ. 何度も言ってるだろう\ldots{}

\begin{flushright}
--- アルカージィ\&ボリス=ストルガツキィ

\end{flushright}
\end{quote}

主にウクライナ風のレシピを紹介するが, ロシア風に作るヒントも残しておく.
広く普及してるのでいろいろなバリエーションはあるが, 典型的な特徴は

\begin{enumerate}
\def\labelenumi{\arabic{enumi}.}
\item
  ウクライナでは豚肉を使うことが多い
\item
  ロシアでは牛肉を使うことが多い
\end{enumerate}

である. 脂っこさと煮込んだ野菜の甘味が重要なのはおそらく両国共通.
ここではクロポテンコのレシピに近いものを紹介する
(図\ref{fig:finished-borscht}).

\begin{figure}

{\centering \includegraphics[width=1\linewidth,height=1\textheight,keepaspectratio]{img/borscht/finished} 

}

\caption{スペアリブのボルシチ}\label{fig:finished-borscht}
\end{figure}

\begin{tabular}[t]{rl}
\toprule
 & 難易度\\
\midrule
材料調達 & {\fontspec{Noto Sans CJK JP} ★★★★☆ }\\
調理 & {\fontspec{Noto Sans CJK JP} ★★★☆☆ }\\
\bottomrule
\end{tabular}

\hypertarget{ux6750ux6599-5-6ux98dfux5206}{%
\subsection{材料 (5-6食分)}\label{ux6750ux6599-5-6ux98dfux5206}}

分量は目安. 好みでよい.

\begin{itemize}
\item
  肉 バラ肉やスペアリブなど脂の多い部位 500 g
\item
  水 2 l
\item
  ビーツ 大きめの 1個

  \begin{itemize}
  \tightlist
  \item
    缶詰や真空パックの水煮ではなく, 生のものが最適
  \end{itemize}
\item
  人参 1 - 2本
\item
  セロリ 1本

  \begin{itemize}
  \tightlist
  \item
    セロリアックが良いらしいが日本で売ってるのを見たことがないので茎と葉で良い.
  \end{itemize}
\item
  玉ねぎ 大 1個
\item
  ジャガイモ 1-2個
\item
  塩胡椒 適量
\item
  トマトピューレ 50g程度
\item
  (オプション) 香り付けのスパイス・ハーブ

  \begin{itemize}
  \tightlist
  \item
    ローリエの葉
  \item
    オールスパイス
  \end{itemize}
\item
  (オプション) 薬味
\item
  イタリアンパセリ
\item
  ディルの葉
\item
  万能ねぎ
\item
  サワークリーム
\end{itemize}

\hypertarget{ux9053ux5177-27}{%
\subsection{道具}\label{ux9053ux5177-27}}

\begin{itemize}
\tightlist
\item
  鍋
\item
  フライパン
\item
  野菜をすりおろす器具
\end{itemize}

\hypertarget{ux4f5cux308aux65b9-38}{%
\subsection{作り方}\label{ux4f5cux308aux65b9-38}}

まず肉と野菜を煮込んでブイヨンを作ってから,
改めて肉と野菜を入れ直して作る.

\begin{enumerate}
\def\labelenumi{\arabic{enumi}.}
\tightlist
\item
  ブイヨン用に玉ねぎ半分, 両端を切った人参を用意する
  (皮はむかなくてもよい)

  \begin{itemize}
  \tightlist
  \item
    玉ねぎは煮崩れしないように爪楊枝で留めておくと楽
  \end{itemize}
\item
  フライパンで肉に焼き目をつける. フライパンは洗わずにとっておく

  \begin{itemize}
  \tightlist
  \item
    クロポテンコによるとアク取りをサボるためのテクニックであり,
    必須ではない
  \end{itemize}
\item
  鍋に火をかけ水を注ぎ, 肉, 半分に切った玉ねぎ, 人参,
  セロリを1時間ほど煮込む
\item
  必要に応じてアク取りをする
\item
  玉ねぎ, セロリ, 人参, 肉を取りだす, 肉以外は不要
\item
  (オプション) 肉を食べやすい大きさに切って鍋に戻す

  \begin{itemize}
  \tightlist
  \item
    どうせ柔らかくなるので食べやすくない大きさにしても良い
  \end{itemize}
\item
  ジャガイモ, パプリカ, 人参を食べやすい大きさに切り,
  玉ねぎもみじん切りにする

  \begin{itemize}
  \tightlist
  \item
    今度は皮をむく
  \end{itemize}
\item
  ジャガイモを鍋に入れる

  \begin{itemize}
  \tightlist
  \item
    ジャガイモの大きさによっては茹で時間の間を開ける
  \end{itemize}
\item
  フライパンに油を引き, 上記を炒める

  \begin{itemize}
  \tightlist
  \item
    ビーツも一緒に炒めることが多いが, 熱しすぎると色が薄くなるので注意
  \end{itemize}
\item
  塩胡椒と砂糖を加える
\item
  炒めた野菜とローリエの葉を鍋に移す
\item
  トマトピューレとみじん切りにしたニンニクを鍋に入れる
\item
  ビーツも皮をむいてすりおろし, 鍋に入れる

  \begin{itemize}
  \tightlist
  \item
    ビーツの表面は固く筋っぽいので表皮だけでなく黒ずんだ部分も切り取ると良い
  \item
    すり下ろし器があると楽
  \end{itemize}
\item
  キャベツを千切りにして鍋に入れる

  \begin{itemize}
  \tightlist
  \item
    これもクロポテンコの好みによるもので, もっと早く入れても良い
  \end{itemize}
\item
  味見しつつ塩胡椒で味を調える
\item
  スープ皿などに盛り付け, 上から薬味をふりかける
\end{enumerate}

\hypertarget{ux88dcux8db3-37}{%
\subsection{補足}\label{ux88dcux8db3-37}}

黒パンやサーロ, ウォッカ/ホリルカとともに食べるとよい.

ロシア風に作る場合も大筋で同じだが, 肉は牛肉を使うこと. また,
キドニービーンズやザワークラウトを入れることもある.
材料はだいたい共通しているが,
どれくらい大きく切るかは個人差がかなりあるので正しい方法というものはおそらく存在しない.

トマトとビーツの色は全く違うので,
ボルシチはトマトで色を付けると思ってはならない
(トマトを入れない派閥も存在する).
さらにビーツの色素は熱分解するため長時間煮込むと色あせてしまうため,
炒めに時間をかけすぎないように注意.
缶詰や真空パックのビーツも加熱処理済みであるためどうしても色彩に劣ってしまう.
なるべく生のビーツを入手したい.

\hypertarget{ux53c2ux8003ux8cc7ux6599-34}{%
\subsection{参考資料}\label{ux53c2ux8003ux8cc7ux6599-34}}

\begin{itemize}
\item
  クロポテンコの料理動画 (ウクライナ語, 日本語字幕あり)
  \url{https://www.youtube.com/watch?v=LgjumwEC7zo}
\item
  クロポテンコの「ヘトマン風ボルシチ」レシピ (ウクライナ語, ロシア語)
  \url{https://klopotenko.com/getmanskij-borshh/}
\item
  ``Всегда Вкусно!'' の料理動画 (ロシア語)
  \url{https://www.youtube.com/watch?v=_-9nH4zFThs}
\item
  『ナスチャンネル』のロシア式の料理動画 (日本語音声)
  \url{https://www.youtube.com/watch?v=5mXOepWrzGc}
\item
  \citet{OginoNumano2017} pp.~36-37
\end{itemize}

\hypertarget{ux30b7ux30c1ux30fc-ux9732-ux449ux438}{%
\section{シチー (露: Щи)}\label{ux30b7ux30c1ux30fc-ux9732-ux449ux438}}

ボルシチより古い歴史があると言われるロシア料理.
以前公開したスライドと同じ.

\begin{figure}

{\centering \includegraphics[width=1\linewidth,height=1\textheight,keepaspectratio]{img/schi/finished} 

}

\caption{シチーその他}\label{fig:schi-finished}
\end{figure}

\begin{tabular}[t]{rl}
\toprule
 & 難易度\\
\midrule
材料調達 & {\fontspec{Noto Sans CJK JP} ★★★★☆ }\\
調理 & {\fontspec{Noto Sans CJK JP} ★★★☆☆ }\\
\bottomrule
\end{tabular}

\hypertarget{ux6750ux6599-21}{%
\subsection{材料}\label{ux6750ux6599-21}}

\begin{itemize}
\tightlist
\item
  牛肉

  \begin{itemize}
  \tightlist
  \item
    煮込み料理に適した部位が良い
  \end{itemize}
\item
  玉ねぎ
\item
  人参
\item
  セロリ
\item
  塩漬けキノコ
\item
  ザワークラウト
\item
  ジャガイモ
\item
  ニンニク
\item
  塩胡椒
\item
  (オプション) 付け合せ

  \begin{itemize}
  \tightlist
  \item
    キノコの塩漬け
  \item
    サワークリーム
  \item
    ディルの葉
  \item
    パセリ
  \item
    青ネギ
  \end{itemize}
\end{itemize}

\hypertarget{ux9053ux5177-28}{%
\subsection{道具}\label{ux9053ux5177-28}}

\begin{itemize}
\tightlist
\item
  鍋
\item
  できれば土鍋も欲しい
\end{itemize}

\hypertarget{ux4f5cux308aux65b9-39}{%
\subsection{作り方}\label{ux4f5cux308aux65b9-39}}

\begin{enumerate}
\def\labelenumi{\arabic{enumi}.}
\tightlist
\item
  ボルシチと同様にブイヨンを作る

  \begin{enumerate}
  \def\labelenumii{\arabic{enumii}.}
  \tightlist
  \item
    牛肉を下茹でし, アクを取る
  \item
    アクをあらかた取ったら, 玉ねぎとニンジンとセロリを入れる
  \item
    牛骨があれば使ってもよい
  \end{enumerate}
\item
  軽く絞ったザワークラウトと, それが浸かるくらいの水を入れる
\item
  バターを小さじ2杯入れ, 弱火で10分ほど蒸す
\item
  上記とみじん切りにした玉ねぎとニンジン,
  そしてジャガイモを鍋に入れてよく煮る
\item
  塩胡椒で味を調える

  \begin{itemize}
  \tightlist
  \item
    ザワークラウトの塩気があるためおそらくあまり必要ない
  \end{itemize}
\item
  火を止める直前にみじん切りにしたニンニクを入れる
\item
  サワークリーム, ディルの葉, パセリ, 青ネギ,
  キノコの塩漬けなどを添えて食べる
\end{enumerate}

\hypertarget{ux88dcux8db3-38}{%
\subsection{補足}\label{ux88dcux8db3-38}}

ゲニス\&ワイリによれば,
「シチーはスプーンが立つほど具だくさんにするのが良い」とのこと.

\hypertarget{ux53c2ux8003ux8cc7ux6599-35}{%
\subsection{参考資料}\label{ux53c2ux8003ux8cc7ux6599-35}}

\begin{itemize}
\tightlist
\item
  \citet{OginoNumano2017} p.~25
\item
  \citet{boumei} Ch. 3
\end{itemize}

\hypertarget{ux30bdux30eaux30e3ux30f3ux30abux30beux30ebux30e4ux30f3ux30ab-ux5b87ux9732-ux441ux43eux43bux44fux43dux43aux430ux72ec-soljanka}{%
\section{\texorpdfstring{ソリャンカ/ゾルヤンカ (宇/露: Солянка/独:
Soljanka)
\index{ソリャンカ}\index{ゾルヤンカ|see{ソリャンカ}}\index{soljanka|see{ソリャンカ}}\index{Солянка|see{ソリャンカ}}}{ソリャンカ/ゾルヤンカ (宇/露: Солянка/独: Soljanka) }}\label{ux30bdux30eaux30e3ux30f3ux30abux30beux30ebux30e4ux30f3ux30ab-ux5b87ux9732-ux441ux43eux43bux44fux43dux43aux430ux72ec-soljanka}}

ロシア (ウクライナ) 発祥のスープ (図\ref{fig:solyanka-finished})
と東ドイツローカライズされたもの.

\begin{figure}

{\centering \includegraphics[width=1\linewidth,height=1\textheight,keepaspectratio]{img/solyanka/finished} 

}

\caption{ロシア風ソリャンカ}\label{fig:solyanka-finished}
\end{figure}

\begin{tabular}[t]{rl}
\toprule
 & 難易度\\
\midrule
材料調達 & {\fontspec{Noto Sans CJK JP} ★★★★★ }\\
調理 & {\fontspec{Noto Sans CJK JP} ★★★☆☆ }\\
\bottomrule
\end{tabular}

\hypertarget{ux6750ux6599-22}{%
\subsection{材料}\label{ux6750ux6599-22}}

\begin{itemize}
\tightlist
\item
  牛肉
\item
  牛タン
\item
  サラミ
\item
  ボローニャソーセージ
\item
  玉ねぎ
\item
  ニンジン
\item
  トマトピューレ
\item
  ザワークラウト
\item
  塩漬けキュウリ
\item
  バター
\item
  オリーブの実
\item
  ケーパーの塩漬け
\item
  万能ねぎ
\item
  レモン
\item
  ディルの葉
\item
  イタリアンパセリ
\item
  ベイリーフ
\item
  サワークリーム
\end{itemize}

\hypertarget{ux4f5cux308aux65b9-40}{%
\subsection{作り方}\label{ux4f5cux308aux65b9-40}}

以下を参考に

\url{https://speakerdeck.com/ktgrstsh/how-to-make-solyanka}

\hypertarget{ux88dcux8db3-39}{%
\subsection{補足}\label{ux88dcux8db3-39}}

ゾルヤンカも概ね同じであるが, パプリカを入れることが多い.
さらにキュウリの塩漬けはピクルスに置き換わっている.
ロシアのスープ料理によく見られる香草も振りかけないことが多い.

\hypertarget{ux53c2ux8003ux8cc7ux6599-36}{%
\subsection{参考資料}\label{ux53c2ux8003ux8cc7ux6599-36}}

\begin{itemize}
\tightlist
\item
  \citet{boumei} Ch. 6
\item
  \url{https://young-germany.jp/2019/04/ddr4/}
\item
  \url{https://www.youtube.com/watch?v=8KxbfDXqwyU}
\end{itemize}

\hypertarget{ux4e88ux5b9a-ux30d3ux30fcux30d5ux30b9ux30c8ux30edux30acux30ceux30d5-ux9732-ux431ux435ux444ux441ux442ux440ux43eux433ux430ux43dux43eux432ux433ux43eux432ux44fux434ux438ux43dux430-ux43fux43e-ux441ux442ux440ux43eux433ux430ux43dux43eux432ux441ux43aux438}{%
\section{\texorpdfstring{(予定) ビーフストロガノフ (露:
Бефстроганов/Говядина По-строгановски)
\index{ビーフストロガノフ}}{(予定) ビーフストロガノフ (露: Бефстроганов/Говядина По-строгановски) }}\label{ux4e88ux5b9a-ux30d3ux30fcux30d5ux30b9ux30c8ux30edux30acux30ceux30d5-ux9732-ux431ux435ux444ux441ux442ux440ux43eux433ux430ux43dux43eux432ux433ux43eux432ux44fux434ux438ux43dux430-ux43fux43e-ux441ux442ux440ux43eux433ux430ux43dux43eux432ux441ux43aux438}}

名前がロシアっぽくないのは当時のロシアの貴族階級がフランスかぶれだったためか.

\begin{figure}

{\centering \includegraphics[width=1\linewidth,height=1\textheight,keepaspectratio]{img/bef-stroganov/finished} 

}

\caption{ビーフストロガノフとライ麦パン}\label{fig:finished-bef-stroganov}
\end{figure}

\begin{tabular}[t]{rl}
\toprule
 & 難易度\\
\midrule
材料調達 & {\fontspec{Noto Sans CJK JP} ★★★★☆ }\\
調理 & {\fontspec{Noto Sans CJK JP} ★★★☆☆ }\\
\bottomrule
\end{tabular}

\hypertarget{ux6750ux6599-23}{%
\subsection{材料}\label{ux6750ux6599-23}}

\begin{itemize}
\tightlist
\item
  牛肉
\item
  玉ねぎ
\item
  トマト
\item
  サワークリーム
\end{itemize}

\hypertarget{ux4f5cux308aux65b9-41}{%
\subsection{作り方}\label{ux4f5cux308aux65b9-41}}

TODO

\hypertarget{ux88dcux8db3-40}{%
\subsection{補足}\label{ux88dcux8db3-40}}

\citet{OginoNumano2017} によればキノコのカーシャ,
または炊き込みご飯とも相性が良い.

\begin{figure}

{\centering \includegraphics[width=1\linewidth,height=1\textheight,keepaspectratio]{img/bef-stroganov/rice} 

}

\caption{キノコのカーシャとの盛り合わせ}\label{fig:bef-stroganov-rice}
\end{figure}

\hypertarget{ux53c2ux8003ux8cc7ux6599-37}{%
\subsection{参考資料}\label{ux53c2ux8003ux8cc7ux6599-37}}

\begin{itemize}
\tightlist
\item
  \citet{OginoNumano2017} pp.~42-43
\end{itemize}

\hypertarget{ux771fux306eux3046ux3069ux3093ux8c46ux8150-ux6c5fux6238}{%
\section{真のうどん豆腐
(江戸?)}\label{ux771fux306eux3046ux3069ux3093ux8c46ux8150-ux6c5fux6238}}

18世紀の『豆腐百珍』で100品目に紹介されている料理.

\begin{figure}

{\centering \includegraphics[width=1\linewidth,height=1\textheight,keepaspectratio]{img/udon-tofu/finished} 

}

\caption{実際のレシピよりかなり手抜きしている}\label{fig:udon-tofu}
\end{figure}

\begin{tabular}[t]{rl}
\toprule
 & 難易度\\
\midrule
材料調達 & {\fontspec{Noto Sans CJK JP} ★★☆☆☆ }\\
調理 & {\fontspec{Noto Sans CJK JP} ★★★★☆ }\\
\bottomrule
\end{tabular}

\hypertarget{ux6750ux6599-24}{%
\subsection{材料}\label{ux6750ux6599-24}}

\begin{itemize}
\tightlist
\item
  絹豆腐 1 丁
\item
  つけ汁

  \begin{itemize}
  \tightlist
  \item
    醤油 1 升
  \item
    酒 3 合
  \item
    だし汁 5 合
  \end{itemize}
\item
  薬味

  \begin{itemize}
  \tightlist
  \item
    大根おろし
  \item
    とうがらし粉
  \item
    ねぎの白根
  \item
    みかんの皮
  \item
    浅草海苔
  \end{itemize}
\end{itemize}

\hypertarget{ux9053ux5177-29}{%
\subsection{道具}\label{ux9053ux5177-29}}

\begin{itemize}
\tightlist
\item
  豆腐を崩さずに熱湯からすくうための大きめの器具
\end{itemize}

\hypertarget{ux4f5cux308aux65b9-42}{%
\subsection{作り方}\label{ux4f5cux308aux65b9-42}}

\begin{enumerate}
\def\labelenumi{\arabic{enumi}.}
\tightlist
\item
  豆腐をうどんのように細く切る
\item
  熱湯でゆでて温める
\item
  崩さないように取りだす

  \begin{itemize}
  \tightlist
  \item
    塩を加えて煮締めるとちょうどいい固さになる
  \end{itemize}
\item
  うどんのようにつけ汁や薬味とともに食べる
\end{enumerate}

\hypertarget{ux88dcux8db3-41}{%
\subsection{補足}\label{ux88dcux8db3-41}}

『豆腐百珍』にはところてんの突き出しを湯の中に浸けて豆腐を細く切れとあるが,
崩れさえしなければ方法はなんでもいいと思われる.

真のうどん豆腐から派生する料理も『豆腐百珍』にいくつか挙げられている.
ここでは冬の料理に挙げたが冷やして食べるものは紹介されていないため未知数である.

\hypertarget{ux53c2ux8003ux8cc7ux6599-38}{%
\subsection{参考資料}\label{ux53c2ux8003ux8cc7ux6599-38}}

\begin{itemize}
\tightlist
\item
  『豆腐百珍』\url{http://www.toyama-smenet.or.jp/~tohfu/tofuhyakutin.html}
\item
  『豆腐百珍』正本 \url{https://dl.ndl.go.jp/info:ndljp/pid/2536494}
\end{itemize}

\hypertarget{ancient-medieval}{%
\chapter{古代・中世編}\label{ancient-medieval}}

材料調達と作業のどちらかが極端に難しいものが多いが,
極端に簡単なものもある.

セクション\ref{cesnecka}の「チェスネチカ」の原型とされる料理は14世紀の料理書にも見られる.

\hypertarget{ux30ecux30f3ux30baux8c46ux306eux30b9ux30fcux30d7-ux6377-ux4e2dux4e16}{%
\section{レンズ豆のスープ (捷,
中世)}\label{ux30ecux30f3ux30baux8c46ux306eux30b9ux30fcux30d7-ux6377-ux4e2dux4e16}}

\begin{quote}
ひもじい思いをしたことがある?
数ヶ月間も豆のスープだけで暮らしたことがあるの?

\begin{flushright}
--- ファイナルファンタジータクティクス

\end{flushright}
\end{quote}

\ldots\ldots 実際には大量の豆を使った濃厚な料理なのでさほどひもじさは感じない.

\begin{figure}

{\centering \includegraphics[width=1\linewidth,height=1\textheight,keepaspectratio]{img/lentil-soup/finished} 

}

\caption{レンズ豆のスープ}\label{fig:finished-lentil-soup}
\end{figure}

\begin{tabular}[t]{rl}
\toprule
 & 難易度\\
\midrule
材料調達 & {\fontspec{Noto Sans CJK JP} ★★☆☆☆ }\\
調理 & {\fontspec{Noto Sans CJK JP} ★★★☆☆ }\\
\bottomrule
\end{tabular}

\hypertarget{ux6750ux6599-25}{%
\subsection{材料}\label{ux6750ux6599-25}}

\begin{itemize}
\tightlist
\item
  レンズ豆
\item
  ベーコン
\item
  玉ねぎ
\item
  ニンニク
\item
  ラード
\item
  塩胡椒
\item
  水またはブイヨン, または豆の茹で汁
\end{itemize}

\hypertarget{ux4f5cux308aux65b9-43}{%
\subsection{作り方}\label{ux4f5cux308aux65b9-43}}

\begin{enumerate}
\def\labelenumi{\arabic{enumi}.}
\tightlist
\item
  レンズ豆を2時間, 水に浸して柔らかくする
\item
  塩を加えてそのまま煮る
\item
  柔らかくなったら水を切る, ただし茹で汁を半分ほど残す
\item
  ベーコンを細かく切る
\item
  玉ねぎとニンニクもみじん切りにする
\item
  鍋を熱してラードを溶かし, ベーコンを炒める
\item
  焼き目がつくまでベーコンを炒める
\item
  玉ねぎを加え, きつね色になるまで炒める
\item
  豆の半分と水を加える
\item
  残り半分の豆をよく潰す

  \begin{itemize}
  \tightlist
  \item
    ミキサーを使っても良い
  \end{itemize}
\item
  これも鍋に入れてよく混ぜる
\item
  塩胡椒で味を調える
\end{enumerate}

\hypertarget{ux88dcux8db3-42}{%
\subsection{補足}\label{ux88dcux8db3-42}}

ひもじさを演出したい場合は豆やベーコンを少なくする.

現代的には Čočka na kyselo となり, 目玉焼きを乗せたりする.

\hypertarget{ux53c2ux8003ux8cc7ux6599-39}{%
\subsection{参考資料}\label{ux53c2ux8003ux8cc7ux6599-39}}

\begin{itemize}
\tightlist
\item
  Random Innkeeper の動画 (日本語字幕あり)
  \url{https://www.youtube.com/watch?v=5FbGKARjuBU}
\item
  Čočka na kyselo の料理動画 (英語)
  \url{https://www.youtube.com/watch?v=0uVRiei7yxk}
\end{itemize}

\hypertarget{ux3059ux308aux3064ux3076ux3057ux305fux30ecux30f3ux30baux8c46-ux6377-15ux4e16ux7d00}{%
\section{すりつぶしたレンズ豆 (捷,
15世紀)}\label{ux3059ux308aux3064ux3076ux3057ux305fux30ecux30f3ux30baux8c46-ux6377-15ux4e16ux7d00}}

スープよりも固形物が多いので満腹感がある. アレンジ版も同時に紹介する.

\begin{figure}

{\centering \includegraphics[width=1\linewidth,height=1\textheight,keepaspectratio]{img/mashed-lentil/finished} 

}

\caption{すりつぶしたレンズ豆}\label{fig:unnamed-chunk-49}
\end{figure}

\begin{tabular}[t]{rl}
\toprule
 & 難易度\\
\midrule
材料調達 & {\fontspec{Noto Sans CJK JP} ★★☆☆☆ }\\
調理 & {\fontspec{Noto Sans CJK JP} ★★☆☆☆ }\\
\bottomrule
\end{tabular}

\hypertarget{ux6750ux6599-26}{%
\subsection{材料}\label{ux6750ux6599-26}}

\begin{itemize}
\tightlist
\item
  レンズ豆 500 g
\item
  タマネギ 1個
\item
  塩 小さじ2杯
\item
  胡椒
\item
  ラード 大さじ1
\item
  水 300 ml
\item
  (オプション) ニンニク 2欠片
\item
  (オプション) 砂糖 大さじ2
\item
  (オプション) 酢, リンゴ酢がよい 大さじ6
\item
  (オプション) ベーコン 120 g
\item
  (オプション) 小麦粉 大さじ1
\end{itemize}

オプションは「豪華バージョン」を作るのに必要

\hypertarget{ux9053ux5177-30}{%
\subsection{道具}\label{ux9053ux5177-30}}

\begin{itemize}
\tightlist
\item
  野菜すりおろし器
\item
  ポテトマッシャー
\end{itemize}

\hypertarget{ux4f5cux308aux65b9-44}{%
\subsection{作り方}\label{ux4f5cux308aux65b9-44}}

\begin{enumerate}
\def\labelenumi{\arabic{enumi}.}
\tightlist
\item
  レンズ豆を水に浸けて柔らかくする
\item
  そのまま火にかけて茹でる
\item
  ほとんど茹で上がったら, 最後に塩を加える
\item
  鍋を加熱しラードを溶かす
\item
  すりおろした or みじん切りにしたニンジンと玉ねぎを加えて炒める

  \begin{itemize}
  \tightlist
  \item
    豪華版ではオプションの材料をここで投入し調理する
  \end{itemize}
\item
  胡椒と豆の茹で汁を加え, さらに数分火にかける
\item
  豆を加え, よく潰す
\end{enumerate}

\hypertarget{ux53c2ux8003ux8cc7ux6599-40}{%
\subsection{参考資料}\label{ux53c2ux8003ux8cc7ux6599-40}}

\begin{itemize}
\tightlist
\item
  Random Innkeeper の動画
  \url{https://www.youtube.com/watch?v=tkkB7AhjGFM}
\end{itemize}

\hypertarget{ux30d1ux30f3ux30b1ux30fcux30ad-ux6377-palaux10dinky}{%
\section{\texorpdfstring{パンケーキ (捷: palačinky)
\index{palačinky|see{パンケーキ}}}{パンケーキ (捷: palačinky) }}\label{ux30d1ux30f3ux30b1ux30fcux30ad-ux6377-palaux10dinky}}

単体ではただの薄いパンなのでジャムやベリーやチョコレートやヨーグルトを乗せて巻いて食べる

\begin{tabular}[t]{rl}
\toprule
 & 難易度\\
\midrule
材料調達 & {\fontspec{Noto Sans CJK JP} ★☆☆☆☆ }\\
調理 & {\fontspec{Noto Sans CJK JP} ★★☆☆☆ }\\
\bottomrule
\end{tabular}

\hypertarget{ux6750ux6599-27}{%
\subsection{材料}\label{ux6750ux6599-27}}

\begin{itemize}
\tightlist
\item
  バター
\item
  小麦粉 100g
\item
  牛乳 200 cc
\item
  卵 1個
\item
  パンケーキにつける甘い物 (参考資料参照)
\end{itemize}

\hypertarget{ux9053ux5177-31}{%
\subsection{道具}\label{ux9053ux5177-31}}

小さめのフライパン

\hypertarget{ux4f5cux308aux65b9-45}{%
\subsection{作り方}\label{ux4f5cux308aux65b9-45}}

\begin{enumerate}
\def\labelenumi{\arabic{enumi}.}
\tightlist
\item
  小麦粉と水と溶き卵をゆっくり混ぜる

  \begin{itemize}
  \tightlist
  \item
    塊ができないよう, 少しづつ混ぜ合わせるとよい
  \end{itemize}
\item
  フライパンを弱火にかけ, バターを引く
\item
  お玉一杯分の生地をゆっくり注ぐ
\item
  片面が焼けたら手首のスナップをきかせて裏返す

  \begin{itemize}
  \tightlist
  \item
    自信がなかったらヘラを使う
  \end{itemize}
\item
  両面を焼いたら取りだす
\end{enumerate}

\hypertarget{ux88dcux8db3-43}{%
\subsection{補足}\label{ux88dcux8db3-43}}

いわゆるホットケーキのような膨らましたものではなく, British pancake
と呼ばれるタイプと同じでクレープみたいな薄いものである.
どうやら古代ローマの時代からあるらしい. 製法もいたってシンプルなので,
チェコに限らず古代から中世にかけてのヨーロッパ世界では至るところで見られたのではないかと想像する.

\hypertarget{ux53c2ux8003ux8cc7ux6599-41}{%
\subsection{参考資料}\label{ux53c2ux8003ux8cc7ux6599-41}}

\begin{itemize}
\tightlist
\item
  Random Innkeeper のレシピ, チェコ風のスプレッドのレシピつき
  \url{https://www.youtube.com/watch?v=okHyqdnKPfY}
\item
  Helltaker のクリア特典には作者 (ポーランド人)
  の祖母のレシピが含まれている
  \url{https://store.steampowered.com/app/1289310/Helltaker}
\end{itemize}

\hypertarget{ux4e88ux5b9a-ux4e2dux4e16ux306eux30c9ux30fcux30caux30c4-ux6377-16ux4e16ux7d00}{%
\section{(予定) 中世のドーナツ (捷,
16世紀)}\label{ux4e88ux5b9a-ux4e2dux4e16ux306eux30c9ux30fcux30caux30c4-ux6377-16ux4e16ux7d00}}

現代のポンチキと似ているが, 甘いものは使わない塩気の多い料理である.

TODO

\begin{tabular}[t]{rl}
\toprule
 & 難易度\\
\midrule
材料調達 & {\fontspec{Noto Sans CJK JP} ★★★★☆ }\\
調理 & {\fontspec{Noto Sans CJK JP} ★★★★☆ }\\
\bottomrule
\end{tabular}

\hypertarget{ux30edux30fcux30ebux30d1ux30f3-ux6377-15ux4e16ux7d00}{%
\section{ロールパン (捷,
15世紀)}\label{ux30edux30fcux30ebux30d1ux30f3-ux6377-15ux4e16ux7d00}}

かなり脂っこいパン

\begin{figure}

{\centering \includegraphics[width=1\linewidth,height=1\textheight,keepaspectratio]{img/roll-bread/finished} 

}

\caption{中世のロールパン}\label{fig:unnamed-chunk-53}
\end{figure}

\begin{tabular}[t]{rl}
\toprule
 & 難易度\\
\midrule
材料調達 & {\fontspec{Noto Sans CJK JP} ★★★☆☆ }\\
調理 & {\fontspec{Noto Sans CJK JP} ★★★☆☆ }\\
\bottomrule
\end{tabular}

\hypertarget{ux6750ux6599-28}{%
\subsection{材料}\label{ux6750ux6599-28}}

\begin{itemize}
\tightlist
\item
  小麦粉 500 g
\item
  卵 1個
\item
  ドライイースト 5g
\item
  ラード 150g
\item
  キャラウェイシード 適量
\item
  クラックリング (油かす) 300g
\item
  室温に戻したミルク 120 ml
\item
  白ワイン 200 ml
\item
  蜂蜜 小さじ1
\item
  塩 適量
\end{itemize}

\hypertarget{ux9053ux5177-32}{%
\subsection{道具}\label{ux9053ux5177-32}}

\begin{itemize}
\tightlist
\item
  ボウル
\item
  オーブン, またはオーブントースター
\end{itemize}

\hypertarget{ux4f5cux308aux65b9-46}{%
\subsection{作り方}\label{ux4f5cux308aux65b9-46}}

\begin{enumerate}
\def\labelenumi{\arabic{enumi}.}
\tightlist
\item
  ボウルに小麦粉とラードを入れるが, かき混ぜてはならない
\item
  小麦粉にくぼみを作ってイーストを入れる
\item
  牛乳と蜂蜜をイーストにかける
\item
  予備発酵を待つ

  \begin{itemize}
  \tightlist
  \item
    15分程度かかる
  \end{itemize}
\item
  塩とキャラウェイシードを好みの量だけ加える
\item
  卵と白ワインを注ぎ混ぜ, 生地をこねる
\item
  細かく砕いたクラックリングを混ぜる
\item
  布を被せて3時間ほど寝かせる
\item
  オーブントレーに油をぬる
\item
  適当な大きさにちぎって並べる

  \begin{enumerate}
  \def\labelenumii{\arabic{enumii}.}
  \tightlist
  \item
    トレーにくっつかないよう下に打ち粉すると良い
  \item
    このとき塩やキャラウェイシードを追加でふりかけても良い
  \end{enumerate}
\item
  上をナイフでひっかき切れ目を入れる
\item
  牛乳をはけで塗る(図\ref{fig:bread-roll-before-bake})
\item
  10分間寝かせる
\item
  オーブンを190度にして10-20分焼く

  \begin{itemize}
  \tightlist
  \item
    オーブンがないなら体感で努力する
  \end{itemize}
\end{enumerate}

\begin{figure}

{\centering \includegraphics[width=1\linewidth,height=1\textheight,keepaspectratio]{img/roll-bread/bred_roll_dough} 

}

\caption{焼く直前}\label{fig:bread-roll-before-bake}
\end{figure}

\hypertarget{ux88dcux8db3-44}{%
\subsection{補足}\label{ux88dcux8db3-44}}

元はハンガリーの料理であるらしい.

油かすは背脂からラードを精製する際に発生する.
もちろん余った脂身を炙って作ってもいい.

\hypertarget{ux53c2ux8003ux8cc7ux6599-42}{%
\subsection{参考資料}\label{ux53c2ux8003ux8cc7ux6599-42}}

\begin{itemize}
\tightlist
\item
  Random Innkeeper の動画
  \url{https://www.youtube.com/watch?v=Lm-FoNIFJE0}
\end{itemize}

\hypertarget{ux4e88ux5b9a-ux30a4ux30e9ux30afux30b5ux306eux30b7ux30c1ux30e5ux30fc-ux6377-ux4e2dux4e16}{%
\section{(予定) イラクサのシチュー (捷,
中世)}\label{ux4e88ux5b9a-ux30a4ux30e9ux30afux30b5ux306eux30b7ux30c1ux30e5ux30fc-ux6377-ux4e2dux4e16}}

COVID-19 に対する効果は明らかになっていない.

\hypertarget{ux8089ux306aux3057ux30bdux30fcux30bbux30fcux30b8-ux6377-15ux4e16ux7d00}{%
\section{\texorpdfstring{肉なしソーセージ (捷, 15世紀)
\index{肉なしソーセージ}}{肉なしソーセージ (捷, 15世紀) }}\label{ux8089ux306aux3057ux30bdux30fcux30bbux30fcux30b8-ux6377-15ux4e16ux7d00}}

中世の料理書に「ソーセージ」という名前で掲載されているレシピ.
しかし肉を一切使わない.

\begin{figure}

{\centering \includegraphics[width=1\linewidth,height=1\textheight,keepaspectratio]{img/sausage/finished} 

}

\caption{「ソーセージ」}\label{fig:unnamed-chunk-55}
\end{figure}

\begin{tabular}[t]{rl}
\toprule
 & 難易度\\
\midrule
材料調達 & {\fontspec{Noto Sans CJK JP} ★★☆☆☆ }\\
調理 & {\fontspec{Noto Sans CJK JP} ★★★☆☆ }\\
\bottomrule
\end{tabular}

\hypertarget{ux6750ux6599-29}{%
\subsection{材料}\label{ux6750ux6599-29}}

\begin{itemize}
\tightlist
\item
  卵 6個
\item
  小麦粉 大さじ6
\item
  レーズン 50g
\item
  パセリ 適量
\item
  セージ 適量
\item
  バター 適量
\end{itemize}

\hypertarget{ux4f5cux308aux65b9-47}{%
\subsection{作り方}\label{ux4f5cux308aux65b9-47}}

\begin{enumerate}
\def\labelenumi{\arabic{enumi}.}
\tightlist
\item
  卵でスクランブルエッグを作る

  \begin{itemize}
  \tightlist
  \item
    あとで成形しづらくなるので加熱しすぎてはならない
  \end{itemize}
\item
  小麦粉, レーズン, パセリ, バターを加えてよくこねる
\item
  ソーセージのように細長く成形する
\item
  フライパンにバターを溶かす
\item
  「ソーセージ」を焼く
\item
  接している面が焼けたら少しづつ転がし他の面も焼く
\end{enumerate}

\hypertarget{ux53c2ux8003ux8cc7ux6599-43}{%
\subsection{参考資料}\label{ux53c2ux8003ux8cc7ux6599-43}}

\begin{itemize}
\tightlist
\item
  Random Innkeeper の動画 (英語音声, 日本語字幕あり)
  \url{https://www.youtube.com/watch?v=dKNu_qnQZkM}
\end{itemize}

\hypertarget{ux30d1ux30f3ux7c89ux306eux30c9ux30fcux30caux30c4ux3068ux6797ux6a8eux306eux30ddux30eaux30c3ux30b8-ux6377-15ux4e16ux7d00}{%
\section{パン粉のドーナツと林檎のポリッジ (捷,
15世紀)}\label{ux30d1ux30f3ux7c89ux306eux30c9ux30fcux30caux30c4ux3068ux6797ux6a8eux306eux30ddux30eaux30c3ux30b8-ux6377-15ux4e16ux7d00}}

この料理は šišky と表現されているため,
辞書的には「団子」「ダンプリング」である(図\ref{fig:breadcrumbs-dumpling-finished}).
しかし, バターで焼き上げる工程があるためここでは「ドーナツ」と書いた.

\begin{figure}

{\centering \includegraphics[width=1\linewidth,height=1\textheight,keepaspectratio]{img/breadcrumbs-dumplings/finished} 

}

\caption{パン粉のドーナツと林檎のポリッジ}\label{fig:breadcrumbs-dumpling-finished}
\end{figure}

\begin{tabular}[t]{rl}
\toprule
 & 難易度\\
\midrule
材料調達 & {\fontspec{Noto Sans CJK JP} ★★★☆☆ }\\
調理 & {\fontspec{Noto Sans CJK JP} ★★☆☆☆ }\\
\bottomrule
\end{tabular}

\hypertarget{ux6750ux6599-30}{%
\subsection{材料}\label{ux6750ux6599-30}}

正確に量らずに作ったので適切な比率は不明. 適宜継ぎ足してほしい.

\begin{itemize}
\tightlist
\item
  ドーナツ

  \begin{itemize}
  \tightlist
  \item
    卵 2-3個
  \item
    パン粉 適量
  \item
    砂糖 適量
  \item
    干しぶどう 適量
  \item
    バター
  \end{itemize}
\item
  ポリッジ

  \begin{itemize}
  \tightlist
  \item
    りんご
  \item
    シナモンスティック
  \item
    白ワイン
  \item
    バター
  \end{itemize}
\end{itemize}

\hypertarget{ux4f5cux308aux65b9-48}{%
\subsection{作り方}\label{ux4f5cux308aux65b9-48}}

ドーナツ

\begin{enumerate}
\def\labelenumi{\arabic{enumi}.}
\tightlist
\item
  ボウルに卵を割り, よくかき混ぜる
\item
  砂糖と干しブドウを入れる
\item
  パン粉を少しづつ入れながらかき混ぜる
\item
  固まってきたらこねて生地にする
\item
  団子の大きさにする

  \begin{itemize}
  \tightlist
  \item
    あまり大きすぎると中まで火が通らない.
  \end{itemize}
\item
  フライパンを火にかけ, バターを多めに溶かす
\item
  表面に焼き目が付く程度に団子を数分焼く
\item
  フライパンから取り出し, 油を切る
\end{enumerate}

ポリッジ

\begin{enumerate}
\def\labelenumi{\arabic{enumi}.}
\tightlist
\item
  りんごの皮をむき, みじん切りにする
\item
  鍋を弱火にかけ, バターを溶かす
\item
  りんごを入れて焦げ付かないように加熱する
\item
  シナモンスティックを入れる
\item
  必要に応じて水を足し, りんごが柔らかく崩れるまで加熱する
\item
  白ワインを注ぎ, さらに数分加熱する
\end{enumerate}

ポリッジの上にドーナツを乗せると良いだろう.

\hypertarget{ux88dcux8db3-45}{%
\subsection{補足}\label{ux88dcux8db3-45}}

日本のパン粉は粗いので袋にいれて叩くなどして粒を細かくしてから使うと良い.
しかし当然ながら小麦粉よりは目が粗いので,
あまりきめの細かい生地にはならない.

りんごのポリッジは \citet{Ju2018} pp.~78-79
に類似したイングランドのレシピが見られる.

\hypertarget{ux53c2ux8003ux6587ux732e-1}{%
\subsection{参考文献}\label{ux53c2ux8003ux6587ux732e-1}}

\begin{itemize}
\tightlist
\item
  Random Innkeeper の動画
  \url{https://www.youtube.com/watch?v=J4iEFjWsMqM}
\item
  \citet{Ju2018}
\end{itemize}

\hypertarget{ux30d7ux30ecux30c3ux30c4ux30a7ux30eb-ux6377-15ux4e16ux7d00-plecluxedk}{%
\section{プレッツェル (捷, 15世紀:
pleclík)}\label{ux30d7ux30ecux30c3ux30c4ux30a7ux30eb-ux6377-15ux4e16ux7d00-plecluxedk}}

現代のドイツやオーストリアのものとはかなり違う中世のレシピ
(図\ref{fig:pleclik-finished}). イーストも苛性ソーダも使用しない.
グラインダーが無いと重労働なのでもう作りたくない.

\begin{figure}

{\centering \includegraphics[width=1\linewidth,height=1\textheight,keepaspectratio]{img/preclik/finished} 

}

\caption{中世のプレッツェル}\label{fig:pleclik-finished}
\end{figure}

\begin{tabular}[t]{rl}
\toprule
 & 難易度\\
\midrule
材料調達 & {\fontspec{Noto Sans CJK JP} ★★☆☆☆ }\\
調理 & {\fontspec{Noto Sans CJK JP} ★★★★★★★★★★ }\\
\bottomrule
\end{tabular}

\hypertarget{ux6750ux6599-4ux500bux5206}{%
\subsection{材料 (4個分)}\label{ux6750ux6599-4ux500bux5206}}

\begin{itemize}
\tightlist
\item
  小麦粉 200 g
\item
  蜂蜜 100 ml + 「蜂蜜粉」の3/4 *
\item
  油または小麦粉 少量
\item
  (オプション) 味付け用のスパイス
  (胡椒、クローブ、ショウガ、オールスパイス、シナモンなど)
\end{itemize}

\hypertarget{ux9053ux5177-33}{%
\subsection{道具}\label{ux9053ux5177-33}}

\begin{itemize}
\tightlist
\item
  オーブンまたはオーブントースター
\item
  固いものを細かく粉砕するグラインダー
\end{itemize}

\hypertarget{ux4f5cux308aux65b9-49}{%
\subsection{作り方}\label{ux4f5cux308aux65b9-49}}

\begin{enumerate}
\def\labelenumi{\arabic{enumi}.}
\tightlist
\item
  蜂蜜を温めて溶かす
\item
  火傷に注意して小麦粉と混ぜてこねる

  \begin{itemize}
  \tightlist
  \item
    粘りのある生地にするのは難しいのでほどほどでやめる
  \end{itemize}
\item
  オーブントレーに油を塗る, または小麦粉を振る
\item
  平たくつぶした生地を乗せ, 170度に熱したオーブンで,
  表面が焼きあがるまで焼く

  \begin{itemize}
  \tightlist
  \item
    20分程度かかる
  \end{itemize}
\item
  焼いた生地を細かく粉砕する

  \begin{itemize}
  \tightlist
  \item
    グラインダーがないと極めて重労働となる
  \end{itemize}
\item
  粉末の 3/4 の量の蜂蜜を温めて溶かし, 混ぜる

  \begin{itemize}
  \tightlist
  \item
    多すぎるとベタつくのでやや少なめでも良い
  \end{itemize}
\item
  オプションでスパイスを粉末にして混ぜる
\item
  細長くこねてプレッツェルの形にする
\item
  天日干しか, オーブンで焼いて固める
\end{enumerate}

\hypertarget{ux53c2ux8003ux8cc7ux6599-44}{%
\subsection{参考資料}\label{ux53c2ux8003ux8cc7ux6599-44}}

\begin{itemize}
\tightlist
\item
  Random Innkeeper の動画 (英語, 日本語字幕あり)
  \url{https://www.youtube.com/watch?v=6JXrlNuYHF8}
\end{itemize}

\hypertarget{ux30a8ux30f3ux30c9ux30a6ux30deux30e1ux306eux56e3ux5b50-ux6377-15ux4e16ux7d00}{%
\section{エンドウマメの団子 (捷,
15世紀)}\label{ux30a8ux30f3ux30c9ux30a6ux30deux30e1ux306eux56e3ux5b50-ux6377-15ux4e16ux7d00}}

すぐ崩れるので難しい上にぶっちゃけうまくない.

\begin{figure}

{\centering \includegraphics[width=1\linewidth,height=1\textheight,keepaspectratio]{img/pea-ball/finished} 

}

\caption{最大限うまそうに見せた様子}\label{fig:pea-ball-finished}
\end{figure}

\begin{tabular}[t]{rl}
\toprule
 & 難易度\\
\midrule
材料調達 & {\fontspec{Noto Sans CJK JP} ★★★☆☆ }\\
調理 & {\fontspec{Noto Sans CJK JP} ★★★★☆ }\\
\bottomrule
\end{tabular}

\hypertarget{ux6750ux6599-31}{%
\subsection{材料}\label{ux6750ux6599-31}}

\begin{itemize}
\tightlist
\item
  エンドウマメ 500 g
\item
  砂糖 50 - 100 g
\item
  干しブドウ 100 g
\item
  油
\item
  追加の砂糖
\end{itemize}

\hypertarget{ux4f5cux308aux65b9-50}{%
\subsection{作り方}\label{ux4f5cux308aux65b9-50}}

\begin{enumerate}
\def\labelenumi{\arabic{enumi}.}
\tightlist
\item
  エンドウマメを数時間水に浸す
\item
  鍋で茹でて柔らかくする

  \begin{itemize}
  \tightlist
  \item
    この時点では柔らかすぎてはならない
  \end{itemize}
\item
  水気を良く切り, 砂糖を加えて潰してかき混ぜる
\item
  手で触れられる温度になるまで待つ
\item
  適当な量を手に取り, 干しブドウを何粒か包んで団子にする
\item
  フライパンに油を引き, 絡める色になるまで弱火でこれを焼く
\item
  砂糖をまぶす
\end{enumerate}

\hypertarget{ux88dcux8db3-46}{%
\subsection{補足}\label{ux88dcux8db3-46}}

火加減を間違えると固まりにくくなるが, 適切な方法はわからない.

植物性の油を使えばいちおうは断食用になる.

小豆とか甘い豆をつかうとうまくなるかもしれないが,
中世ヨーロッパにはたぶん小豆はない.

\hypertarget{ux53c2ux8003ux8cc7ux6599-45}{%
\subsection{参考資料}\label{ux53c2ux8003ux8cc7ux6599-45}}

\begin{itemize}
\tightlist
\item
  Random Innkeeper の動画
  \url{https://www.youtube.com/watch?v=XSOw8thwkgU}
\end{itemize}

\hypertarget{ux4e88ux5b9a-4ux98dfux306eux30ddux30eaux30c3ux30b8}{%
\section{(予定)
4食のポリッジ}\label{ux4e88ux5b9a-4ux98dfux306eux30ddux30eaux30c3ux30b8}}

\citet{feyfrlikova2015Kuchyne} p.~66 がたぶん元ネタ

\hypertarget{ux53c2ux8003ux8cc7ux6599-46}{%
\subsection{参考資料}\label{ux53c2ux8003ux8cc7ux6599-46}}

\begin{itemize}
\tightlist
\item
  Random Innkeeper の動画
  \url{https://www.youtube.com/watch?v=J4iEFjWsMqM}
\item
  \citet{feyfrlikova2015Kuchyne}
\end{itemize}

\hypertarget{ux8c5aux8089ux306eux30d3ux30fcux30ebux30bdux30fcux30b9ux548cux3048-vepux159ovina-divokuxe1-ux6377-16ux4e16ux7d00}{%
\section{\texorpdfstring{豚肉のビールソース和え (vepřovina divoká, 捷,
16世紀)
\index{vepřovina divoká|see{豚肉のビールソース和え}}}{豚肉のビールソース和え (vepřovina divoká, 捷, 16世紀) }}\label{ux8c5aux8089ux306eux30d3ux30fcux30ebux30bdux30fcux30b9ux548cux3048-vepux159ovina-divokuxe1-ux6377-16ux4e16ux7d00}}

おそらく酒場で古くなったビールやパンを再利用するレシピとのこと.
発泡酒ではなくビールを使おう.

\begin{figure}

{\centering \includegraphics[width=1\linewidth,height=1\textheight,keepaspectratio]{img/beer-pork/finished} 

}

\caption{豚肉のビールソース和え}\label{fig:beer-pork-finished}
\end{figure}

\begin{tabular}[t]{rl}
\toprule
 & 難易度\\
\midrule
材料調達 & {\fontspec{Noto Sans CJK JP} ★★★☆☆ }\\
調理 & {\fontspec{Noto Sans CJK JP} ★★★☆☆ }\\
\bottomrule
\end{tabular}

\hypertarget{ux6750ux6599-32}{%
\subsection{材料}\label{ux6750ux6599-32}}

\begin{itemize}
\tightlist
\item
  豚肉 300 g
\item
  ビール 500 ml
\item
  固くなったパン 食パン2-3枚程度
\item
  酢, リンゴ酢が良い
\item
  ショウガ
\item
  胡椒
\item
  塩
\item
  ニンニク
\item
  砂糖
\item
  ラード
\item
  (オプション) ブイヨン
\end{itemize}

\hypertarget{ux9053ux5177-34}{%
\subsection{道具}\label{ux9053ux5177-34}}

\begin{itemize}
\tightlist
\item
  ストレイナー/裏ごし器
\item
  豚を茹でるのと並行してソースを作るため, 鍋が2つあるとよい
\end{itemize}

\hypertarget{ux4f5cux308aux65b9-51}{%
\subsection{作り方}\label{ux4f5cux308aux65b9-51}}

\begin{enumerate}
\def\labelenumi{\arabic{enumi}.}
\tightlist
\item
  多めに塩を加えた水またはブイヨンで豚肉を茹でる
\item
  別の鍋にビールと少量の酢を注ぎ, パンを砕いて入れる
\item
  パンをほぐしながらビールと混ぜる
\item
  ストレイナーにかけ, 濾し取ったものを再度加熱する
\item
  上記の調味料とスパイスで味を調える

  \begin{itemize}
  \tightlist
  \item
    砂糖を多めに入れるとよい
  \end{itemize}
\end{enumerate}

\hypertarget{ux88dcux8db3-47}{%
\subsection{補足}\label{ux88dcux8db3-47}}

本来のレシピではソースの材料はビール, 酢,
パンしか言及されていなかったが,
それだけではとうてい食べられたものではないため上記のスパイスで味付けしている.

\hypertarget{ux53c2ux8003ux8cc7ux6599-47}{%
\subsection{参考資料}\label{ux53c2ux8003ux8cc7ux6599-47}}

\begin{itemize}
\tightlist
\item
  Random Innkeeper の動画
  \url{https://www.youtube.com/watch?v=qWsfd6tn2VY}
\end{itemize}

\hypertarget{ux9d8fux8089ux3068ux30d6ux30e9ux30c3ux30afux30bdux30fcux30b9-kuux159e-v-ux10dernuxe9-juxedux161e-ux6377-15ux4e16ux7d00}{%
\section{鶏肉とブラックソース kuře v černé jíše (捷,
15世紀)}\label{ux9d8fux8089ux3068ux30d6ux30e9ux30c3ux30afux30bdux30fcux30b9-kuux159e-v-ux10dernuxe9-juxedux161e-ux6377-15ux4e16ux7d00}}

\begin{figure}

{\centering \includegraphics[width=1\linewidth,height=1\textheight,keepaspectratio]{img/chicken-dark-sauce/finished} 

}

\caption{鶏肉とブラックソース}\label{fig:chicken-dark-sauce-finished}
\end{figure}

\begin{tabular}[t]{rl}
\toprule
 & 難易度\\
\midrule
材料調達 & {\fontspec{Noto Sans CJK JP} ★★☆☆☆ }\\
調理 & {\fontspec{Noto Sans CJK JP} ★★★☆☆ }\\
\bottomrule
\end{tabular}

\hypertarget{ux6750ux6599-33}{%
\subsection{材料}\label{ux6750ux6599-33}}

\begin{itemize}
\tightlist
\item
  鶏胸肉
\item
  リンゴ
\item
  赤ワイン
\item
  パン粉
\item
  シナモン
\item
  胡椒
\item
  サフラン
\item
  ショウガ
\item
  塩
\item
  砂糖
\end{itemize}

\hypertarget{ux4f5cux308aux65b9-52}{%
\subsection{作り方}\label{ux4f5cux308aux65b9-52}}

\begin{enumerate}
\def\labelenumi{\arabic{enumi}.}
\tightlist
\item
  鍋に湯を沸かし, 多めに塩を加える
\item
  軽く沸騰したら鶏肉を入れる
\item
  乳鉢でスパイスを全て挽く
\item
  フライパンを加熱し, パン粉を熱する
\item
  焦がさないように茶色くなるまで煎る
\item
  フライパンに赤ワインとスパイスを注ぐ
\item
  よく混ぜる
\item
  鶏肉が半分ほど煮えたら取り出す
\item
  鶏肉をフライパンに入れ, ソースを絡める
\item
  赤ワインまたは茹で汁を少し加える
\item
  鶏肉全体にソースの色が染み込むまで調理する
\item
  リンゴをスライスする
\item
  フライパンにリンゴと少々の赤ワインを入れ, 蓋をして蒸す
\item
  リンゴに色が移ったら取りだす
\item
  好みで砂糖を使いソースの味を調える
\item
  全て皿に盛り付ける
\end{enumerate}

\hypertarget{ux88dcux8db3-48}{%
\subsection{補足}\label{ux88dcux8db3-48}}

味の濃い食材がないため,
鶏肉の下茹で時には十分に塩を入れて味付けしておく.

\hypertarget{ux53c2ux8003ux8cc7ux6599-48}{%
\subsection{参考資料}\label{ux53c2ux8003ux8cc7ux6599-48}}

\begin{itemize}
\tightlist
\item
  Random Innkeeper の料理
  \url{https://www.youtube.com/watch?v=XEdpYA_xHLM}
\end{itemize}

\hypertarget{ux30cfux30f3ux30acux30eaux30fcux98a8ux30edux30fcux30b9ux30c8ux30c1ux30adux30f3-kuux159e-po-uhersku-ux6377-15ux4e16ux7d00}{%
\section{\texorpdfstring{ハンガリー風ローストチキン (kuře po Uhersku,
捷, 15世紀)
\index{kuře po Uhersku|see{ハンガリー風ローストチキン}}}{ハンガリー風ローストチキン (kuře po Uhersku, 捷, 15世紀) }}\label{ux30cfux30f3ux30acux30eaux30fcux98a8ux30edux30fcux30b9ux30c8ux30c1ux30adux30f3-kuux159e-po-uhersku-ux6377-15ux4e16ux7d00}}

例によって「ハンガリー風」という名の中世チェコ料理.
現在の同名の料理はパプリカやトマトを使った赤いソースで鶏肉を煮込んだものだが,
当時のヨーロッパにはトマトはない.

\begin{figure}

{\centering \includegraphics[width=1\linewidth,height=1\textheight,keepaspectratio]{img/hungarian-chicken/finished} 

}

\caption{かなり小さい鶏で料理した}\label{fig:hungarian-chicken-finished}
\end{figure}

\begin{tabular}[t]{rl}
\toprule
 & 難易度\\
\midrule
材料調達 & {\fontspec{Noto Sans CJK JP} ★★★★☆ }\\
調理 & {\fontspec{Noto Sans CJK JP} ★★★☆☆ }\\
\bottomrule
\end{tabular}

\hypertarget{ux6750ux6599-34}{%
\subsection{材料}\label{ux6750ux6599-34}}

\begin{itemize}
\tightlist
\item
  ローストチキン用の鶏肉 1羽
\item
  詰め物

  \begin{itemize}
  \tightlist
  \item
    ベーコン 50 g
  \item
    粟 150 g
  \item
    生クリーム 40 ml
  \item
    レーズン 1掴み
  \item
    卵 2個
  \end{itemize}
\item
  ソースの材料

  \begin{itemize}
  \tightlist
  \item
    生クリーム 200 ml
  \item
    サフラン
  \item
    生姜
  \item
    砂糖
  \item
    卵黄 1個
  \end{itemize}
\end{itemize}

\hypertarget{ux9053ux5177-35}{%
\subsection{道具}\label{ux9053ux5177-35}}

\begin{itemize}
\tightlist
\item
  オーブン
\end{itemize}

\hypertarget{ux4f5cux308aux65b9-53}{%
\subsection{作り方}\label{ux4f5cux308aux65b9-53}}

\begin{enumerate}
\def\labelenumi{\arabic{enumi}.}
\tightlist
\item
  卵をかたゆで卵にする
\item
  粟を塩水でよく茹でる
\item
  ベーコンを小さくさいの目切りにする
\item
  ゆで卵も同様に小さく切る
\item
  フライパンに油を引き, ベーコンを炒める
\item
  さらにゆで卵と粟も加えてよく混ぜる
\item
  レーズンと生クリームも加える
\item
  柔らかさを残しつつ, 余分な水分が飛ぶまで加熱する
\item
  鶏肉に詰める
\item
  頭と尻を縛って漏れないようにする
\item
  オーブンを160度Cに加熱し1時間焼く
\item
  その間にソースを作る

  \begin{enumerate}
  \def\labelenumii{\arabic{enumii}.}
  \tightlist
  \item
    鍋で生クリームを温める
  \item
    卵黄を入れ, よくかき混ぜる
  \item
    サフランと生姜を1つまみ加える
  \item
    濃いソースになるまで混ぜ続ける
  \end{enumerate}
\item
  鶏肉を切り分け, 上からソースをかける
\end{enumerate}

\hypertarget{ux88dcux8db3-49}{%
\subsection{補足}\label{ux88dcux8db3-49}}

オーブンがないのでかなり小さな鶏で試してみたが, ほとんど入らなかった.

\hypertarget{ux53c2ux8003ux8cc7ux6599-49}{%
\subsection{参考資料}\label{ux53c2ux8003ux8cc7ux6599-49}}

\begin{itemize}
\tightlist
\item
  Random Innkeeper の動画
  \url{https://www.youtube.com/watch?v=ZZxF8H4ArbA}
\end{itemize}

\hypertarget{ux9d8fux8089ux306eux30edux30fcux30d5ux3068ux9ed2ux3044ux30bdux30fcux30b9-ux6377-15ux4e16ux7d00-sekanina-slepiux10duxed-s-ux10dernou-omuxe1ux10dkou}{%
\section{\texorpdfstring{鶏肉のローフと黒いソース (捷, 15世紀, sekanina
slepičí s černou omáčkou)
\index{sekanina slepičí|see{鶏肉のローフ}}}{鶏肉のローフと黒いソース (捷, 15世紀, sekanina slepičí s černou omáčkou) }}\label{ux9d8fux8089ux306eux30edux30fcux30d5ux3068ux9ed2ux3044ux30bdux30fcux30b9-ux6377-15ux4e16ux7d00-sekanina-slepiux10duxed-s-ux10dernou-omuxe1ux10dkou}}

肉挽き器が発明されたのは19世紀であり,
それ以前のひき肉は包丁で念入りに叩くという重労働の産物であった.
しかしこの料理は挽き肉を使い, 見た目にもこだわった贅沢な料理である
(図\ref{fig:sekanina-slepici}).

\begin{figure}

{\centering \includegraphics[width=1\linewidth,height=1\textheight,keepaspectratio]{img/sekanina/finished} 

}

\caption{鶏肉のローフ (崩れている)}\label{fig:sekanina-slepici}
\end{figure}

\begin{tabular}[t]{rl}
\toprule
 & 難易度\\
\midrule
材料調達 & {\fontspec{Noto Sans CJK JP} ★★★☆☆ }\\
調理 & {\fontspec{Noto Sans CJK JP} ★★★★☆ }\\
\bottomrule
\end{tabular}

\hypertarget{ux6750ux6599-35}{%
\subsection{材料}\label{ux6750ux6599-35}}

\begin{itemize}
\tightlist
\item
  鶏もも肉 (骨付き)
\item
  ベーコン
\item
  卵
\item
  塩胡椒
\item
  パセリ
\item
  サフラン
\item
  ジンジャーパウダー

  \begin{itemize}
  \tightlist
  \item
    なければおろしショウガでもよい
  \end{itemize}
\item
  (オプション) 小麦粉
\end{itemize}

ソースの材料

\begin{itemize}
\tightlist
\item
  赤ワイン
\item
  パン粉
\item
  シナモン
\item
  砂糖
\end{itemize}

\hypertarget{ux9053ux5177-36}{%
\subsection{道具}\label{ux9053ux5177-36}}

\begin{itemize}
\tightlist
\item
  挽き肉器\footnote{挽き肉器の発明は19世紀になってからなので,
    中世気分を味わいたい場合は手作業でミンチにする}
\item
  複数の鍋
\end{itemize}

\hypertarget{ux4f5cux308aux65b9-54}{%
\subsection{作り方}\label{ux4f5cux308aux65b9-54}}

ミートローフの作り方

\begin{enumerate}
\def\labelenumi{\arabic{enumi}.}
\tightlist
\item
  モモ肉を柔らかくなるまで茹でる
\item
  骨から肉を外す (骨と茹で汁を捨ててはならない)
\item
  肉をミンチにする
\item
  卵を固茹でにする
\item
  ベーコンとゆで卵を細かくみじん切りにする
\item
  ベーコンをカリカリになるまで炒める
\item
  ミンチ, ベーコン, ゆで卵, パセリ, 塩胡椒,
  ジンジャーパウダーを混ぜてこねる

  \begin{itemize}
  \tightlist
  \item
    崩れやすかったら小麦粉を混ぜる
  \end{itemize}
\item
  骨に生地をつけて骨付き肉のような形にする
\item
  茹で汁に静かに入れ, 静かに茹でて固める
\end{enumerate}

ソースの作り方

\begin{enumerate}
\def\labelenumi{\arabic{enumi}.}
\tightlist
\item
  パン粉をフライパンで茶色くなるまで煎る
\item
  赤ワインを注ぐ
\item
  サフラン, シナモン, ジンジャーパウダーなどで香りを付ける

  \begin{itemize}
  \tightlist
  \item
    砂糖を加えても良い
  \end{itemize}
\end{enumerate}

\hypertarget{ux88dcux8db3-50}{%
\subsection{補足}\label{ux88dcux8db3-50}}

鶏肉に一度火を通してからひき肉にするため, 十分細かくないと崩れやすい.
よって手作業の場合は覚悟すること.

\hypertarget{ux53c2ux8003ux8cc7ux6599-50}{%
\subsection{参考資料}\label{ux53c2ux8003ux8cc7ux6599-50}}

\begin{itemize}
\tightlist
\item
  Random Innkeeper の動画
  \url{https://www.youtube.com/watch?v=vaBtiTrlmVc}
\item
  自作記録
  \url{https://under-identified.hatenablog.com/entry/2019/04/13/220424}
\end{itemize}

\hypertarget{ux30cfux30f3ux30acux30eaux30fcux98a8ux30edux30fcux30b9ux30c8ux30d3ux30fcux30d5-1-ux6377-15ux4e16ux7d00-peux10denux11b-hovux11bzuxed-po-uhersku}{%
\section{\texorpdfstring{「ハンガリー風」ローストビーフ \#1 (捷, 15世紀,
Pečeně hovězí po Uhersku)
\index{Pečeně hovězí po Uhersku|see{ハンガリー風ローストビーフ (15世紀)}}}{「ハンガリー風」ローストビーフ \#1 (捷, 15世紀, Pečeně hovězí po Uhersku) }}\label{ux30cfux30f3ux30acux30eaux30fcux98a8ux30edux30fcux30b9ux30c8ux30d3ux30fcux30d5-1-ux6377-15ux4e16ux7d00-peux10denux11b-hovux11bzuxed-po-uhersku}}

チェコの文献に見られる最も古いレシピの1つ.
ハンガリーに同じ料理があったのかは不明 (図\ref{fig:hungraian-beef15}).

\begin{figure}

{\centering \includegraphics[width=1\linewidth,height=1\textheight,keepaspectratio]{img/hungarian-beef15/finished} 

}

\caption{ハンガリー風牛肉}\label{fig:hungraian-beef15}
\end{figure}

\begin{tabular}[t]{rl}
\toprule
 & 難易度\\
\midrule
材料調達 & {\fontspec{Noto Sans CJK JP} ★★★★☆ }\\
調理 & {\fontspec{Noto Sans CJK JP} ★★★☆☆ }\\
\bottomrule
\end{tabular}

\hypertarget{ux6750ux6599-36}{%
\subsection{材料}\label{ux6750ux6599-36}}

\begin{itemize}
\tightlist
\item
  牛肉 400 g
\item
  玉ねぎ 小3個

  \begin{itemize}
  \tightlist
  \item
    玉ねぎの個数にこだわる必要はない
  \end{itemize}
\item
  赤ワインビネガー 50 ml
\item
  ブイヨン 300 ml
\item
  キャラウェイシード
\item
  ジュニパーベリー
\item
  黒胡椒
\item
  ショウガ
\item
  クローブ
\item
  メース
\item
  塩
\item
  (オプション) 砂糖
\end{itemize}

\hypertarget{ux9053ux5177-37}{%
\subsection{道具}\label{ux9053ux5177-37}}

\begin{itemize}
\tightlist
\item
  牛肉が入るオーブン
\item
  裏ごし器
\end{itemize}

\hypertarget{ux4f5cux308aux65b9-55}{%
\subsection{作り方}\label{ux4f5cux308aux65b9-55}}

\begin{enumerate}
\def\labelenumi{\arabic{enumi}.}
\tightlist
\item
  牛肉を水に浸して涼しい場所で2時間寝かせる
\item
  水気を切り, 全体に塩をよく振る
\item
  オーブンを160度Cに熱して20分焼く

  \begin{itemize}
  \tightlist
  \item
    ここでは完全に火が通っている必要はない
  \end{itemize}
\item
  玉ねぎをみじん切りにする
\item
  キャラウェイシード, 生姜, ジュニパーベリーをすりつぶす
\item
  牛肉を鍋に入れ, ビネガー, ブイヨン, 玉ねぎ, 塩ひとつまみ,
  そして砂糖以外の残りのスパイスを全て入れる
\item
  肉が柔らかくなるまで, 1.5時間ほど煮込む
\item
  残った汁から固形物を濾し取る
\item
  さらに加熱して水気を減らし, 好みの濃さにする

  \begin{itemize}
  \tightlist
  \item
    もし酸味が強すぎるなら砂糖で調整する
  \end{itemize}
\item
  牛肉にソースをかける
\end{enumerate}

\hypertarget{ux88dcux8db3-51}{%
\subsection{補足}\label{ux88dcux8db3-51}}

\citet{feyfrlikova2015Kuchyne} pp.~80-81
にこのレシピの現代語訳が記載されている.
疫病に関する本に記載されていたレシピの1つであるという.

\hypertarget{ux53c2ux8003ux8cc7ux6599-51}{%
\subsection{参考資料}\label{ux53c2ux8003ux8cc7ux6599-51}}

\begin{itemize}
\tightlist
\item
  Random Innkeeper の動画
  \url{https://www.youtube.com/watch?v=QnSHjogWAMA}
\item
  \citet{feyfrlikova2015Kuchyne}
\end{itemize}

\hypertarget{ux30cfux30f3ux30acux30eaux30fcux98a8ux30edux30fcux30b9ux30c8ux30d3ux30fcux30d5-2-ux6377-16ux4e16ux7d00-hovux11bzie-po-uhersku}{%
\section{\texorpdfstring{「ハンガリー風」ローストビーフ \#2 (捷, 16世紀,
Hovězie po Uhersku)
\index{Hovězie po Uhersku|see{ハンガリー風ローストビーフ (16世紀)}}}{「ハンガリー風」ローストビーフ \#2 (捷, 16世紀, Hovězie po Uhersku) }}\label{ux30cfux30f3ux30acux30eaux30fcux98a8ux30edux30fcux30b9ux30c8ux30d3ux30fcux30d5-2-ux6377-16ux4e16ux7d00-hovux11bzie-po-uhersku}}

これもチェコの料理.
現代的なローストビーフとあまり変わらない味付けであり,
現代人にとってのハードルは低いと思われる (図\ref{fig:hungarian-beef16}).

\begin{figure}

{\centering \includegraphics[width=1\linewidth,height=1\textheight,keepaspectratio]{img/hungarian-beef16/finished} 

}

\caption{ハンガリー風ローストビーフ}\label{fig:hungarian-beef16}
\end{figure}

\begin{tabular}[t]{rl}
\toprule
 & 難易度\\
\midrule
材料調達 & {\fontspec{Noto Sans CJK JP} ★★★★☆ }\\
調理 & {\fontspec{Noto Sans CJK JP} ★★★☆☆ }\\
\bottomrule
\end{tabular}

\hypertarget{ux6750ux6599-37}{%
\subsection{材料}\label{ux6750ux6599-37}}

\begin{itemize}
\tightlist
\item
  牛肉, ブリスケットが良い
\item
  リンゴ 1個
\item
  玉ねぎ 1個
\item
  小麦粉
\item
  酢, リンゴ酢が良い
\item
  塩
\end{itemize}

\hypertarget{ux9053ux5177-38}{%
\subsection{道具}\label{ux9053ux5177-38}}

\begin{itemize}
\tightlist
\item
  オーブン
\item
  ストレイナー/裏ごし器
\end{itemize}

\hypertarget{ux4f5cux308aux65b9-56}{%
\subsection{作り方}\label{ux4f5cux308aux65b9-56}}

\begin{enumerate}
\def\labelenumi{\arabic{enumi}.}
\tightlist
\item
  玉ねぎとリンゴを荒く切り, オーブン用の鍋に入れる
\item
  小さな肉の切れ端や脂身を切り取って混ぜる
\item
  塊肉の全体によく塩を振ってこれらの上に乗せる
\item
  コップ半分の水を注ぎ, 蓋をする
\item
  オーブンを160度Cに熱して柔らかくなるまでローストする

  \begin{itemize}
  \tightlist
  \item
    この温度だと2時間程度かかる
  \end{itemize}
\item
  肉を取りだす
\item
  鍋に残ったものからソースを作る

  \begin{enumerate}
  \def\labelenumii{\arabic{enumii}.}
  \tightlist
  \item
    フライパンに注ぎ熱して水気を飛ばす
  \item
    酢をひとふりする
  \item
    小麦粉をふりかけ, とろみをつける
  \item
    水を加え, かき混ぜつつ適切な濃さになるまで熱する
  \item
    ソースを濾し取る
  \end{enumerate}
\item
  ローストビーフを切り分け, ソースをかけ, リンゴを添える
\end{enumerate}

\hypertarget{ux88dcux8db3-52}{%
\subsection{補足}\label{ux88dcux8db3-52}}

日本の家庭にはローストビーフを作れるオーブンがないことが多い.
オーブントースターや魚焼きグリルで代用すると,
熱源との距離が近すぎ焦げやすいため注意.
私は魚焼きグリルで少し加熱し後は圧力鍋を使ったが,
おそらくオーブンでローストしたほうがうまいだろう.

なぜリンゴを添えるのかはよくわからない.

\hypertarget{ux53c2ux8003ux8cc7ux6599-52}{%
\subsection{参考資料}\label{ux53c2ux8003ux8cc7ux6599-52}}

\begin{itemize}
\tightlist
\item
  Random Innkeeper の動画
  \url{https://www.youtube.com/watch?v=u5A8wbyBI9Y}
\end{itemize}

\hypertarget{ux30e9ux30e0ux8089ux306eux30bbux30fcux30b8ux30bdux30fcux30b9ux548cux3048-ux6377-15ux4e16ux7d00}{%
\section{ラム肉のセージソース和え (捷,
15世紀)}\label{ux30e9ux30e0ux8089ux306eux30bbux30fcux30b8ux30bdux30fcux30b9ux548cux3048-ux6377-15ux4e16ux7d00}}

当時の羊肉は富裕層でなくても食べる機会が多かったそうなので,
庶民の食事の1つだったのかもしれない.

TODO: 画像

\begin{tabular}[t]{rl}
\toprule
 & 難易度\\
\midrule
材料調達 & {\fontspec{Noto Sans CJK JP} ★★★★☆ }\\
調理 & {\fontspec{Noto Sans CJK JP} ★★☆☆☆ }\\
\bottomrule
\end{tabular}

\hypertarget{ux6750ux6599-38}{%
\subsection{材料}\label{ux6750ux6599-38}}

\begin{itemize}
\tightlist
\item
  羊肉 (ラムのショルダーがよい) 600 g
\item
  塩
\item
  胡椒
\item
  (オプション) メース
\item
  ソース

  \begin{itemize}
  \tightlist
  \item
    ブイヨン 200 ml

    \begin{itemize}
    \tightlist
    \item
      羊の茹で汁を使いまわしても良い
    \end{itemize}
  \item
    酢, リンゴ酢がよい 大さじ2
  \item
    砂糖 大さじ2
  \item
    セージの葉 大さじ2
  \end{itemize}
\end{itemize}

\hypertarget{ux9053ux5177-39}{%
\subsection{道具}\label{ux9053ux5177-39}}

\begin{itemize}
\tightlist
\item
  オーブン
\end{itemize}

\hypertarget{ux4f5cux308aux65b9-57}{%
\subsection{作り方}\label{ux4f5cux308aux65b9-57}}

\begin{enumerate}
\def\labelenumi{\arabic{enumi}.}
\tightlist
\item
  鍋に水をはり, 塩を加える
\item
  肉をよく茹でる
\item
  ソースを作る

  \begin{enumerate}
  \def\labelenumii{\arabic{enumii}.}
  \tightlist
  \item
    別の小さい鍋を弱火で熱し, ブイヨンを注ぐ
  \item
    酢と砂糖を加える
  \item
    セージを細かく刻んで加える
  \item
    よく混ぜてなじませる
  \item
    およそ10分ほど加熱を続ける

    \begin{itemize}
    \tightlist
    \item
      気になるなら葉を濾し取る
    \end{itemize}
  \end{enumerate}
\item
  肉に塩胡椒とメースをふりかける
\item
  肉が柔らかくなるまで焼く
\item
  オーブンを200度に熱し, 肉が柔らかくなるまでローストする

  \begin{itemize}
  \tightlist
  \item
    おそらく45分程度
  \end{itemize}
\item
  肉を切り分け, ソースをかける
\end{enumerate}

\hypertarget{ux88dcux8db3-53}{%
\subsection{補足}\label{ux88dcux8db3-53}}

メースは当時最も高価な部類に入るスパイスであるので,
ロールプレイがしたい人は注意. なお \citet{feyfrlikova2015Kuchyne}
によれば当時の一般的な都市の住居にはすでにオーブンが備わっていたため\footnote{これはビデオゲーム
  ``Kingdom Come: Deliverance'' でも視覚的に再現されている},
こちらを気にする必要はないと思われる.

\hypertarget{ux53c2ux8003ux8cc7ux6599-53}{%
\subsection{参考資料}\label{ux53c2ux8003ux8cc7ux6599-53}}

\begin{itemize}
\tightlist
\item
  Random Innkeeper の動画
  \url{https://www.youtube.com/watch?v=BsYVAQfuWqc}
\end{itemize}

\hypertarget{ux9e7fux8089ux98a8ux725bux8089-ux6377-15ux4e16ux7d00}{%
\section{「鹿肉風」牛肉 (捷,
15世紀)}\label{ux9e7fux8089ux98a8ux725bux8089-ux6377-15ux4e16ux7d00}}

牛肉をワインに浸けて煮込み, 鹿肉のように赤黒くする料理.
ジビエを食べられるのは森を所有していた貴族たちだったので,
狩猟権のない富裕層が牛肉を楽しむために作られたレシピなのかもしれない.

\begin{figure}

{\centering \includegraphics[width=1\linewidth,height=1\textheight,keepaspectratio]{img/venison-beef/finished} 

}

\caption{鹿肉風牛肉}\label{fig:venison-beef-finished}
\end{figure}

\begin{tabular}[t]{rl}
\toprule
 & 難易度\\
\midrule
材料調達 & {\fontspec{Noto Sans CJK JP} ★★★☆☆ }\\
調理 & {\fontspec{Noto Sans CJK JP} ★★★☆☆ }\\
\bottomrule
\end{tabular}

\hypertarget{ux6750ux6599-39}{%
\subsection{材料}\label{ux6750ux6599-39}}

\begin{itemize}
\tightlist
\item
  牛肉
\item
  塩
\item
  赤ワイン 十分な量
\item
  クローブ
\item
  胡椒粒
\item
  リンゴ
\item
  サフラン
\item
  バター 大さじ1
\item
  砂糖 大さじ1
\end{itemize}

\hypertarget{ux4f5cux308aux65b9-58}{%
\subsection{作り方}\label{ux4f5cux308aux65b9-58}}

\begin{enumerate}
\def\labelenumi{\arabic{enumi}.}
\tightlist
\item
  牛肉を軽く叩いて柔らかくする
\item
  牛肉を鍋に入れ, 赤ワインに浸し, 涼し場所で最低2時間寝かせる
\item
  弱火でゆっくり加熱し, 塩とクローブと胡椒とサフランを加える
\item
  1時間ほどかけて肉が柔らかくなるまで煮る
\item
  肉が煮えてきたらフライパンを温めバターを溶かす.
\item
  砂糖を加えてスライスしたリンゴの量を焼く
\item
  焼いたリンゴの半分を細かく切り刻む
\item
  皿の中央に刻んだリンゴを乗せる
\item
  その周りに茹でた牛肉と残りのリンゴを並べる
\item
  茹で汁をソースとしてかける

  \begin{itemize}
  \tightlist
  \item
    ソースが濃くなるように適宜蒸発させる
  \end{itemize}
\end{enumerate}

\hypertarget{ux88dcux8db3-54}{%
\subsection{補足}\label{ux88dcux8db3-54}}

赤ワインは蒸発しやすいので蓋をするなどして対処する.

\hypertarget{ux53c2ux8003ux8cc7ux6599-54}{%
\subsection{参考資料}\label{ux53c2ux8003ux8cc7ux6599-54}}

\begin{itemize}
\tightlist
\item
  Random Innkeeper の動画
  \url{https://www.youtube.com/watch?v=xZe7p8ofCfg}
\end{itemize}

\hypertarget{ux4e88ux5b9a-ux30edux30fcux30b9ux30c8ux30c0ux30c3ux30af-ux6377-15ux4e16ux7d00}{%
\section{(予定) ローストダック (捷,
15世紀)}\label{ux4e88ux5b9a-ux30edux30fcux30b9ux30c8ux30c0ux30c3ux30af-ux6377-15ux4e16ux7d00}}

\begin{figure}

{\centering \includegraphics[width=1\linewidth,height=1\textheight,keepaspectratio]{img/roast-duck-cz/finished} 

}

\caption{中世風}\label{fig:roast-duck-cz-finished}
\end{figure}

\begin{figure}

{\centering \includegraphics[width=1\linewidth,height=1\textheight,keepaspectratio]{img/roast-duck-cz/moravsky} 

}

\caption{現代モラヴィア風}\label{fig:roast-duck-cz-moravia}
\end{figure}

\hypertarget{ux53c2ux8003ux8cc7ux6599-55}{%
\subsection{参考資料}\label{ux53c2ux8003ux8cc7ux6599-55}}

\begin{itemize}
\tightlist
\item
  \url{https://www.youtube.com/watch?v=mtta0W-kBh4}
\end{itemize}

\hypertarget{ux91ceux30a6ux30b5ux30aeux306eux716eux8fbcux307f-ux6377-15ux4e16ux7d00}{%
\section{野ウサギの煮込み (捷,
15世紀)}\label{ux91ceux30a6ux30b5ux30aeux306eux716eux8fbcux307f-ux6377-15ux4e16ux7d00}}

現代のチェコでもウサギ料理はそれなりにポピュラーである.

\begin{figure}

{\centering \includegraphics[width=1\linewidth,height=1\textheight,keepaspectratio]{img/kralik-medieval//finished} 

}

\caption{穴ウサギでもいい}\label{fig:hare-finished}
\end{figure}

\begin{tabular}[t]{rl}
\toprule
 & 難易度\\
\midrule
材料調達 & {\fontspec{Noto Sans CJK JP} ★★★★★ }\\
調理 & {\fontspec{Noto Sans CJK JP} ★★★☆☆ }\\
\bottomrule
\end{tabular}

\hypertarget{ux6750ux6599-40}{%
\subsection{材料}\label{ux6750ux6599-40}}

\begin{itemize}
\tightlist
\item
  ウサギのモモ 2本
\item
  赤ワイン 150 ml
\item
  ブイヨン (無塩がよい) 800 ml
\item
  タマネギ 2個
\item
  リンゴ 2個
\item
  パン粉
\item
  胡椒
\item
  生姜
\item
  クローブ
\item
  サフラン
\end{itemize}

\hypertarget{ux4f5cux308aux65b9-59}{%
\subsection{作り方}\label{ux4f5cux308aux65b9-59}}

参照動画を参考にせよ

\hypertarget{ux53c2ux8003ux8cc7ux6599-56}{%
\subsection{参考資料}\label{ux53c2ux8003ux8cc7ux6599-56}}

\begin{itemize}
\tightlist
\item
  Random Innkeeper の動画 (日本語字幕あり)
  \url{https://www.youtube.com/watch?v=CcjKye3use8}
\end{itemize}

\hypertarget{kcd-ux306eux30b7ux30c1ux30e5ux30fc}{%
\section{KCD のシチュー}\label{kcd-ux306eux30b7ux30c1ux30e5ux30fc}}

\begin{figure}

{\centering \includegraphics[width=1\linewidth,height=1\textheight,keepaspectratio]{img/KCD-stew/finished} 

}

\caption{15世紀にグラーシュはない}\label{fig:kcd-stew}
\end{figure}

\begin{tabular}[t]{rl}
\toprule
 & 難易度\\
\midrule
材料調達 & {\fontspec{Noto Sans CJK JP} ★★★★☆ }\\
調理 & {\fontspec{Noto Sans CJK JP} ★★★☆☆ }\\
\bottomrule
\end{tabular}

ゲーム \href{}{Kingdom Come: Deliverance} に登場する謎のシチュー.
チェコの伝統料理グラーシュに似ているが,
現代のグラーシュはトマトやパプリカを使うため15世紀には存在しなかった.

これは正確には当時の文献にあったものではないが,
当時の文献に着想を得たものであるという.

\hypertarget{ux6750ux6599-41}{%
\subsection{材料}\label{ux6750ux6599-41}}

\begin{itemize}
\tightlist
\item
  牛ブリスケット

  \begin{itemize}
  \tightlist
  \item
    なければ煮込み用のスネ肉でもよい
  \end{itemize}
\item
  根パセリ
\item
  ニンジン
\item
  セロリアック
\item
  玉ねぎ
\item
  赤ワイン
\item
  白パン
\item
  ラード
\item
  塩
\item
  ブイヨン
\item
  リンゴ

  \begin{itemize}
  \tightlist
  \item
    リンゴ酢を使っても良い
  \end{itemize}
\item
  メース
\item
  ショウガ

  \begin{itemize}
  \tightlist
  \item
    乾燥粉末がなければすりおろす
  \end{itemize}
\item
  胡椒
\item
  (オプション) ラベージの葉
\end{itemize}

\hypertarget{ux4f5cux308aux65b9-60}{%
\subsection{作り方}\label{ux4f5cux308aux65b9-60}}

\begin{enumerate}
\def\labelenumi{\arabic{enumi}.}
\tightlist
\item
  根菜と林檎全てみじん切りにする
\item
  牛肉を 2cm 大に切る
\item
  鍋にラードを溶かし, 牛肉に焼き目を付ける
\item
  表面がよく焦げ茶色になったら取り出し, 代わりに玉ねぎ以外の根菜を炒める
\item
  火が通ってきたら玉ねぎと林檎も入れる
\item
  軽く焦げ付く (カラメル化する) まで弱火でじっくり炒める.
\item
  ショウガ, メース. 胡椒, 塩を加えてよく混ぜる
\item
  ワインを注ぎ, 蒸発するまで混ぜる
\item
  水気が飛んだら肉を再度入れる
\item
  肉が水没する程度までブイヨンを注ぐ
\item
  蓋をして2時間ほど煮込む

  \begin{itemize}
  \tightlist
  \item
    肉が柔らかくなるまで
  \end{itemize}
\item
  この間にフライパンでパンを乾煎りする

  \begin{itemize}
  \tightlist
  \item
    たぶん普通にトースターでもやってもよい
  \end{itemize}
\item
  鍋にラベージの葉を入れる
\item
  鍋にパンを崩して入れる
\end{enumerate}

\hypertarget{ux88dcux8db3-55}{%
\subsection{補足}\label{ux88dcux8db3-55}}

参考動画では塩ひとつまみ (a pinch of \ldots) と言っているが,
実際には小さじ2杯かそれより多いくらいがちょうどよい

日本では根パセリとセロリアックが入手しづらい.
セロリアックはセロリの茎でも良いかもしれないが,
根パセリは代わりのきく食材が思いつかない.
パースニップは見た目こそ似ているが風味は全く異なる.

\hypertarget{ux53c2ux8003ux8cc7ux6599-57}{%
\subsection{参考資料}\label{ux53c2ux8003ux8cc7ux6599-57}}

\begin{itemize}
\tightlist
\item
  Random Innkeeper の動画 (日本語字幕あり)
  \url{https://www.youtube.com/watch?v=zqc4iQrE-FQ}
\end{itemize}

\hypertarget{ux30cfux30c3ux30b7ux30e5ux30c9ux30ecux30d0ux30fc-ux6377-15ux4e16ux7d00}{%
\section{ハッシュドレバー (捷,
15世紀)}\label{ux30cfux30c3ux30b7ux30e5ux30c9ux30ecux30d0ux30fc-ux6377-15ux4e16ux7d00}}

「見た目はひどいが, 味は良い」レバーの臭みが苦手な人におすすめ.
多少見た目を改善したレシピも紹介する

\begin{figure}

{\centering \includegraphics[width=1\linewidth,height=1\textheight,keepaspectratio]{img/liver-hash/finished} 

}

\caption{見た目に難がある}\label{fig:finished-liver-hash}
\end{figure}

\begin{tabular}[t]{rl}
\toprule
 & 難易度\\
\midrule
材料調達 & {\fontspec{Noto Sans CJK JP} ★★★★☆ }\\
調理 & {\fontspec{Noto Sans CJK JP} ★★☆☆☆ }\\
\bottomrule
\end{tabular}

\hypertarget{ux6750ux6599-42}{%
\subsection{材料}\label{ux6750ux6599-42}}

\begin{itemize}
\item
  レバー (豚か牛) 400 g
\item
  赤ワイン 200 ml
\item
  卵 1個
\item
  塩
\item
  胡椒
\item
  アニスシード
\item
  シナモン粉末
\item
  メース
\item
  ショウガ
\item
  キャラウェイシード
\item
  (オプション) パン粉
\item
  (オプション) パテの材料

  \begin{itemize}
  \tightlist
  \item
    レーズン 50 g
  \item
    ラード 100-200 g
  \item
    付け合せにタイム、チャイブなど緑色のハーブ
  \end{itemize}
\end{itemize}

\hypertarget{ux9053ux5177-40}{%
\subsection{道具}\label{ux9053ux5177-40}}

\begin{itemize}
\tightlist
\item
  パテにしたい場合は肉挽き器が必要
\end{itemize}

\hypertarget{ux4f5cux308aux65b9-61}{%
\subsection{作り方}\label{ux4f5cux308aux65b9-61}}

\begin{enumerate}
\def\labelenumi{\arabic{enumi}.}
\tightlist
\item
  レバーを小さく切る
\item
  茹でるために湯を沸かす

  \begin{itemize}
  \tightlist
  \item
    無塩ブイヨンを使っても良い
  \item
    ブイヨンに使うような野菜やハーブを入れても良い
  \end{itemize}
\item
  レバーを茹で, アクを随時取り除く
\item
  レバーに火が通ったら, 包丁で細かく刻む
\item
  フライパンを加熱し, 赤ワインを注ぐ
\item
  胡椒, シナモン, メース, 生姜を加える
\item
  レバーと塩も加える
\item
  好みでパン粉を加え, よく混ぜる
\item
  キャラウェイシードとアニスシードをすりつぶし, 加える
\item
  火を消し, 卵を加えてよく混ぜ, 余熱で調理する
\end{enumerate}

\hypertarget{ux88dcux8db3-56}{%
\subsection{補足}\label{ux88dcux8db3-56}}

上記は本来のレシピだが, 見た目がとても悪い. ため,
改善のためパテにする方法も提案されている.
そのためにはレーズンとともに肉挽き器で細かく挽いた後,
ラードを加えてよくこねる. パンに塗り, ハーブをふりかけて食べる.

\hypertarget{ux53c2ux8003ux8cc7ux6599-58}{%
\subsection{参考資料}\label{ux53c2ux8003ux8cc7ux6599-58}}

\begin{itemize}
\tightlist
\item
  Random Innkeeper の動画
  \url{https://www.youtube.com/watch?v=w5oWJZVPRK4}
\end{itemize}

\hypertarget{ux30e1ux30ebux30ebux30fcux30b5ux306eux30d9ux30fcux30b3ux30f3ux716e-ux6377-15ux4e16ux7d00-ux161tiku-s-slaninami}{%
\section{メルルーサのベーコン煮 (捷, 15世紀: Štiku s
slaninami)}\label{ux30e1ux30ebux30ebux30fcux30b5ux306eux30d9ux30fcux30b3ux30f3ux716e-ux6377-15ux4e16ux7d00-ux161tiku-s-slaninami}}

イギリス由来の料理のため, 海の魚を使用している.
メルルーサを使うのが本来のレシピだが,
今回は入手できなかったためスケトウダラでアレンジした
(図\ref{fig:stiku-s-slaninami-finished}).

\begin{figure}

{\centering \includegraphics[width=1\linewidth,height=1\textheight,keepaspectratio]{img/stiku-s-slaninami/finished} 

}

\caption{タラのベーコン煮}\label{fig:stiku-s-slaninami-finished}
\end{figure}

\begin{tabular}[t]{rl}
\toprule
 & 難易度\\
\midrule
材料調達 & {\fontspec{Noto Sans CJK JP} ★★★★☆ }\\
調理 & {\fontspec{Noto Sans CJK JP} ★★☆☆☆ }\\
\bottomrule
\end{tabular}

\hypertarget{ux6750ux6599-43}{%
\subsection{材料}\label{ux6750ux6599-43}}

\begin{itemize}
\tightlist
\item
  スケトウダラ

  \begin{itemize}
  \tightlist
  \item
    中世ヨーロッパで獲れそうな海魚
  \end{itemize}
\item
  パースニップ

  \begin{itemize}
  \tightlist
  \item
    なければ人参で代用
  \item
    もしあれば根パセリという手もある
  \end{itemize}
\item
  胡椒
\item
  生姜
\item
  ベーコン
\end{itemize}

\hypertarget{ux4f5cux308aux65b9-62}{%
\subsection{作り方}\label{ux4f5cux308aux65b9-62}}

\begin{enumerate}
\def\labelenumi{\arabic{enumi}.}
\tightlist
\item
  パースニップを拍子木切りにする
\item
  ベーコンを小さく角切りにする
\item
  胡椒と生姜を細かく挽く
\item
  鍋に水を注ぎ, パースニップ, 魚, ベーコンの順に入れ,
  その上から胡椒と生姜をふりかける
\item
  蓋をして10分ほど弱火で蒸す
\end{enumerate}

\hypertarget{ux88dcux8db3-57}{%
\subsection{補足}\label{ux88dcux8db3-57}}

元のレシピでは全ての材料を一度に入れるとしか書いてないらしい.
パースニップを下敷きにするのは Random Innkeeper のアイディアである.

なお, 日本国内の市販ベーコンは味が薄く脂身も少ないため,
塩を余分に加えたほうが良いかもしれない.

\citet{Ju2018} p.~23 でも中世イギリスのタラ料理が言及されている.

\hypertarget{ux53c2ux8003ux8cc7ux6599-59}{%
\subsection{参考資料}\label{ux53c2ux8003ux8cc7ux6599-59}}

\begin{itemize}
\tightlist
\item
  Random Innkeeper \url{https://www.youtube.com/watch?v=pQthdIYfRR0}
\item
  \citet{Ju2018}
\end{itemize}

\hypertarget{ux30c8ux30a5ux30d5ux30fc-ux30a2ux30c3ux30abux30c9-ux30d0ux30d3ux30edux30f3ux7b2c1ux738bux671dux6642ux4ee3-ux12305ux12314ux12311}{%
\section{\texorpdfstring{トゥ・フー (アッカド, バビロン第1王朝時代?
{\fontspec{Akkadian}𒌅𒌔𒌑})}{トゥ・フー (アッカド, バビロン第1王朝時代? 𒌅𒌔𒌑)}}\label{ux30c8ux30a5ux30d5ux30fc-ux30a2ux30c3ux30abux30c9-ux30d0ux30d3ux30edux30f3ux7b2c1ux738bux671dux6642ux4ee3-ux12305ux12314ux12311}}

古代メソポタミア料理. 材料調達の難易度がかなり高い.

\begin{figure}

{\centering \includegraphics[width=1\linewidth,height=1\textheight,keepaspectratio]{img/mesopotamia/finished} 

}

\caption{トゥ・フー}\label{fig:mesopotamia-finished}
\end{figure}

\begin{tabular}[t]{rl}
\toprule
 & 難易度\\
\midrule
材料調達 & {\fontspec{Noto Sans CJK JP} ★★★★★★★★★★ }\\
調理 & {\fontspec{Noto Sans CJK JP} ★★★☆☆ }\\
\bottomrule
\end{tabular}

材料と作り方は参考資料を参照せよ

\hypertarget{ux53c2ux8003ux8cc7ux6599-60}{%
\subsection{参考資料}\label{ux53c2ux8003ux8cc7ux6599-60}}

\begin{itemize}
\tightlist
\item
  \citet{Damerow2012SumerianBT}
\item
  \citet{GojkoEtAl2019-blog}
\item
  \citet{GojkoEtAl2019}
\item
  これらを元にした自作記録
  \url{https://under-identified.hatenablog.com/entry/2020/08/01/213350}
\end{itemize}

\hypertarget{ux305dux306eux4ed6ux306eux52a0ux5de5ux98dfux54c1}{%
\chapter{その他の加工食品}\label{ux305dux306eux4ed6ux306eux52a0ux5de5ux98dfux54c1}}

単体で付け合せにできるし, 他の料理に使うこともできる.

\hypertarget{ux30c8ux30deux30c8ux30bdux30fcux30b9-ux4f0a}{%
\section{トマトソース
(伊)}\label{ux30c8ux30deux30c8ux30bdux30fcux30b9-ux4f0a}}

\begin{tabular}[t]{rl}
\toprule
 & 難易度\\
\midrule
材料調達 & {\fontspec{Noto Sans CJK JP} ★★☆☆☆ }\\
調理 & {\fontspec{Noto Sans CJK JP} ★★☆☆☆ }\\
\bottomrule
\end{tabular}

\hypertarget{ux6750ux6599-44}{%
\subsection{材料}\label{ux6750ux6599-44}}

\begin{itemize}
\tightlist
\item
  ホールトマト缶 400 ml
\item
  ニンニク 1 欠片
\item
  玉ねぎ 半分
\item
  月桂樹の葉
\item
  パセリ
\item
  オリーブオイル
\end{itemize}

\hypertarget{ux4f5cux308aux65b9-63}{%
\subsection{作り方}\label{ux4f5cux308aux65b9-63}}

\begin{enumerate}
\def\labelenumi{\arabic{enumi}.}
\tightlist
\item
  フライパンにオリーブオイルを引き, 弱火で加熱する
\item
  ニンニクと玉ねぎを炒める
\item
  ホールトマト缶を入れ, 煮崩れるまで加熱を続ける
\item
  水分が飛び, 濃厚になるまで続ける
\item
  火を止める直前に月桂樹の葉やパセリを入れる
\end{enumerate}

\hypertarget{ux30b6ux30efux30fcux30afux30e9ux30a6ux30c8-ux72ec-sauerkraut}{%
\section{\texorpdfstring{ザワークラウト\index{ザワークラウト} (独:
Sauerkraut)\index{sauerkraut|see{ザワークラウト}}}{ザワークラウト (独: Sauerkraut)}}\label{ux30b6ux30efux30fcux30afux30e9ux30a6ux30c8-ux72ec-sauerkraut}}

ドイツや東欧で広く食べられる. ボルシチやシチーの材料にもなる.
よく「酢漬け」と間違えられがちだが, 酸味は乳酸発酵によるもの.
ただし市販品にはビネガーやワインで味付けしたものもある.

\begin{tabular}[t]{rl}
\toprule
 & 難易度\\
\midrule
材料調達 & {\fontspec{Noto Sans CJK JP} ★☆☆☆☆ }\\
調理 & {\fontspec{Noto Sans CJK JP} ★★★☆☆ }\\
\bottomrule
\end{tabular}

\begin{infobox}{important}
発酵・熟成食品は衛生環境に注意して作成してください

\end{infobox}

\hypertarget{ux6750ux6599-45}{%
\subsection{材料}\label{ux6750ux6599-45}}

\begin{itemize}
\tightlist
\item
  キャベツ 1玉 (1.2-1.5 kg程度)

  \begin{itemize}
  \tightlist
  \item
    身の詰まった重いものがよい
  \end{itemize}
\item
  塩 大さじ 2杯

  \begin{itemize}
  \tightlist
  \item
    キャベツ 500 g あたり塩 10g が相場
  \end{itemize}
\item
  (オプション) 人参
\item
  (オプション) ハーブ・スパイス類

  \begin{itemize}
  \tightlist
  \item
    胡椒, オールスパイス, ローレル, ディルなど
  \item
    白ワイン
  \end{itemize}
\end{itemize}

\hypertarget{ux9053ux5177-41}{%
\subsection{道具}\label{ux9053ux5177-41}}

\begin{itemize}
\tightlist
\item
  漬け物容器

  \begin{itemize}
  \tightlist
  \item
    押し付ける機能のあるものが良い
  \end{itemize}
\end{itemize}

\hypertarget{ux4f5cux308aux65b9-64}{%
\subsection{作り方}\label{ux4f5cux308aux65b9-64}}

\begin{enumerate}
\def\labelenumi{\arabic{enumi}.}
\tightlist
\item
  キャベツを半分に割る
\item
  V字の切り込みを入れ, 芯を切り取る
\item
  太い葉脈も適宜切り取る
\item
  キャベツをさらに半分に割ってから千切りにする
\item
  塩でもむ
\item
  しんなりしてきたら漬け物容器に入れる

  \begin{itemize}
  \tightlist
  \item
    最初から漬け物容器内でやってもよい
  \end{itemize}
\item
  香り付けにハーブ・スパイス類や白ワインを少し混ぜても良い
\item
  冷蔵庫に保管する

  \begin{itemize}
  \tightlist
  \item
    冬なら常温放置でも可
  \end{itemize}
\item
  一晩すると水分が抜けてくる
\item
  キャベツ全体が水に完全に浸かるように上から押す力を調整する
\item
  2週間-1月程度待つ
\item
  全体的に白くなり, 発酵臭がしたら食べごろ

  \begin{itemize}
  \tightlist
  \item
    刺激臭がしたり黒ずんだり, カビが生えたりしたら捨てる
  \item
    常温放置すると表面に酵母が発生することがある
  \end{itemize}
\end{enumerate}

\hypertarget{ux88dcux8db3-58}{%
\subsection{補足}\label{ux88dcux8db3-58}}

夏場は管理が難しいので秋や冬にやるとよい.
ロシアやポーランドでは人参やディルを混ぜることがよくあるらしい.
ドイツやチェコではあまり混ぜない.

\hypertarget{ux53c2ux8003ux8cc7ux6599-61}{%
\subsection{参考資料}\label{ux53c2ux8003ux8cc7ux6599-61}}

\begin{itemize}
\tightlist
\item
  \citet{katz2012Art}
\item
  日本語の解説付きの作成例
  \url{https://www.youtube.com/watch?v=uTgx3Gcixcc}
\end{itemize}

\hypertarget{ux30a6ux30c8ux30daux30cdux30c4-ux6377-utopenci}{%
\section{\texorpdfstring{ウトペネツ\index{ウトペネツ} (捷:
Utopenci)\index{utopenec|see{ウトペネツ}}\index{utopenci|see{ウトペネツ}}}{ウトペネツ (捷: Utopenci)}}\label{ux30a6ux30c8ux30daux30cdux30c4-ux6377-utopenci}}

ソーセージと野菜を乳酸発酵させたチェコの伝統食品
(図\ref{fig:utopenci-finished}).

\begin{figure}

{\centering \includegraphics[width=1\linewidth,height=1\textheight,keepaspectratio]{img/utopenci/finished} 

}

\caption{瓶に詰めたウトペネツ}\label{fig:utopenci-finished}
\end{figure}

\begin{tabular}[t]{rl}
\toprule
 & 難易度\\
\midrule
材料調達 & {\fontspec{Noto Sans CJK JP} ★★★☆☆ }\\
調理 & {\fontspec{Noto Sans CJK JP} ★★☆☆☆ }\\
\bottomrule
\end{tabular}

\begin{infobox}{important}
発酵・熟成食品は衛生環境に注意して作成してください

\end{infobox}

\hypertarget{ux6750ux6599-46}{%
\subsection{材料}\label{ux6750ux6599-46}}

\begin{itemize}
\tightlist
\item
  ベーコンやソーセージ
\item
  パプリカ
\item
  玉ねぎ
\item
  唐辛子
\item
  ニンニク
\item
  オールスパイス
\item
  胡椒粒
\item
  キャラウェイシード
\item
  月桂樹の葉
\item
  砂糖大さじ1
\item
  酢 500 cc
\item
  塩 大さじ1
\item
  水 750 cc
\end{itemize}

\hypertarget{ux9053ux5177-42}{%
\subsection{道具}\label{ux9053ux5177-42}}

\begin{itemize}
\tightlist
\item
  密閉できる清潔な瓶
\end{itemize}

\hypertarget{ux4f5cux308aux65b9-65}{%
\subsection{作り方}\label{ux4f5cux308aux65b9-65}}

\begin{enumerate}
\def\labelenumi{\arabic{enumi}.}
\item
  鍋に水, 塩, 砂糖, オールスパイス, 胡椒, 月桂樹の葉を入れ沸騰させる
\item
  ソーセージの皮をむき, 割る. または輪切りにする

  \begin{itemize}
  \tightlist
  \item
    日本の市販のソーセージは小さいので半分に割るくらいがよいと思われる
  \end{itemize}
\item
  瓶にスライスしたパプリカと玉ねぎ, そして残りの材料を詰め込む
\item
  \begin{enumerate}
  \def\labelenumii{(\arabic{enumii})}
  \tightlist
  \item
    の熱湯の粗熱が取れたら瓶に注ぎ, 蓋をする
  \end{enumerate}
\item
  冷暗所で乳酸発酵するまで寝かせる

  \begin{itemize}
  \tightlist
  \item
    1-2週間程度
  \end{itemize}
\end{enumerate}

\hypertarget{ux88dcux8db3-59}{%
\subsection{補足}\label{ux88dcux8db3-59}}

\aruby{utpenci}{ウトペンツィ} は複数形で, \aruby{utpenec}{ウトペネツ}
は単数形. 意味は 「溺死体」

現地では špekáček というソーセージを使用することが多いらしい.
参考動画でも大きめのソーセージを4本に割っている

図\ref{fig:utopenci-finished}
は玉ねぎやパプリカを多く入れているが「溺死体」という名前が示すように,
本来は肉がメインである.

ピックル液の量は材料の比率を維持しつつ瓶のサイズに合わせて調整する.

\hypertarget{ux53c2ux8003ux8cc7ux6599-62}{%
\subsection{参考資料}\label{ux53c2ux8003ux8cc7ux6599-62}}

\begin{itemize}
\tightlist
\item
  動画: \url{https://www.youtube.com/watch?v=niRxxO8NSHM}
\item
  古典的なレシピ
  \url{https://www.toprecepty.cz/recept/39156-utopenci-klasika/}
\end{itemize}

\hypertarget{todo-ux30adux30ceux30b3ux306eux5869ux6f2cux3051-ux9732}{%
\section{(TODO) キノコの塩漬け
(露)}\label{todo-ux30adux30ceux30b3ux306eux5869ux6f2cux3051-ux9732}}

\begin{tabular}[t]{rl}
\toprule
 & 難易度\\
\midrule
材料調達 & {\fontspec{Noto Sans CJK JP} ★★★☆☆ }\\
調理 & {\fontspec{Noto Sans CJK JP} ★★★☆☆ }\\
\bottomrule
\end{tabular}

\hypertarget{todo-ux30c8ux30deux30c8ux306eux5869ux6f2cux3051-ux9732}{%
\section{(TODO) トマトの塩漬け
(露)}\label{todo-ux30c8ux30deux30c8ux306eux5869ux6f2cux3051-ux9732}}

\begin{tabular}[t]{rl}
\toprule
 & 難易度\\
\midrule
材料調達 & {\fontspec{Noto Sans CJK JP} ★★☆☆☆ }\\
調理 & {\fontspec{Noto Sans CJK JP} ★★★☆☆ }\\
\bottomrule
\end{tabular}

\hypertarget{todo-ux30adux30e5ux30a6ux30eaux306eux5869ux6f2cux3051-ux9732}{%
\section{(TODO) キュウリの塩漬け
(露)}\label{todo-ux30adux30e5ux30a6ux30eaux306eux5869ux6f2cux3051-ux9732}}

ピクルスと混同されるとおそらくロシア人は不満に思うことだろう.
ウォッカのつまみにもよい.

\begin{tabular}[t]{rl}
\toprule
 & 難易度\\
\midrule
材料調達 & {\fontspec{Noto Sans CJK JP} ★★☆☆☆ }\\
調理 & {\fontspec{Noto Sans CJK JP} ★★★☆☆ }\\
\bottomrule
\end{tabular}

\hypertarget{ux30d9ux30fcux30b3ux30f3bacon-ux72ec-spek}{%
\section{\texorpdfstring{ベーコン\index{ベーコン}(bacon\index{bacon|see{ベーコン}};
独:
spek\index{spek|see{ベーコン}})}{ベーコン(bacon; 独: spek)}}\label{ux30d9ux30fcux30b3ux30f3bacon-ux72ec-spek}}

塩とスパイスをまぶして熟成させる乾式と, ソミュール液に漬ける湿式がある.
湿式のほうが簡単.

\begin{figure}

{\centering \includegraphics[width=1\linewidth,height=1\textheight,keepaspectratio]{img/bacon/finished} 

}

\caption{ベーコン}\label{fig:finished-bacon}
\end{figure}

\begin{tabular}[t]{rl}
\toprule
 & 難易度\\
\midrule
材料調達 & {\fontspec{Noto Sans CJK JP} ★★★★☆ }\\
調理 & {\fontspec{Noto Sans CJK JP} ★★★★★ }\\
\bottomrule
\end{tabular}

\begin{infobox}{important}
発酵・熟成食品は衛生環境に注意して作成してください

\end{infobox}

\hypertarget{ux6750ux6599-47}{%
\subsection{材料}\label{ux6750ux6599-47}}

\begin{itemize}
\tightlist
\item
  豚バラ肉
\item
  塩
\item
  その他好きなハーブ・スパイス

  \begin{itemize}
  \tightlist
  \item
    胡椒, オールスパイス, タイム, ローズマリー, ニンニクなどが合う
  \end{itemize}
\end{itemize}

\hypertarget{ux9053ux5177-43}{%
\subsection{道具}\label{ux9053ux5177-43}}

\begin{itemize}
\tightlist
\item
  豚バラ肉の入る, 密閉できる袋 (湿式の場合)
\item
  キッチンペーパーまたはピチットシート (乾式の場合)
\item
  スモークウッド
\item
  大きめのダンボール箱
\item
  燃えない丈夫な受け皿
\item
  タコ紐
\end{itemize}

\hypertarget{ux4f5cux308aux65b9-66}{%
\subsection{作り方}\label{ux4f5cux308aux65b9-66}}

\begin{enumerate}
\def\labelenumi{\arabic{enumi}.}
\tightlist
\item
  肉に下味をつける

  \begin{itemize}
  \tightlist
  \item
    バラ肉に十分な塩とスパイスをすりつける
  \item
    または十分な濃度の塩水 (ピックル液/ソミュール液) に漬ける

    \begin{itemize}
    \tightlist
    \item
      15-20\%程度の濃度の塩水に好みのハーブ・スパイスを入れて一度沸騰させた液
    \end{itemize}
  \end{itemize}
\item
  2週間-1ヶ月ほど冷蔵庫で熟成させる

  \begin{itemize}
  \tightlist
  \item
    乾式の場合, 表面に染み出してくる水分をまめに取り除く
  \item
    ピチットシートなら説明書に従い,
    キッチンペーパーならなるべく1日ごとに取り替える
  \end{itemize}
\item
  軽く水洗いし, 表面についたスパイスを洗い流す
\item
  表面の水分を拭き取る
\item
  さらに天日干しで表面を乾燥させる

  \begin{itemize}
  \tightlist
  \item
    または扇風機で1-2時間風を当て続ける
  \item
    燻製全般に言えることだが,
    表面に水分が残っているとホルムアルデヒドが発生し致命的に味が悪くなる
  \end{itemize}
\item
  6時間ほど燻製する

  \begin{itemize}
  \tightlist
  \item
    40-60度程度を維持すると良い. これは温燻と呼ばれる処理.
  \end{itemize}
\item
  1日冷蔵庫で寝かせる
\end{enumerate}

\hypertarget{ux88dcux8db3-60}{%
\subsection{補足}\label{ux88dcux8db3-60}}

比較的安価な方法は, 段ボールで燻製室を作り,
継続的な加熱の必要ない「スモークウッド」で燻製すること.
しかし火災のリスクがまったくないわけではないため, 目を離さないように.

\hypertarget{ux53c2ux8003ux8cc7ux6599-63}{%
\subsection{参考資料}\label{ux53c2ux8003ux8cc7ux6599-63}}

\begin{itemize}
\tightlist
\item
  乾式チロリアン風ベーコン (英語字幕あり)
  \url{https://www.youtube.com/watch?v=VZ9MmxevNeQ}
\item
  『Speck ドイツ式ベーコン』
  \url{http://jdg-nishinihon.org/upfiles/files/speck.pdf}
\end{itemize}

\hypertarget{ux30b5ux30fcux30ed-ux5b87ux9732-ux441ux430ux43bux43e}{%
\section{\texorpdfstring{サーロ\index{サーロ} (宇/露:
Сало\index{сало|see{サーロ})}}{サーロ (宇/露: Сало}}\label{ux30b5ux30fcux30ed-ux5b87ux9732-ux441ux430ux43bux43e}}

衛生面で問題ないのかは不明.

\begin{tabular}[t]{rl}
\toprule
 & 難易度\\
\midrule
材料調達 & {\fontspec{Noto Sans CJK JP} ★★☆☆☆ }\\
調理 & {\fontspec{Noto Sans CJK JP} ★☆☆☆☆ }\\
\bottomrule
\end{tabular}

\begin{infobox}{important}
発酵・熟成食品は衛生環境に注意して作成してください.

特にこのレシピは生で大丈夫な理由がよくわかりません.

\end{infobox}

\hypertarget{ux6750ux6599-48}{%
\subsection{材料}\label{ux6750ux6599-48}}

\begin{itemize}
\tightlist
\item
  背脂
\item
  塩
\end{itemize}

\hypertarget{ux4f5cux308aux65b9-67}{%
\subsection{作り方}\label{ux4f5cux308aux65b9-67}}

\begin{enumerate}
\def\labelenumi{\arabic{enumi}.}
\tightlist
\item
  塩をまぶして冷蔵庫で寝かせる

  \begin{itemize}
  \tightlist
  \item
    胡椒等のスパイスをブレンドしてもよい
  \end{itemize}
\end{enumerate}

\hypertarget{ux88dcux8db3-61}{%
\subsection{補足}\label{ux88dcux8db3-61}}

黒パンと一緒に食べる, ウォッカで流し込む, 温かいスープと一緒に食べる.

\hypertarget{ux53c2ux8003ux8cc7ux6599-64}{%
\subsection{参考資料}\label{ux53c2ux8003ux8cc7ux6599-64}}

謎

\hypertarget{ux30adux30e0ux30c1-ux97d3-uxae40uxce58}{%
\section{キムチ (韓:
김치)}\label{ux30adux30e0ux30c1-ux97d3-uxae40uxce58}}

\begin{tabular}[t]{rl}
\toprule
 & 難易度\\
\midrule
材料調達 & {\fontspec{Noto Sans CJK JP} ★★★★☆ }\\
調理 & {\fontspec{Noto Sans CJK JP} ★★★☆☆ }\\
\bottomrule
\end{tabular}

\begin{infobox}{important}
発酵・熟成食品は衛生環境に注意して作成してください

\end{infobox}

\hypertarget{ux6750ux6599-49}{%
\subsection{材料}\label{ux6750ux6599-49}}

\begin{itemize}
\tightlist
\item
  白菜の塩漬け 1株
\item
  エクチョッ (魚醤)
\item
  セウジョッ (アミエビの塩辛など)
\item
  唐辛子の粉末 250 g
\item
  白玉粉 (片栗粉や小麦粉でも可) 50 g
\item
  ニンニク 半株 - 1株
\item
  ショウガ 50 g
\item
  大根またはカブ 適量
\item
  ネギ類や青菜 (大根の葉も可) 適量
\end{itemize}

\hypertarget{ux9053ux5177-44}{%
\subsection{道具}\label{ux9053ux5177-44}}

\begin{itemize}
\tightlist
\item
  白菜漬けの容器
\item
  大根または株を細切りするためのスライサー
\item
  大量のキムチソースを入れる大きめの容器 (かなり大きめのボウルが必要)
\item
  大量のキムチを保管できるタッパー
\end{itemize}

\hypertarget{ux4f5cux308aux65b9-68}{%
\subsection{作り方}\label{ux4f5cux308aux65b9-68}}

\begin{enumerate}
\def\labelenumi{\arabic{enumi}.}
\tightlist
\item
  白菜を塩漬けにしてザワークラウトのように発酵させる
\item
  水に白玉粉を溶いてふやかしを作る
\item
  唐辛子の粉末を混ぜる
\item
  すりおろしたニンニクと生姜を混ぜる
\item
  セウジョッとエクチョッも混ぜる
\item
  すりおろした野菜もよくまぜる
\item
  大根は塩水に浸けて柔らかくすると良い
\item
  白菜の葉1枚1枚にヤンニョムを刷り込む
\item
  清潔な箱に入れ, 冷蔵庫で保管する
\end{enumerate}

\hypertarget{ux88dcux8db3-62}{%
\subsection{補足}\label{ux88dcux8db3-62}}

ヤンニョムや抜けた水分に完全に浸かる状態でないとカビが発生しやすいので注意

\hypertarget{ux9999ux8fa3ux7d05ux6cb9-ux4e2d}{%
\section{香辣紅油 (中)}\label{ux9999ux8fa3ux7d05ux6cb9-ux4e2d}}

\begin{tabular}[t]{rl}
\toprule
 & 難易度\\
\midrule
材料調達 & {\fontspec{Noto Sans CJK JP} ★★★☆☆ }\\
調理 & {\fontspec{Noto Sans CJK JP} ★☆☆☆☆ }\\
\bottomrule
\end{tabular}

ネギ類とシナモンで香りを付けた辣油. 四川料理と合う

\hypertarget{ux5200ux53e3ux8fa3ux6912-ux4e2d}{%
\section{刀口辣椒 (中)}\label{ux5200ux53e3ux8fa3ux6912-ux4e2d}}

麻婆豆腐を始め四川料理に合う

\begin{tabular}[t]{rl}
\toprule
 & 難易度\\
\midrule
材料調達 & {\fontspec{Noto Sans CJK JP} ★★★☆☆ }\\
調理 & {\fontspec{Noto Sans CJK JP} ★☆☆☆☆ }\\
\bottomrule
\end{tabular}

\hypertarget{ux6750ux6599-50}{%
\subsection{材料}\label{ux6750ux6599-50}}

\begin{itemize}
\tightlist
\item
  乾燥唐辛子

  \begin{itemize}
  \tightlist
  \item
    小弾頭とか朝天じゃないとダメと言う人もいるが気にしなくてもいい
  \end{itemize}
\item
  青花椒 (藤椒)

  \begin{itemize}
  \tightlist
  \item
    花椒より高価だがこちらのほうが香りが強い
  \end{itemize}
\item
  油
\end{itemize}

\hypertarget{ux4f5cux308aux65b9-69}{%
\subsection{作り方}\label{ux4f5cux308aux65b9-69}}

\begin{enumerate}
\def\labelenumi{\arabic{enumi}.}
\tightlist
\item
  中華鍋に油を引く
\item
  弱火で唐辛子と青花椒を炒める
\item
  包丁で細かく刻む
\end{enumerate}

\hypertarget{ux88dcux8db3-63}{%
\subsection{補足}\label{ux88dcux8db3-63}}

焦げると味を損なうので加熱時間に注意

\hypertarget{todo-ux8c46ux8c49-ux4e2d}{%
\section{(TODO) 豆豉 (中)}\label{todo-ux8c46ux8c49-ux4e2d}}

四川料理でうま味を加えるのに使う.

ぶっちゃけ買ったほうが安い

\begin{tabular}[t]{rl}
\toprule
 & 難易度\\
\midrule
材料調達 & {\fontspec{Noto Sans CJK JP} ★★★★★ }\\
調理 & {\fontspec{Noto Sans CJK JP} ★★★★★ }\\
\bottomrule
\end{tabular}

\hypertarget{ux53c2ux8003ux8cc7ux6599-65}{%
\subsection{参考資料}\label{ux53c2ux8003ux8cc7ux6599-65}}

\begin{itemize}
\tightlist
\item
  \citet{Jing1992}
\end{itemize}

\hypertarget{ux6ce1ux8fa3ux6912ux6ce1ux59dc-ux4e2d}{%
\section{泡辣椒・泡姜
(中)}\label{ux6ce1ux8fa3ux6912ux6ce1ux59dc-ux4e2d}}

それぞれ唐辛子と生姜を塩水に浸けて乳酸発酵させたもの.
四川料理でよく使う.

\begin{tabular}[t]{rl}
\toprule
 & 難易度\\
\midrule
材料調達 & {\fontspec{Noto Sans CJK JP} ★★☆☆☆ }\\
調理 & {\fontspec{Noto Sans CJK JP} ★★★☆☆ }\\
\bottomrule
\end{tabular}

\begin{infobox}{important}
発酵・熟成食品は衛生環境に注意して作成してください

\end{infobox}

\hypertarget{ux6750ux6599-51}{%
\subsection{材料}\label{ux6750ux6599-51}}

\begin{itemize}
\tightlist
\item
  泡辣椒

  \begin{itemize}
  \tightlist
  \item
    唐辛子
  \end{itemize}
\item
  泡生姜

  \begin{itemize}
  \tightlist
  \item
    生姜
  \end{itemize}
\item
  (共通) ピックル液

  \begin{itemize}
  \tightlist
  \item
    水 適量
  \item
    塩 適量
  \item
    (オプション) 桂皮
  \item
    (オプション) 八角
  \item
    (オプション) 胡椒粒
  \end{itemize}
\end{itemize}

\hypertarget{ux9053ux5177-45}{%
\subsection{道具}\label{ux9053ux5177-45}}

\begin{itemize}
\tightlist
\item
  清潔な密閉できる瓶
\end{itemize}

\hypertarget{ux4f5cux308aux65b9-70}{%
\subsection{作り方}\label{ux4f5cux308aux65b9-70}}

\begin{enumerate}
\def\labelenumi{\arabic{enumi}.}
\tightlist
\item
  水 500cc あたり塩大さじ1を加える
\item
  オプションのスパイスを加えて沸騰させる
\item
  瓶に唐辛子または生姜を入れ, 上記のピックル液を入れて浸す
\item
  冷暗所で\textbf{1ヶ月}ほど寝かせる
\end{enumerate}

\hypertarget{ux305dux306eux4ed6ux88dcux8db3}{%
\chapter*{その他補足}\label{ux305dux306eux4ed6ux88dcux8db3}}
\addcontentsline{toc}{chapter}{その他補足}

\begin{itemize}
\tightlist
\item
  中世編で生姜がしばしば使用されるが,
  当時のヨーロッパには生の生姜はほとんどなく,
  もっぱら乾燥させた粉末だったと思われる.
\item
  華北の文化が主だろうが, \citet{zuien} の記述にも中華料理のヒントがある
\end{itemize}


% --- bibliography settings ---
% bibtex (natbib) mode
  \addcontentsline{toc}{chapter}{\bibname}
  \bibliography{Cooking.bib}



\printindex

\end{document}
